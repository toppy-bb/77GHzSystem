\documentclass[uplatex,a4paper,10pt]{jsarticle}
\usepackage{amsmath,amsthm,amssymb}
\usepackage{multicol}
\usepackage{url}
\usepackage[subrefformat=parens]{subcaption}
\captionsetup{compatibility=false}
\usepackage[dvipdfmx]{graphicx}
%\usepackage[headsep=10pt, head=30pt,foot=10pt]{geometry}
\usepackage{epsf}
\usepackage{bm}
\setlength{\columnseprule}{0.3pt}
\begin{document}
\begin{center}
\vspace*{3cm} \underline{\HUGE 卒業論文中間報告 }\\
\vspace{1cm}
\bf{ \Huge 四元数リザバーコンピューティングに基づく
偏波リモートセンシングによるヒトの動作の分類\\}
\vspace{3cm}
\huge 2022年9月30日提出 \\
\vspace{3cm}
\end{center}
\begin{minipage}{0.4\hsize}
\hspace{1zw}
\end{minipage}
\begin{center}
\begin{minipage}{0.7\hsize}
{\huge 指導教員   廣瀬 明 教授\\ 夏秋 嶺 准教授}
\vspace{1cm}\\
\centering
{\huge 電気電子工学科\\}
{\huge 03-210478 上野 俊樹}
\end{minipage}
\end{center}


\newpage
\tableofcontents

\newpage 

\section{序論}
\subsection{背景}
近年,電磁波を用いたセンシング技術は医療,セキュリティなど
様々な分野において利用されている.
例えば,自宅や病院にいる高齢者,患者が安全に過ごすための遠隔システムなどに応用される.
電磁波を使って非接触で生体計測を行うことで,身体の異常検知を行うものである.
可視光によるイメージングも多く使用されているが,
電磁波による観測には多くのメリットが存在する.

第一に,電磁波を使用することでプライバシーを保護することができる.
身体の異常検知において,対象の観測が長期間に及ぶ.
カメラでの観測は個人のプライバシー面において不快に思うものも多いだろう.
その点,電磁波ではプライバシー面の問題が生じない.

第二に,電磁波による観測は,明暗によらず,また観測範囲が広い.
カメラであれば,夜間の観測は困難である.
また,カメラは範囲外の可視光を捉えられなかったり,
障害物があれば,その後ろの観測を行えない.
一方,電磁波であれば,障害物を回折し,見えないものの観測も可能である.

以上のように,電磁波によるセンシングは有効であるが,非接触型であるために,
精度に欠けるなど,課題が残されている.

\subsection{目的}
以上から,電磁波を用いてセンシングを行い,ヒトの動作を分類することは.医療分野,
セキュリティー分野において重要である.
そこで,本研究では,四元数リザバーコンピューティングを提案し,その有用性を示す.

\section{関連技術}
\subsection{偏波とポアンカレ球}
偏波の状態は,ストークスベクトル$\bm{g}$により,
ポアンカレ球で表現できる.4つのストークスパラメータを
$g_0,g_1,g_2,g_3$とすると,
\begin{align}
    \bm{g} = 
      \left(
        \begin{array}{c}
            g_0 \\
            g_1 \\
            g_2 \\
            g_3
        \end{array}
      \right)
    = \left(
        \begin{array}{c}
            |E_{\rm{H}}^{\rm{r}}|^2 + |E_{\rm{V}}^{\rm{r}}|^2 \\
            |E_{\rm{H}}^{\rm{r}}|^2 - |E_{\rm{V}}^{\rm{r}}|^2 \\
            2\mathrm{Re}((E_{\rm{H}}^{\rm{r}})^*E_{\rm{V}}^{\rm{r}}) \\
            2\mathrm{Im}((E_{\rm{H}}^{\rm{r}})^*E_{\rm{V}}^{\rm{r}})
        \end{array}
      \right)
\end{align}
とかける.ここで,$E_{\rm{H}}^{\rm{r}}$は受信した偏波の水平成分,
$E_{\rm{V}}^{\rm{r}}$は受信した偏波の垂直成分である.
また,$g_0$は受信した全電力,$g_1$は水平偏波成分と垂直偏波成分の電力差,
$g_2$は$+45^\circ$偏波成分と$-45^\circ$偏波成分の電力差,
$g_3$は左回り円偏波成分(left-handed circle: LHC)
と右回り円偏波成分(right-handed circle:RHC)の電力差である.
ストークスパラメータは全て実数であり,受信電力の測定から与えられる.
さらに,
\begin{align*}
    &g_1^2 + g_2^2 + g_3^2 \\
    &= (|E_{\rm{H}}^{\rm{r}}|^2 - |E_{\rm{V}}^{\rm{r}}|^2)^2
      + (|E_{+45^\circ}^{\rm{r}}|^2 - |E_{-45^\circ}^{\rm{r}}|^2)^2
      + (|E_{\mathrm{LHC}}^{\rm{r}}|^2 - |E_{\mathrm{RHC}}^{\rm{r}}|^2)^2\\
    &= (|E_{\rm{H}}^{\rm{r}}|^2 - |E_{\rm{V}}^{\rm{r}}|^2)^2
      + ((E_{\rm{H}}^{\rm{r}})^*E_{\rm{V}}^{\rm{r}} + (E_{\rm{V}}^{\rm{r}})^*E_{\rm{H}}^{\rm{r}})^2
      + (-j)((E_{\rm{H}}^{\rm{r}})^*E_{\rm{V}}^{\rm{r}} - (E_{\rm{V}}^{\rm{r}})^*E_{\rm{H}}^{\rm{r}})\\
    &= |E_{\rm{H}}^{\rm{r}}|^4 - 2|E_{\rm{H}}^{\rm{r}}|^2|E_{\rm{V}}^{\rm{r}}|^2 + |E_{\rm{V}}^{\rm{r}}|^4 + 4|E_{\rm{H}}^{\rm{r}}E_{\rm{V}}^{\rm{r}}|^2\\
    &= |E_{\rm{H}}^{\rm{r}}|^4 + 2|E_{\rm{H}}^{\rm{r}}|^2|E_{\rm{V}}^{\rm{r}}|^2 + |E_{\rm{V}}^{\rm{r}}|^4\\
    &= (|E_{\rm{H}}^{\rm{r}}|^2 + |E_{\rm{V}}^{\rm{r}}|^2)^2\\
    &= g_0^2
\end{align*}
が成り立つので,$g_1$,$g_2$,$g_3$を$g_0$で割ると,
3つのパラメータから得られる座標は,単位球上の点になる.
この3つのパラメータからポアンカレベクトル$\bm{P}$は以下のように得られる.
\begin{equation}
    \bm{P} 
    = \left(
        \begin{array}{c}
            g_1/g_0 \\
            g_2/g_0 \\
            g_3/g_0 \\
        \end{array}
      \right)
    = \left(\begin{array}{c}
            x \\
            y \\
            z \\
        \end{array}
      \right)
\end{equation}

ポアンカレベクトル$\bm{P}$は図\ref{1}のようにポアンカレ球で表現できる.
これは偏波の状態を視覚化するのに便利な表現方法である.


\subsection{四元数}
四元数は,複素数を拡張した数である.
実数$a,b,c,d$,虚数単位を$i,j,k$とすると,$q = a + bi + cj  + dk$
とかける.
ポアンカレベクトル$\bm{P}$を四元数で扱うと,実部を0,虚部を$\bm{P}$の要素として
次のように表現できる.
\begin{equation}
    \bm{q} = \left(\begin{array}{c}
        0\\
        g_1/g_0 \\
        g_2/g_0 \\
        g_3/g_0 \\
    \end{array}
  \right)
\end{equation}

四元数は3次元空間における回転と相性が良いため,ポアンカレベクトルを四元数に拡張
することで学習がうまく進みやすい.


\subsection{リザバーコンピューティング}
%まずはオーソドックスなニューラルネットワークについて説明する.この場合は図\ref{fig:neural_network}に示すようにニューロンが層状に並んでいて信号が入力端子から出力端子に向かって一方向に流れていく.各ニューロンに入力された信号たちベクトルを$\bm{x}_{\rm{in}}$そのニューロンに設定された重みという数を要素とする行列を$\mathbf{W}$そして活性化関数$f_{\rm{a}}$とすると各ニューロンの出力$\bm{x}_{\rm{out}}$は次の式で決定される.
%\begin{equation}
    %\bm{x}_{\rm{out}} = f_{\rm{a}}(\mathbf{W} \bm{x}_{\rm{in}})
%\end{equation}
%\begin{figure}[hbtp]
%	\centering
%	\includegraphics[width=110mm]{../img/neural_network.pdf}
%	\caption{ニューラルネットワーク}
%	\label{fig:neural_network}
%\end{figure}

%\begin{figure}[hbtp]
%	\centering
%	\includegraphics[width=110mm]{../img/Echo_State_Network.pdf}
%	\label{fig:echo_state}
%\end{figure}
%それに対して図%\ref{fig:echo_state}
%に示すように 
リザバーコンピューティングは時系列情報処理に適した機械学習の枠組みの一つであり,
学習が極めて高速であるという点が特徴である.
まず,オーソドックスなニューラルネットワークについて説明する.
この場合は図\ref{fig:neural_network}に示すように
ニューロンが層状に並んでいて信号が入力端子から出力端子に向かって一方向に流れていく.
各ニューロンの入力信号ベクトルを$\bm{x}_{\rm{in}}$,
そのニューロンに設定された重みを要素とする行列を
$\bf{W}$,
そして活性化関数を$f_{\rm{a}}$とすると,
各ニューロンの出力$\bm{x}_{\rm{out}}$は次の式で決定される.
\begin{equation}
    \bm{x}_{\rm{out}} = f_{\rm{a}}(\mathbf{W} \bm{x}_{\rm{in}})
\end{equation}
入力端子への入力$\bm{s}_{\rm{in}}$に対応した正解である教師データを
$\bm{t}_{\rm{out}}$,ネットワークの出力層が出力したデータを
$\bm{s}_{\rm{out}}$とすると,$\bm{t}_{\rm{out}}$と${\bm{s}_{\rm{out}}}$
の平均二乗誤差が最小となるように誤差逆伝搬ですべての重み
$\mathbf{W}$を更新していく.この更新作業が学習にあたる.

一方,図\ref{fig:echo_state}に示すように
リザバーコンピューティングシステムの中間層に相当する部分は層状になっていない.
また信号の流れも一方向ではなくフィードバックがかかっている.
このフィードバックのため,過去の信号の影響が残る.
これが記憶を持つことに相当し,時系列データ処理の可能性を生む.
リザバーコンピューティングシステムは
さらに,入力端子信号にかかる重み$\mathbf{W_{\rm{in}}}$と
中間層相当の内部で相互にやりとりする信号にかける重み
$\mathbf{W_{\rm{res}}}$は
学習によって更新されないという特徴をもつ.
この中間層相当にあたる部分はリザバー(reservoir:ため池)と呼ばれる.
更新されるのは中間層から出力層へ信号が流れるときにかかる重み
$\mathbf{W}_{\rm{out}}$のみである.

入力層の信号を$\bm{s}_{\rm{in}}$,リザバー部の各ニューロンの出力信号を$\bm{x}$,
出力層における出力信号を$\bm{s}_{\rm{out}}$とし,
離散時間$n$におけるリザバー部の出力信号を$\bm{x}(n)$と記述すると
\begin{equation}
    \bm{x}(n) = f_{\rm{a}}(\mathbf{W_{\rm{res}}} \bm{x}(n-1) + \mathbf{W}_{\rm{in}} \bm{s}_{\rm{in}}(n))
\end{equation}

\begin{equation}
    \bm{s}_{\rm{out}}(n) = f_{\rm{a}}(\mathbf{W}_{\rm{out}} \bm{x}(n))
\end{equation}

これらの式よりリザバー部の出力信号と出力層の出力信号が求められる.

$\bm{s}_{\rm{in}}(n)$に対応する教師データを$\bm{t}(n)$とする.
$n$を$1$から$n_{\rm{end}}$の範囲で考えて$\mathbf{T}=(\bm{t}(1), \bm{t}(2), \cdots \bm{t}(n_{\rm{end}}))$としさらに$\mathbf{X}=(\bm{x}(1), \bm{x}(2), \cdots \bm{x}(n_{\rm{end}}))$とすると
\begin{equation}
    \mathbf{W}_{\rm{out}} \mathbf{X} = f_{\rm{a}}^{-1}(\mathbf{T})
\end{equation}
となるように$\mathbf{W}_{\rm{out}}$を学習すればよい.すなわち
\begin{equation}
    \mathbf{W_{out}}=f_{\rm{a}}^{-1}(\mathbf{T}) \mathbf{X}_{\rm{pinv}}
\end{equation}
ただし$\mathbf{X}_{\rm{pinv}}$は$\mathbf{X}$の擬逆行列である.更新,学習するのは$W_{\rm{out}}$だけなので高速な学習が期待できる.


\section{四元数リザバーコンピューティングに基づく偏波リモートセンシングの提案}

\subsection{システムの構成}
図\ref{fig:system}は本研究で用いるシステムのフロントエンドである.
また,このシステムのブロックダイアグラムを図\ref{fig:diagram}に示す.

送信アンテナTx1で垂直偏波を照射し,観測対象で散乱された偏波の垂直成分と水平成分を
それぞれ,受信アンテナRx1,Rx2で同時に受信する.次に,
送信アンテナTx2で垂直偏波を照射し,観測対象で散乱された偏波の垂直成分と水平成分を
それぞれ,受信アンテナRx1,Rx2で同時に受信する.
この送受信を高速で繰り返すことで測定をする.
送信アンテナの切り替えはマイクロコンピュータで制御する.
得られた受信偏波はIQ信号として得られ,ここからA/D変換を行ったのち,
ストークスベクトルなどを計算する.

使用する電磁波は77GHzのミリ波である.ミリ波は,直進性,分解能が高いという性質がある.
人体に対して無害であるため,生体に対する計測にも多く用いられている.

\subsection{四元数リザバーコンピューティング}



\section{実験}


\section{今後の予定}


\newpage 
\begin{thebibliography}{99}
		
	\bibitem{echostate} JAEGER, Herbert. The “echo state” approach to analysing and training recurrent neural networks-with an erratum note. Bonn, Germany: German National Research Center for Information Technology GMD Technical Report, 2001, 148.34: 13.
	
	\bibitem{physical_reservoir} TANAKA, Gouhei, et al. Recent advances in physical reservoir computing: A review. Neural Networks, 2019, 115: 100-123.
	
	\bibitem{spin_wave_reservoir} NAKANE, Ryosho; TANAKA, Gouhei; HIROSE, Akira. Reservoir computing with spin waves excited in a garnet film. IEEE Access, 2018, 6: 4462-4469.
	
	\bibitem{autoencoder} HINTON, Geoffrey E.; SALAKHUTDINOV, Ruslan R. Reducing the dimensionality of data with neural networks. science, 2006, 313.5786: 504-507.
	
	\bibitem{ad_by_autoencoder_space} SAKURADA, Mayu; YAIRI, Takehisa. Anomaly detection using autoencoders with nonlinear dimensionality reduction. In: Proceedings of the MLSDA 2014 2nd Workshop on Machine Learning for Sensory Data Analysis. 2014. p. 4-11.
	
	\bibitem{ad_by_autoencoder_space_in_japanese} 櫻田麻由; 矢入健久. オートエンコーダを用いた次元削減による宇宙機の異常検知. In: 人工知能学会全国大会論文集 第 28 回全国大会 (2014). 一般社団法人 人工知能学会, 2014. p. 2F32-2F32.
	
	\bibitem{ad_by_autoencoder_machine_in_japanese} 峯誉明, et al. オートエンコーダを用いた機械装置の異常検知. In: 日本知能情報ファジィ学会 ファジィ システム シンポジウム 講演論文集 第 35 回ファジィシステムシンポジウム. 日本知能情報ファジィ学会, 2019. p. 506-508.
	    
	\bibitem{dcase2020}
	    \url{http://dcase.community/challenge2020/task-unsupervised-detection-of-anomalous-sounds}
	
	\bibitem{dataset_toy} KOIZUMI, Yuma, et al. ToyADMOS: A dataset of miniature-machine operating sounds for anomalous sound detection. In: 2019 IEEE Workshop on Applications of Signal Processing to Audio and Acoustics (WASPAA). IEEE, 2019. p. 313-317.
	
	\bibitem{dataset_machine} PUROHIT, Harsh, et al. MIMII dataset: Sound dataset for malfunctioning industrial machine investigation and inspection. arXiv preprint arXiv:1909.09347, 2019.
	
	\bibitem{ecg200}
	    \url{http://www.timeseriesclassification.com/description.php?Dataset=ECG200}

	\bibitem{esn_ae} CHOUIKHI, Naima, et al. Bi-level multi-objective evolution of a Multi-Layered Echo-State Network Autoencoder for data representations. Neurocomputing, 2019, 341: 195-211.\\
	    \url{https://codeocean.com/capsule/1279641/tree/v2}
	
	\bibitem{esn_ae_simple} CHOUIKHI, Naima; AMMAR, Boudour; ALIMI, Adel M. Genesis of basic and multi-layer echo state network recurrent autoencoders for efficient data representations. arXiv preprint arXiv:1804.08996, 2018.\\
	    \url{https://codeocean.com/capsule/6293716/tree/v2}
	    

\end{thebibliography}

\end{document}