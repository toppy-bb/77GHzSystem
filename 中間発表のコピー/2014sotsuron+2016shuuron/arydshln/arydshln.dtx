% \iffalse
%% Style file `arydshln'
%% Copyright (C) 1993,1998-2016
%%       Hiroshi Nakashima <h.nakasima'AT'media'DOT'kyoto-u'DOT'ac'DOT'jp>
%%       (Kyoto Univ.)
%% This program can be redistributed and/or modified under the terms
%% of the LaTeX Project Public License Distributed from CTAN
%% archives in directory macros/latex/base/lppl.txt; either
%% version 1 of the License, or any later version.
% \fi
%
%\CheckSum{3618}
%% \CharacterTable
%%  {Upper-case    \A\B\C\D\E\F\G\H\I\J\K\L\M\N\O\P\Q\R\S\T\U\V\W\X\Y\Z
%%   Lower-case    \a\b\c\d\e\f\g\h\i\j\k\l\m\n\o\p\q\r\s\t\u\v\w\x\y\z
%%   Digits        \0\1\2\3\4\5\6\7\8\9
%%   Exclamation   \!     Double quote  \"     Hash (number) \#
%%   Dollar        \$     Percent       \%     Ampersand     \&
%%   Acute accent  \'     Left paren    \(     Right paren   \)
%%   Asterisk      \*     Plus          \+     Comma         \,
%%   Minus         \-     Point         \.     Solidus       \/
%%   Colon         \:     Semicolon     \;     Less than     \<
%%   Equals        \=     Greater than  \>     Question mark \?
%%   Commercial at \@     Left bracket  \[     Backslash     \\
%%   Right bracket \]     Circumflex    \^     Underscore    \_
%%   Grave accent  \`     Left brace    \{     Vertical bar  \|
%%   Right brace   \}     Tilde         \~}
%%
% \iffalse
%%
%<*dtx>
\ProvidesFile{arydshln.dtx}
%</dtx>
%<package>\def\next{LaTeX2e}
%<package>\ifx\fmtname\next\def\next{
%<package>\NeedsTeXFormat{LaTeX2e}[1994/12/01]
%<package>\ProvidesPackage{arydshln}}
%<package>\else\def\next[#1]{}\fi\next
%<driver>\ProvidesFile{arydshln-man.tex}
% \fi
% \ProvidesFile{arydshln.dtx}
[2016/04/28 v1.73 ]
%
% \iffalse
%<*driver>
\documentclass{ltxdoc}
\usepackage{arydshln}
\DisableCrossrefs
\PageIndex
\CodelineNumbered
\RecordChanges
%</driver>
%<driver>\OnlyDescription
%<driver>\def\ONLYDESCRIPTION{}
%<*driver>
\advance\textwidth4em
\begin{document}
\DocInput{arydshln.dtx}
\end{document}
%</driver>
% \fi
% \MakeShortVerb{\|}
% \GetFileInfo{arydshln.dtx}
% \IndexPrologue{\section*{Index}}
% \setcounter{IndexColumns}{2}
% \def\th{^{\it th}}
% \def\LOR{{\ {\lor}\ }}
% \def\LAND{{\ {\land}\ }}
% \let\togets\leftrightarrow
% \let\epsilon\varepsilon
% \def\cs#1{\hbox{\tt\char\escapechar#1}}
%
% \makeatletter
% \def\Meta#1{\penalty\@highpenalty \hbox{{\tt\char`\{}\meta{#1}{\tt\char`\}}}}
%
% \def\eqnarray{\stepcounter{equation}\let\@currentlabel=\theequation
% \global\@eqnswtrue
% \global\@eqcnt\z@\tabskip\@centering\let\\=\@eqncr
% $$\halign to \displaywidth\bgroup\hskip\@centering
%   $\displaystyle\tabskip\z@{##}$\@eqnsel&\global\@eqcnt\@ne
%   \hfil$\;{##}\;$\hfil
%   &\global\@eqcnt\tw@ $\displaystyle\tabskip\z@{##}$\hfil
%    \tabskip\@centering&\llap{##}\tabskip\z@\cr}
%
% \let\latex@addmarginpar\@addmarginpar
% \let\latex@latex@warning@no@line\@latex@warning@no@line
% \def\@addmarginpar{\let\@latex@warning@no@line\@gobble
%	\latex@addmarginpar
%	\let\@latex@warning@no@line\latex@latex@warning@no@line}
% \let\@font@warning\@gobble
%
% \long\def\m@cro@#1#2{\endgroup \topsep\MacroTopsep \trivlist
%    \edef\saved@macroname{\string#2}
%   \def\makelabel##1{\llap{##1\hskip\@totalleftmargin}}
%   \if@inlabel
%     \let\@tempa\@empty \count@\macro@cnt
%     \loop \ifnum\count@>\z@
%       \edef\@tempa{\@tempa\hbox{\strut}}\advance\count@\m@ne \repeat
%     \edef\makelabel##1{\llap{\vtop to\baselineskip
%                                {\@tempa\hbox{##1}\vss}
%			       \hskip\@totalleftmargin}}
%     \advance \macro@cnt \@ne
%   \else  \macro@cnt\@ne  \fi
%   \edef\@tempa{\noexpand\item[
%      #1
%        \noexpand\PrintMacroName
%      \else
%        \noexpand\PrintEnvName
%      \fi
%      {\string#2}]}
%   \@tempa
%   \global\advance\c@CodelineNo\@ne
%    #1
%       \SpecialMainIndex{#2}\nobreak
%       \DoNotIndex{#2}
%    \else
%       \SpecialMainEnvIndex{#2}\nobreak
%    \fi
%   \global\advance\c@CodelineNo\m@ne
%   \ignorespaces}
%
% \def\DescribeOpt{\leavevmode\@bsphack\begingroup\MakePrivateLetters
%   \Describe@Opt}
% \def\Describe@Opt#1{\endgroup \Describe@pt{#1}}
% \def\Describe@pt#1#2{
%               \marginpar{\raggedleft\PrintDescribeMacro{#1}}
% 		\SpecialOptUsageIndex{#1}{#2}}
% \def\SpecialOptUsageIndex#1#2{
% 	\index{#1\actualchar{\protect\ttfamily#1} (#2)\encapchar usage}
%	\index{#2s:\levelchar{\protect\ttfamily#1}\encapchar usage}
%	\@esphack\ignorespaces}
% \def\SpecialOptMainIndex#1#2{
% 	\index{#1\actualchar{\protect\ttfamily#1} (#2)\encapchar main}
%	\index{#2s:\levelchar{\protect\ttfamily#1}\encapchar main}
%	\@esphack\ignorespaces}
% \def\SpecialOptIndex#1#2{
% 	\index{#1\actualchar{\protect\ttfamily#1} (#2)}
%	\index{#2s:\levelchar{\protect\ttfamily#1}}
%	\@esphack\ignorespaces}
% \def\SpecialOptIndexS#1#2{
% 	\index{#1\actualchar{\protect\ttfamily#1} (#2)}
%	\index{#2es:\levelchar{\protect\ttfamily#1}}
%	\@esphack\ignorespaces}
%
% \def\SpecialMainIndex#1{\@bsphack
%	\RemoveAtFromCS{#1}
% 	\protected@edef\@tempa{\noexpand\special@index{\RAFCtempa
%		\actualchar \string\verb
% 		\quotechar*\verbatimchar\string#1\verbatimchar
%		\encapchar main}}\@tempa
%	\@esphack}
% 
% \def\SpecialIndex#1{\@bsphack
%	\RemoveAtFromCS{#1}
% 	\protected@edef\@tempa{\noexpand\special@index{\RAFCtempa
%		\actualchar \string\verb
% 		\quotechar*\verbatimchar\string#1\verbatimchar}}\@tempa
%	\@esphack}
% 
% \def\Macro{\begingroup \@sanitize \iMacro}
% \def\iMacro#1{\endgroup
%	\def\SpecialMainIndex##1{\SWiIndex{##1}{\encapchar main}}
%	\macro{#1}}
% \let\endMacro\endmacro
% 
% \def\SpecialWeirdIndex{\@bsphack \begingroup \@sanitize \SWIndex{}}
% \def\SWIndex#1#2{\endgroup \SWiIndex{#2}{#1}}
% \def\SWiIndex#1#2{%
%	\protected@edef\@tempa{\noexpand\special@index{
%		#1\actualchar \string\verb
%		\quotechar*\verbatimchar#1\verbatimchar#2}}\@tempa
% 	\@esphack}
% 
% \let\RAFCempty\@empty
% \let\RAFCgobble\@gobble
% \makeatother
% \def\RemoveAtFromCS#1{\edef\RAFCtempa{\expandafter\RAFCgobble\string#1}
% 	\edef\RAFCtempa{\expandafter\string\csname\RAFCtempa\endcsname}
%	\expandafter\iRemoveAtFromCS\RAFCtempa@\NIL}
% \def\iRemoveAtFromCS#1#2@#3\NIL{\def\RAFCtempb{#3}
% 	\ifx\RAFCtempb\RAFCempty \def\RAFCtempa{#2}\let\next\relax
%	\else \def\RAFCtempa{#2@}\def\next{\iiRemoveAtFromCS#3\NIL}
% 	\def\RAFCtempb{#2}
%	\ifx\RAFCtempb\RAFCempty
%	\else\ifx\RAFCtempb\CommonCSPrefix
%	\else\ifx\RAFCtempb\CommonIfPrefix
%	\else \def\RAFCtempa{}\def\next{\xRemoveAtFromCS#2@#3\NIL}\fi\fi\fi\fi
%	\next}
% \def\iiRemoveAtFromCS#1{\ifx#1@\edef\RAFCtempa{\RAFCtempa@}
%	\let\next\iiRemoveAtFromCS
%	\else \def\next{\xRemoveAtFromCS#1}\fi
% 	\next}
% \def\xRemoveAtFromCS#1@\NIL{\edef\RAFCtempa{#1\RAFCtempa}}
% \makeatletter
% 
% \begingroup
% \gdef\@sverbcat{\catcode`\@\active\catcode`\#\active}
% \catcode`\@\active \gdef@{\char`\@\penalty\exhyphenpenalty}
% \catcode`\#\active \gdef#{\penalty\exhyphenpenalty}
% \endgroup
% \let\latex@sverb\@sverb
% \def\@sverb{\@sverbcat\latex@sverb}
%
% \newcount\ChangesCounter \ChangesCounter10000
% \def\changes@#1#2#3{\global\advance\ChangesCounter\@ne
% 	\protected@edef\@tempa{\noexpand\glossary{#1\levelchar
%		x\number\ChangesCounter
%		\ifx\saved@macroname\@empty
%			\actualchar\generalname
%		\else
%			\expandafter\@gobble\saved@macroname
%			\actualchar \string\verb
%			\quotechar*\verbatimchar \saved@macroname
%			\verbatimchar
%		\fi
%		:\levelchar #3}}\@tempa\endgroup\@esphack}
% \def\theglossary{\onecolumn \glossary@prologue
%	\GlossaryParms \let\item\@idxitem \ignorespaces}
% \def\endtheglossary{\relax}
% 
% \let\EQ\relax \let\GT\relax \let\BAR\relax \let\NEQ\relax
% 
% \makeatother
% \errorcontextlines1000
%
% 
% 
% \changes{v1.0}{1993/04/01}
%	{The style was born on a good day \ldots\ (1993/04/01)}
% \changes{v1.05}{1993/06/18}
%	{Cope with \texttt{\char`\\\char`\\} with negative optional
%	vertical space. (1993/06/18)}
% \changes{v1.1}{1993/06/24}
%	{Save and restore the \texttt{\char`\\catcode} for
%	`\texttt{@}'. (1993/06/24)}
% 
% \changes{v1.2}{1998/07/16}
% 	{Various changes shown below. (1998/07/16)}
% \changes{v1.2-1}{1998/07/16}
%	{Add this document.}
% \changes{v1.2-2}{1998/07/16}
%	{Cope with \protect\LaTeXe.}
% \changes{v1.2-3}{1998/07/16}
% 	{Allow mixture of vertical solid- and dash-lines.}
% \changes{v1.2-4}{1998/07/16}
% 	{Add the feature of explicit dash/gap specification.}
% \changes{v1.2-5}{1998/07/16}
% 	{Fix some bugs and change codes.}
% 
% \changes{v1.3}{1998/10/08}
% 	{Fix one bug shown below. (1998/10/08)}
% 
% \changes{v1.4}{1999/06/25}
%	{Make compatible with array package and add new features. (1999/06/25)}
% 
% \changes{v1.4-1}{1999/06/25}
%	{The following are changes of this document.}
% 
% \changes{v1.4-2-1}{1999/06/25}
% 	{The following are for the general compatibility with \textsf{array}.}
% \changes{v1.4-2-2}{1999/06/25}
% 	{The following are to control the effective width of vertical lines.}
% \changes{v1.4-2-3}{1999/06/25}
% 	{The following are for inactivation of dash-line functions.}
% \changes{v1.4-2-4}{1999/06/25}
% 	{The following are for drawing mode to cope with the bug of
%	\cs{xlearders}.}
% \changes{v1.4-2-5}{1999/06/25}
%	{The following are to implement dashed version of \cs{firsthline}
%	and \cs{lasthline} of \textsf{array}.}
% \changes{v1.4-2-6}{1999/06/25}
%	{The following are to fix the bug by which the depth of
%	\texttt{array}\slash\texttt{tabular} was always zero.}
% \changes{v1.4-2-7}{1999/06/25}
%	{The following are to rename macros for \cs{cdashline}.}
% \changes{v1.4-2-8}{1999/06/25}
%	{The following are to cope with very narrow or negative wide columns.}
% \changes{v1.4-3}{2000/07/04}
%	{Released to CTAN on 2000/07/04.}
%
% \changes{v1.5}{2000/07/12}
%	{Make compatible with \textsf{colortab}, and fix bugs. (2000/07/12)}
% \changes{v1.5-1}{2000/07/12}
%	{The following are for the compatibility with \textsf{colortab}.}
% \changes{v1.5-2}{2002/06/26}
%	{The following are for bug fix of \cs{adl@putlrc}.}
% \changes{v1.5-3}{2003/07/11}
%	{The following are for bug fix of \cs{adl@inactivate}.}
% 
% \changes{v1.54}{2003/08/25}
% 	{Bug fixes. (2003/08/25)}
% \changes{v1.54-1}{2003/08/25}
%	{The following are for bug fix of \cs{adl@@vl}.}
% \changes{v1.54-2}{2003/08/25}
%	{The following are for bug fix of activation.}
% \changes{v1.54-3}{2003/08/25}
%	{The following are miscellaneous modifications.}
%
% \changes{v1.6}{2003/08/25}
%	{The following are for the compatibility with
%	 \textsf{longtable}. (2003/08/25)}
%
% \changes{v1.7}{2004/05/21}
%	{The following are for the compatibility with
%	 \textsf{colortbl}. (2004/05/21)}
% \changes{v1.7-1}{2004/5/21}
%	{The following are for null-wide horizontal lines.}
% \changes{v1.7-2}{2004/5/21}
%	{The following are to fix the bug of \cs{arrayrulecolor} etc. in
%	\textsf{colortbl}.}
% \changes{v1.7-3}{2004/5/21}
% 	{The following are for vertical line coloring.}
%
% \changes{v1.71}{2004/7/31}
% 	{The following are for bug fix for \textsf{array}'s
% 	\texttt{m}-columns. (2004/7/31)}
% 
% \changes{v1.72}{2016/03/19}
% 	{Bug fix and revision of \S2.4. (2016/03/19)}
% \changes{v1.72-1}{2016/03/19}
% 	{The followings are for bug fix for footnotes in \textsf{longtable}'s
% 	\texttt{m}-columns.}
% 
% \changes{v1.73}{2016/04/28}
% 	{Bug fix.  (2016/04/28)}
% \changes{v1.73-1}{2016/04/28}
% 	{The followings are to fix the problem that the top edge a vertical
%	 (dash-)line is at the bottom of a horizontal line rather than it
%	 top.} 
% \changes{v1.73-2}{2016/04/28}
% 	{The followings are to fix the bug that \cs{hdashline}
%	 is not properly processed in a \texttt{array}\slash
%	 \texttt{tabular} environment if \textsf{longtable} is loaded.}
% 
% 
%
% \def\EXCLAMATION{\texttt{!}}
% \def\VERTBAR{\texttt{\char`\|}}
% \edef\CommonCSPrefix{\expandafter\RAFCgobble\string\adl}
% \edef\CommonIfPrefix{\expandafter\RAFCgobble\string\ifadl}
% 
% \title{The \textsf{arydshln} package\thanks{
%	This file has version number \fileversion, last revised \filedate.}}
% \author{Hiroshi Nakashima\\(Kyoto University)}
% \date{\filedate}
% \maketitle
%
% \begin{abstract}
% This file gives \LaTeX's |array| and |tabular| environments the capability
% to draw horizontal\slash vertical dash-lines.
% \end{abstract}
%
% \tableofcontents
% \newpage
%
% 
%
% \section{Introduction}
% \label{sec:intro}
%
% In January 1993, Weimin Zhang kindly posted a style \textsf{hvdashln}
% written by the author, which draws horizontal\slash vertical dash-lines in
% \LaTeX's |array| and |tabular| environments, to the news group
% \textsf{comp.text.tex}.  The style, unfortunately, has a known problem
% that vertical lines are broken when an array contains tall rows.
%
% In March of the year, Monty Hayes complained of this problem encouraging
% the author to make a new version \textsf{arydshln} to solve the problem.
% The new style also has new features, such as allowing `|:|' to specify a
% vertical dash-line in preamble, and |\cdashline| being a counterpart of
% |\cline|.
%
% \changes{v1.4-1}{1999/06/25}
%	{The history on the compatibility with \textsf{array} package.}
% In March 1999, Sebastian Rahtz kindly invited the style, which had been
% improved following the bug report from Takahiro Kubota, to be included in
% \TeX{} CTAN and also in the online catalogue compiled by Graham Williams.
% This invitation gave the style new users including Peter Ehrbar who wished 
% to use it with \textsf{array} style in Standard \LaTeX{} Tools Bundle and
% had trouble because these styles were incompatible with each other.
% Therefore, the style became compatible with \textsf{array} and got
% additional new features.
% 
% \changes{v1.5-1}{2000/7/12}
%	{The history on the compatibility with \textsf{colortab} package.}
% In February 2000, Zsuzsanna Nagy reported that \textsf{arydshln} is not
% compatible with \textsf{colortab} style to let the author work on the
% compatibility issue again.
% 
% \changes{v1.6}{2003/8/25}
%	{The history on the compatibility with \textsf{longtable} package.}
% In Feburary 2001, Craig Leech reported another compatibility problem with
% \textsf{longtable}.  Although the author promised that the problem would
% be attacked some day, the issue had left long time\footnote{
%
% Two years and a half!  Sorry Craig.}
%
% until three other complaints were made.  Then the author attacked the
% problem hoping it is the last compatibility issue\footnote{
%
% But his hope was dashed as described below.}.
%
% \changes{v1.7}{2004/5/21}
%	{The history on the compatibility with \textsf{colortbl} package.}
% In May 2004, Klaus Dalinghaus found another incompatibility with
% \textsf{colortbl}. Although he was satisfied by a quick hack for cell
% painting, the author attacked a harder problem for line
% coloring to solve the problem\footnote{
%
% Without dreaming it is the last compatibility issue.}.
%
% 
%
% \section{Usage}
% \label{sec:usage}
% \subsection{Loading Package}
% \label{sec:usage-load}
% 
% \changes{v1.4-1}{1999/06/25}
%	{Explanation of package loading is added.}
% 
% The package is usable to both \LaTeXe{} and \LaTeX-2.09 users with their
% standard package loading declaration.  If you use \LaTeXe, simply do the
% following.
% \begin{quote}
% \SpecialUsageIndex{\usepackage}
% |\usepackage{arydshln}|
% \end{quote}
% If you still love \LaTeX-2.09, the following is what you have to do.
% \begin{quote}
% \SpecialUsageIndex{\documentstyle}
% |\documentstyle[..,arydshln,...]|\Meta{style}
% \end{quote}
%
% \changes{v1.5-1}{2000/7/12}
%	{Caution about loading order of \textsf{colortab} is added.}
% \changes{v1.6}{2003/8/25}
%	{Caution about loading order of \textsf{longtable} is added.}
% \changes{v1.7}{2004/5/21}
%	{Caution about loading order of \textsf{colortbl} is added.}
% \SpecialOptIndex{array}{package}
% \SpecialOptIndex{colortab}{package}
% \SpecialOptIndex{longtable}{package}
% \SpecialOptIndex{colortbl}{package}
%
% Only one caution given to users of \textsf{array} (v2.3m or later) and
% \textsf{longtable} (v4.10 or later) packages, included in Standard \LaTeX{}
% Tools Bundle, and \textsf{colortab} and \textsf{colortbl} package is that
% \textsf{arydshln} has to be loaded {\em after} \textsf{array},
% \textsf{longtable}, \textsf{colortab} and/or \textsf{colortbl} done.  That
% is, the following is correct but reversing the order of |\usepackage| will
% cause some mysterious error.
%
% \begin{quote}
% |\usepackage{array}     % and/or|\\
% |\usepackage{longtable} % and/or|\\
% |\usepackage{colortab}  % or|\\
% |\usepackage{colortbl}|\\
% |\usepackage{arydshln}|
% \end{quote}
%
% 
% 
% \subsection{Basic Usage}
% \label{sec:usage-basic}
%
% \DescribeEnv{array}\DescribeEnv{tabular}
% \SpecialIndex{\hline}
% \SpecialIndex{\cline}
% \SpecialIndex{\multicolumn}
% You can simply use |array| or |tabular(*)| environments with standard
% preamble, such as \verb!{r|c|ll}!, and standard commands |\\|, |\hline|,
% |\cline| and |\multicolumn|.
%
% \leavevmode\marginpar{\raggedleft\PrintDescribeMacro{:}}
% \index{:={\protect\ttfamily:}|usage}
% Drawing a vertical dash-line is quite simple.  Use `|:|' in the preamble
% as the separator of columns separated by the dash-line, just like using
% `\verb!|!' to draw a vertical solid-line.  The {\em preamble} means not
% only that of the environment, but also the first argument of
% |\multicolumn|.
%
% \DescribeMacro{\hdashline}\DescribeMacro{\cdashline}
% \changes{v1.4-1}{1999/06/25}
%	{Description of \cs{first}\slash\texttt{lasthdashline} is added.}
% It is also simple to draw a horizontal dash-line.  Use |\hdashline| and
% |\cdashline| as the counterparts of |\hline| and |\cline|.
%
% For example;
%
%\begin{quote}
%\begin{verbatim}
%\begin{tabular}{|l::c:r|}\hline
%A&B&C\\\hdashline
%AAA&BBB&CCC\\\cdashline{1-2}
%\multicolumn{2}{|l:}{AB}&C\\\hdashline\hdashline
%\end{tabular}
%\end{verbatim}
%\end{quote}
%
% will produce the following result.
%
%\begin{quote}
%\catcode`\|12
%\begin{tabular}{|l::c:r|}\hline
%A&B&C\\\hdashline
%AAA&BBB&CCC\\\cdashline{1-2}
%\multicolumn{2}{|l:}{AB}&C\\\hdashline\hdashline
%\end{tabular}
%\end{quote}
%
% \changes{v1.73-1}{2016/04/28}
% 	{Add a paragraph describing the perfect contacts of vertical and
%	 horizontal lines.}
%
% Note that the intersections of leftmost/rightmost vertical lines and
% horizontal dash-lines are little bit different from those produced by
% ordinary |array|\slash|tabular|.  That is, with very careful examination
% you will find that vertical lines of ordinary ones are {\em broken\/} with
% small white specks at intersections, while in the example above they have
% no specks.  In addition, the four corners of outermost rectangular also
% have specks in ordinary ones, while those in the example above have
% perfect contacts of L-shape.
%
% \DescribeMacro{\firsthdashline}\DescribeMacro{\lasthdashline}
% If you use \textsf{array}, the dashed version of |\firsthline| and
% |\lasthline| named |\first#hdash#line| and |\lasthdashline| are available.
% 
% 
%
% \subsection{Style Parameters}
% \label{sec:usage-par}
%
% \DescribeMacro{\dashlinedash}\DescribeMacro{\dashlinegap}
% You have two style parameters to control the shape of dash-lines:
% |\dashlinedash| is for the length of each dash segment in a dash line;
% |\dashlinegap| controls the amount of each gap between dash segments.
% Both parameters have a common default value, 4\,pt.
%
% 
%
% \subsection{Fine Tuning}
% \label{sec:usage-fine}
%
% \leavevmode\marginpar{\raggedleft\PrintDescribeMacro{;}}
% \index{;={\protect\ttfamily;}|usage}
% Although you can control the shape of dash-lines in an
% |array|\slash|tabular| environment as described in \S\ref{sec:usage-par},
% you might want to draw a dash-line of a shape different from others.  To
% specify the shape of a vertical dash-line explicitly, you may use;
%
% \begin{quote}
% |;{|\meta{dash}|/|\meta{gap}|}|
% \end{quote}
%
% instead of ordinary `|:|' and will have a dash-line with dash segments of
% \meta{dash} long separated by spaces of \meta{gap}.
%
% \DescribeMacro{\hdashline}\DescribeMacro{\cdashline}
% As for horizontal dash-lines, explicit shape specifications may be given
% through optional arguments of |\hdashline| and |\cdashline| as follows.
%
% \begin{quote}
% |\hdashline[|\meta{dash}|/|\meta{gap}|]|\\
% |\cdashline{|\meta{col1}|-|\meta{col2}|}[|\meta{dash}|/|\meta{gap}|]|
% \end{quote}
%
%
% For example;
%
%\begin{quote}
%\begin{verbatim}
%\begin{tabular}{|l::c;{2pt/2pt}r|}\hline
%A&B&C\\\hdashline[1pt/1pt]
%AAA&BBB&CCC\\\cdashline{1-2}[.4pt/1pt]
%\multicolumn{2}{|l;{2pt/2pt}}{AB}&C\\\hdashline\hdashline
%\end{tabular}
%\end{verbatim}
%\end{quote}
%
% will produce the following result.
%
%\begin{quote}
%\catcode`\|12
%\begin{tabular}{|l::c;{2pt/2pt}r|}\hline
%A&B&C\\\hdashline[1pt/1pt]
%AAA&BBB&CCC\\\cdashline{1-2}[.4pt/1pt]
%\multicolumn{2}{|l;{2pt/2pt}}{AB}&C\\\hdashline\hdashline
%\end{tabular}
%\end{quote}
%
% \changes{v1.4-1}{1999/06/25}
%	{Description of the real width of vertical lines is added.}
% 
% \DescribeMacro{\ADLnullwide}
% \DescribeMacro{\ADLsomewide}
% \SpecialIndex{\arrayrulewidth}
%
% The vertical solid and dashed lines are drawn as if their width is zero, as 
% standard \LaTeX's |array| and |tabular| do, if you don't use
% \textsf{array} package.  Otherwise, they have {\em real} width of
% |\arrayrulewidth| as the authors of \textsf{array} prefers.  However, you
% may explicitly tell \textsf{arydshln} to follow your own preference by
% |\ADLnullwide| if you love \LaTeX{} standard, or |\ADLsomewide| if
% you second the preference of \textsf{array} authors.
% 
% 
%
% \subsection{Finer Tuning}
% \label{sec:usage-finer}
% 
% \changes{v1.4-1}{1999/06/25}
%	{Description of drawing mode is added.}
% \changes{v1.72-2}{1999/06/25}
%	{Revise \S2.4 reflecting the fix of \cs{xleaders}.}
% 
% To draw dash-lines, we use a powerful primitive of \TeX{} called |\xleaders|.
% It replicates a segment that consist of a dash and gap so that a dash-line
% has as many segments as possible and distributes {\it remainder} space to
% make the spaces between adjacent dash segments (almost) equal to each
% other.  Therefore, you will have dash-lines with consistent steps of gaps
% and spaces the lines in Figure~\ref{fig:mode}(1) are.
% 
% \begin{figure}
% \catcode`\|12
% \tabcolsep0pt \setbox\strutbox\hbox{\vrule height14pt depth6pt width0pt}
% \def\mb{\makebox[20pt]}
% \dashlinedash3.01pt \dashlinegap3.01pt
% \def\example#1{\begin{tabular}[b]{|c:c;{2pt/2pt}c|}
%	\hline
%	\mb{A}&\mb{A}&\mb{A}\\[-\arrayrulewidth]\hdashline
%	\mb{B}&\mb{B}&\mb{B}\\[-\arrayrulewidth]\hdashline[2pt/2pt]
%	\mb{C}&\mb{C}&\mb{C}\\[-\arrayrulewidth]\hline
% 	\multicolumn3c{(#1)}
%	\end{tabular}}
%
% \centerline{\example1\qquad
%	\ADLdrawingmode{2}\example2\qquad
%	\ADLdrawingmode{3}\example3}
% \caption{Drawing mode controlled by \cs{ADLdrawingmode}}
% \label{fig:mode}
% \end{figure}
%
% However, because of a bug (or buggy feature) of |\xleaders|, there {\em
% had been} a small possibility that a dash segment near the right\slash
% bottom end drops, until it was fixed in the version of 3.141592\footnote{
% 
% By pointing out this problem, the author got a check of \$327.68 plus a
% significantly large amount of interest from DEK\@.  Wow!!}.
% 
% Though the fix ultimately made any effort to cope with the problem
% unnecessary, the pacakge still gives you alternative {\em drawing modes}
% which you may specify by |\ADLdrawingmode|\Meta{$m$} as follows.
%
% \begin{itemize}
% \item $m=1$
% \DescribeMacro{\ADLdrawingmode}\\
% As shown in Figure~\ref{fig:mode}(1), it gives most beautiful result by
% |\xleaders|\footnote{
% 
% Until the fix of \cs{xleaders}, the second bottom\slash rightmost segments
% of right\slash lower lines were dropped.}.
% 
% This is default.
%
% \item $m=2$\\
% As shown in (2) of the figure, beautiful if dash-lines are not so sparse
% as right/lower lines, but dash segments near the both ends may be a little 
% bit too long as left/upper lines, because in this mode the second
% first\slash last segments are drawn by a special mechanism.
% 
% \item $m=3$\\
% As shown in (3) of the figure, beautiful if dash-lines are not so sparse
% as right/lower lines, but gaps near the both ends may be considerably too
% large as left/upper lines, because in this mode the lines are drawn by
% |\cleaders|.
% \end{itemize}
%
% It is strongly recommended to use default mode 1 unless you want to have
% some special effect.
% 
% 
%
% \subsection{Performance Tuning}
% \label{sec:usage-performance}
% 
% \changes{v1.4-1}{1999/06/25}
%	{Description of (in)activation is added.}
%
% Since drawing dash-lines is a hard job, you have to be patient with the
% fact that the performance of typesetting |array|\slash|tabular| with
% dash-lines is poorer than that of ordinary ones.  In fact, according to
% author's small performance evaluation with a |tabular| having nine
% vertical and ten horizontal dash-lines, typesetting the |tabular| is
% approximately ten times as slow as its ordinary counterpart with solid
% lines.
%
% However, this is not a really bad news, unfortunately.  The real one is
% that loading \textsf{arydshln} makes typesetting |array|\slash|tabular|
% slower even if they only have solid lines which the package treats as
% special ones of dash-lines.  The evaluation result shows the degradation
% factor is about nine.  Therefore, if your document has many
% |array|\slash|tabular| with solid lines, \LaTeX{} will run slowly even 
% with quite few (or no) |array|\slash|tabular| with dash-lines, 
%
% \DescribeMacro{\ADLinactivate}
% To cope with this problem, you may inactivate dash-line functions by the
% command |\ADLinactivate| that replaces dash-lines with solid
% lines drawn by a faster (i.e. ordinary) mechanism.  Although the
% inactivation does not completely solve the performance problem, the
% degradation factor will become much smaller and acceptable, approximately
% 1.5 in the author's evaluation.  For example, the draft version of your
% document will have the command in its preamble, which you will remove from
% your final version.
% 
% \DescribeMacro{\ADLactivate}
% Alternatively, you may do |\ADLinactivate| in the preamble, switch on by
% |\ADLactivate| before you really need dash-lines, and switch off again
% afterword.  A wiser way could be surrounding |array|\slash|tabular| by
% |\begin{ADLactivate}| and |\end{ADLactivate}|.
%
% \DescribeEnv{Array}
% \DescribeEnv{Tabular}
% If you feel it tiresome to type the long command\slash environment name for
% the activation, you may use |Array| and |Tabular(*)| environment in which
% dash-line functions are always active.  Note that, however, since these
% environment names are too natural to keep them from being used by authors
% of other packages or yourself, name conflict could occur.  If |Array|
% and/or |Tabular| have already been defined when \textsf{arydshln} is
% loaded, you will get a warning to show you have to define new
% environments, say |dlarray| and |dltabular|, as follows.
%
% \begin{itemize}\item[]
% |\newenvironment{dlarray}{\ADLactivate\begin{array}}%|\\
% |                        {\end{array}}|\\
% |\newenvironment{dltabular}{\ADLactivate\begin{tabular}}%|\\
% |                          {\end{tabular}}|\\
% |\newenvironment{dltabular*}{\ADLactivate\begin{tabular*}}%|\\
% |                           {\end{tabular*}}|
% \end{itemize}
%
% \begingroup
% \let\PDM\PrintDescribeMacro
% \def\PrintDescribeMacro#1{\leavevmode\llap{\PDM{#1}}}
% \DescribeMacro{\ADLnoshorthanded} 
% On the other hand, if they are defined after \textsf{arydshln} is loaded,
% their definitions are silently replaced or \LaTeX{} complains of multiple
% definitions.  The error in the latter case will be avoided by putting
% |\ADLnoshorthanded| just after |\usepackage{arydshln}|.
% \par
% \endgroup
% 
% 
%
% \subsection{Compatibility with Other Packages}
% \label{sec:usage-compat}
% 
% \changes{v1.5-1}{2000/07/12}
%	{Section 2.7 is added.}
% \changes{v1.4-1}{1999/06/25}
%	{Description of characters and commands of \textsf{array} package is
%	 added.}
% \SpecialOptIndex{array}{package}
% \SpecialIndex{\extrarowheight}
% \SpecialIndex{\firsthline}
% \SpecialIndex{\lasthline}
% \SpecialIndex{\newcolumntype}
%
% Users of \textsf{array} package may use all of newly introduced
% preamble characters, such as `|>|', `|<|', `|m|', `|b|', and all the
% commands such as |\extrarowheight|, |\firsthilne| and |\lasthline|.  The
% preamble characters given by \textsf{arydshln} may be included in the
% second argument of |\newcolumntype|.
%
% \changes{v1.5-1}{2000/07/12}
%	{Description of \textsf{colortab} commands is added.}
% \SpecialOptIndex{colortab}{package}
% \SpecialIndex{\LCC}
% \SpecialIndex{\NAC}
% \SpecialIndex{\ENAC}
% \SpecialIndex{\AC}
% \SpecialIndex{\EAC}
%
% Also users of \textsf{colortab} package may use |\LCC|\slash |\ECC|
% construct to color columns.  A horizontal solid\slash dash line may be
% colored by, e.g.\ |\NAC\hdashline\ENAC|\@.  The pair of |\AC| and |\EAC| may 
% be used to color everything between them {\em but}, unfortunately,
% vertical lines are not.  There are no ways to color vertical lines in a
% table having dash lines.  You may color vertical lines of a ordinary table 
% inactivating dash line functions by |\ADLinactivate|.
% 
% \changes{v1.7}{2004/5/21}
%	{Description of \textsf{colortbl} and related commands is added.}
% \SpecialOptIndex{colortbl}{package}
% \SpecialIndex{\rowcolor}
% \SpecialIndex{\columncolor}
% \SpecialIndex{\cellcolor}
% \SpecialIndex{\arrayrulecolor}
% \SpecialIndex{\doublerulesepcolor}
%
% Another (and more convenient) table coloring tool \textsf{colortbl} may be
% also used simply by loading it before \textsf{arydshln}.  Not only the
% painting commands |\rowcolor|, |\columncolor| and |\cellcolor| work well, 
% but both solid and dash lines are also colored by the command
% |\arrayrulecolor| of \textsf{colortbl}\footnote{
%
% The \textsf{colortbl} manual says \cs{arrayrulecolor} and
% \cs{doublerulesepcolor} may be in \texttt{>\char`\{\ldots\char`\}}
% in a preamble but they cause an error with the original implementation.
% This bug is fixed in \textsf{arydshln} and they are now usable to specify
% the color of the vertical (dash) lines whose specifications occur after
% the commands.}.
%
% One caution is that |\arrayrulecolor| defines the color of the dash-part 
% of dash lines and thus gap-part has no color (i.e. color of the paper on
% which the line drawn).  Therefore, if you have a \textsf{tabular} like;
%
%\begin{quote}
%\begin{verbatim}
%\begin{tabular}{|>{\columncolor{red}}l:>{\colomncolor{green}}r|}
%...
%\end{tabular}
%\end{verbatim}
%\end{quote}
%
% you will find the vertical dash line is a sequence of black (or the color of 
% |\arrayrulecolor|) and white segments.  This problem is partly solved by 
% declaring |\ADLnullwide|\footnote{
%
% Since \textsf{colortbl} automatically loads \textsf{array}, the default is
% \cs{ADLsomewide}}
%
% to conjunct the red and blue columns and to draw the dash line on their
% border.
%
% \DescribeMacro{\ADLnullwidehline}
% \DescribeMacro{\ADLsomewidehline}
% Unfortunately, however, |\ADLnullwide| does not affect the real
% width of horizontal (dash) lines and thus you will still see white gaps
% in |\hdashline| and |\cdashline|.  A solution is to put
% |\ADLnullwidehline| before you start a
% \texttt{array}\slash\texttt{tabular}\footnote{
%
% This command also makes \cs{cline} and \cs{cdashline} visisble even if the
% row below is painted.}.
%
% With this command, a horizontal (dash) line is drawn adjusting its bottom
% edge to that of the row above.  The command |\ADLsomewidehline| turns the
% switch to default and the top edge of a horizontal (dash) line will be
% adjusted to the bottom edge of the row above.
%
% \DescribeMacro{\dashgapcolor}
% \DescribeMacro{\nodashgapcolor}
% \SpecialIndex{\color}
% Another method to avoid white gaps is to give a color to gaps by
% |\dashgapcolor| with arguments same as |\color|.  For example;
%
% \begin{quote}
% |\arrayrulecolor{green}\dashgapcolor[rgb]{1,1,0}|
% \end{quote}
%
% makes colorful dash lines with green dashes and yellow gaps.  The command
% can be placed outside of \texttt{array}\slash\texttt{tabular} for dash
% lines in the environment, in the argument of preamble character |>|
% for vertical dash lines following them, or at the beginning of a row for
% horizontal dash lines following the command.  The commmand
% |\nodashgapcolor| (no arguments) nullifies the effect of |\dashgapcolor|.
% Note that |\nodashgapcolor| is different from |\dashgapcolor{white}|
% because the former makes gaps {\it transparent} while the later whiten them.
%
% \DescribeEnv{longtable}
% \DescribeEnv{Longtable}
% \changes{v1.6}{2003/8/25}
%	{Description of \textsf{longtable} is added.}
% \SpecialOptIndex{longtable}{package}
% Usage of \textsf{longtable} with \textsf{arydshln} is quite simple.  Just
% loading \textsf{arydshln} after \textsf{longtable} is enough to make the
% |longtable| environment able to draw dash-lines.  A shorthand activation
% of dash-line functions is also available by |Longtable| environment.  One
% caution to \textsf{longtable} users is that the temporary results before
% the {\em convergence} of the column widths may be different from those
% without \textsf{arydshln}.  For example, the following is the first pass
% result of the example shown in Table~3 of the \textsf{longtable} manual.
%
% \begin{quote}
% \catcode`\|12
% \leavevmode\vtop{\offinterlineskip
%	\hbox{\begin{tabular}{|c|c|c|}\hline 1&2&3\end{tabular}}
%	\hbox{\begin{tabular}{|c|}wide multicolumn spanning 1--3\end{tabular}}
%	\hbox{\begin{tabular}{|c|c|}\omit&
% 		\omit\vrule height0pt width116.66695pt\cr
% 		multicolumn 1--2&3\end{tabular}}
% 	\hbox{\begin{tabular}{|c|c|c|}\omit&
% 		\omit\vrule height0pt width68.05573pt&
% 		\omit\vrule height0pt width116.66695pt\cr
% 		wide 1&2&3\\\hline\end{tabular}}}
% \end{quote}
%
% \SpecialOptIndex{LTchunksize}{counter}
% Since |LTchunksize| is one in the example, columns of each row has their own
% widths and thus has vertical lines drawn at the edges of the columns.  On
% the other hand, you will have the following as the first pass result with
% \textsf{arydshln}.
%
% \begin{quote}
% \catcode`\|12
% \settowidth\unitlength{\begin{tabular}{|c|c|c|}\hline 1&2&3\end{tabular}}
% \begin{tabular}[t]{|l|c|c|}
% 	\multispan3\relax
% 		\leaders\hrule height\arrayrulewidth\hskip\unitlength\cr
% 	\hbox to0pt{1\hskip2\tabcolsep 2\hskip2\tabcolsep 3\hss}&&\\
% 	\multicolumn3{|l|}{wide multicolumn spanning 1--3}\\
% 	\multicolumn2{|l|}{\hbox to0pt{multicolumn 1--2\hskip\tabcolsep
% 		\hbox to116.66695pt{\hss3\hss}\hss}}&\\
% 	wide 1&2&3\\\hline
% 	\omit&
% 		\omit\vrule height0pt width68.05573pt&
% 		\omit\vrule height0pt width116.66695pt\cr
% 	\end{tabular}
% \end{quote}
%
% As you see, the vertical lines are drawn at the column edges of the last
% row\footnote{
%
% More precisely, drawn according to the column widths established by all
% the chunks preceding page output.}
%
% because \textsf{arydshln} draws them when it see the last row.  Anyway,
% you may ignore temporary results and will have a compatible result when
% the column widths are converged like the following.
%
% \begin{quote}
% \catcode`\|12
% \begin{tabular}[t]{|c|c|c|}\hline
%	1&2&3\\
%	\multicolumn{3}{|c|}{wide multicolumn spanning 1--3}\\
%	\multicolumn{2}{|c|}{multicolumn 1--2}&3\\
%	wide 1&2&3\\\hline
%	\end{tabular}
% \end{quote}
% 
% 
%
% \section{Known Problems}
% \label{sec:prob}
%
% There are following known problems.
%
% \begin{enumerate}
% \item
% The new preamble specifiers `|:|' and `|;{|\meta{dash}|/|\meta{gap}|}|'
% cannot be followed or preceded by `|@{|\meta{text}|}|', or you will have
% an ugly result.  More specifically, a specifier to draw a dash-line at the
% left edge of a column cannot be preceded by `|@{|\meta{text}|}|', while
% that to draw at the right edge cannot be followed by `|@{|\meta{text}|}|'.
%
% \item\leavevmode
% \changes{v1.4-1}{1999/06/25}
%	{Description about `\protect\EXCLAMATION' of array package is added.}
%
% If you use \textsf{array} package, the restriction of `|@|' shown above is 
% also applied to `|!|'.
% 
% \item
% In order to make it sure that a dash-line always {\em touches} its both
% end, i.e. a dash-line always begins and ends with a dash segment, the
% amount of a gap will slightly vary depending on the dash-line length.
%
% \changes{v1.4-1}{1999/06/25}
%	{Reference to the section for drawing mode is added.}
% \changes{v1.72-2}{1999/06/25}
%	{Remove the caution about the dash segment dropping.}
% 
% \item
% \changes{v1.4-1}{1999/06/25}
%	{Description on minimum length is added.}
%
% If a dash-line is too short, you will have an ugly result without overfull
% message.  More specifically, in mode 1 or 3, a line will look to protrude
% beyond its column\slash row borders if it is shorter than a half of
% |\dashlinedash|.  In mode 2, the minimum length to avoid the protrusion is
% $1.5\times|\dashlinedash|+ |\dashlinegap|$.
%
% \item
% \changes{v1.4-1}{1999/06/25}
%	{Reference to the performance tuning section is added.}
%
% As described in \S\ref{sec:usage-performance},
% the processing speed for |array| and |tabular| environment will become
% slower even if dash-lines are not included.
% 
% \item
% \changes{v1.5-1}{2000/07/12}
%	{Caution about \cs{AC}/\cs{EAC} pair for vertical line coloring is
%	 added.}
% \changes{v1.7}{2004/5/21}
%	{Comment on vertical line coloring with \textsf{colortbl} is added.}
%
% 
% As described in \S\ref{sec:usage-compat}, |\AC| and |\EAC| pair of
% \textsf{colortab} such as |\AC&\EAC| cannot color the vertical line at
% |&|.  Use |\ADLinactivate| if you want to have a ordinary table with
% colored vertical lines.  Note that you may color vertical lines with
% \textsf{colortbl} package.
% \end{enumerate}
%
% 
%
% \IndexPrologue{\newpage\section*{Index}
% Italicized number refers to the page where the specification of
% corresponding entry are described.}
% \StopEventually{
% \section*{Acknowledgments}
% 
% \changes{v1.4-1}{1999/06/25}
%	{Thank to more people.}
% \changes{v1.5-3}{2003/07/11}
%	{Thank to Yaxin Liu.}
% \changes{v1.6}{2003/08/25}
%	{Thank to people for \textsf{longtable}.}
% \changes{v1.7}{2004/5/21}
%	{Thank to Klaus Dalinghaus and refer orignal \textsf{colortbl}.}
% \changes{v1.71}{2004/8/31}
%	{Thank to Morten H\o gholm.}
% \changes{v1.72-1}{2016/03/19}
% 	{Thank to Ma\"ieul Rouquette.}
% \changes{v1.73}{2016/03/19}
% 	{Thank to Ma\"ieul Rouquette again.}
%
% The author thanks to Monty Hayes who gave the author the opportunity to
% make this style, and Weimin Zhang and Takahiro Kubota who pointed out bugs
% in early versions.  He also thanks to the following people; Sebastian
% Rahtz and Graham Williams who kindly invited the style to \TeX{} CTAN and
% online catalogue compiled by Graham;  Peter Ehrbar who showed the style was
% incompatible with \textsf{array} and kindly accepted the offer to be an
% alpha-user of v1.4 alone;  Zsuzsanna Nagy who reported another
% incompatibility problem with \textsf{colortab};  Ralf Heydenreich who
% reported the bug causing that glues in a column have no effect; Yaxin 
% Liu who reported the incompatibility bug of \textsf{array} and
% \cs{ADLinactivate}; Craig Leech who reported the incompatibility 
% problem with \textsf{longtable}, which was also reported by Uwe Jehmlich,
% Torge Thielemann and Florian Weig, and had waited for two years and a
% half (!)\ for the solution; Klaus Dalinghaus who reported yet another
% incompatibility with \textsf{colortbl}; Morten H\o gholm who reported the
% bug of |m|-type columns of \textsf{array} which had not manifested in five
% (!!)\ years since the author realesed the first \textsf{array}-compatible
% version; and Ma\"ieul Rouquette who reported another bug of |m|-type
% columns of \textsf{longtable} with \textsf{array} which had peacefully
% hidden in the package for eleven years and a half (!!!)\ since the author
% made the bug fix shown above carelessly, yet another bug related to
% \textsf{longtable}, and most surprisingly a problem on intersections of
% horizontal and vertical (dash-)lines which has hidden for 23 years (!!!!)
% since the very first version of the package.
%
% The base implementation of \texttt{array} and \texttt{tabular}
% environments, part of which the author gives new definitions referring
% original ones, are written by Leslie Lamport as a part of \LaTeX-2.09 and
% \LaTeXe{} (1997/12/01) to which Johannes Braams and other authors also
% contributed.  The author also refers \textsf{array} package (v2.3m)
% written by Frank Mittelbach and David Carlisle; \textsf{colortab} package
% (v0.9) written by Timothy van Zandt; and \textsf{longtable} (v4.10) and
% \textsf{colortbl} (v0.1j) packages
% written by David Carlisle; to make the style compatible with those
% packages.
% 
% \PrintIndex}
%
% 
%
% \newpage
% \section{Implementation}
% \label{sec:imp}
% \subsection{Problems and Solutions}
% \label{sec:imp-ps1}
%
% \changes{v1.4-1}{1999/06/25}
%	{The title of section 4.1 is changed.}
% \changes{v1.4-1}{1999/06/25}
%	{\cs{hfil} is replaced with \cs{hss} taking the possibility of
%	negative wide columns into account.}
% \changes{v1.7}{2004/5/21}
%	{Add notes for dash line coloring.}
%
% We have two different problems to solve; how to draw horizontal dash-lines
% and how to draw vertical dash-lines.  The former problem is relatively
% easy because the technique for drawing |\cline|-s can be used.  That is, if
% we know the number of columns, we can draw a dash-line across the
% |\multispan|-ed columns by |\xleaders| of dash.  Modifying a preamble of
% |array|\slash|tabular| to count the number of columns is not hard.  Since
% |\cdashline| is given beginning and ending columns, its implementation is
% also easy.
%
% \SpecialIndex{\@arstrutbox}
%
% The latter problem, however, is much harder.  Remember that
% |array|\slash|tabular| draws vertical solid lines by |\vrule|-s in each row
% without height\slash depth specification exploiting \TeX's sophisticated
% mechanism of the rule extension in the surrounding box.  Since \TeX{} does
% not have such a mechanism for |\xleaders| unfortunately, we at least have
% to know the height and depth of a row which includes vertical dash-lines.
% Although the height and depth are often same as those of |\@arstrutbox|, we
% will have an exceptionally tall and/or deep row that makes dash-lines {\em
% broken} if we assume every row has the standard height and depth.
%
% Moreover, even if we can measure the height\slash depth of each row (in
% fact we will do as describe later), drawing dash-lines in each row will
% not produce a good result.  Look at the following two examples closely.
%
%\begin{quote}
%\makeatletter
%\begin{tabular}{c:c}\hline
%A&B\\\hline
%\noalign{\xdef\adl@rowsL{\adl@rowsL(\relax/0);}
%         \xdef\adl@rowsR{\adl@rowsR(\relax/0);}}
%{\vrule height2\ht\strutbox depth2\dp\strutbox width\z@\LARGE A}&B\\\hline
%\end{tabular}\qquad
%\begin{tabular}{c:c}\hline
%A&B\\\hline
%{\vrule height2\ht\strutbox depth2\dp\strutbox width\z@\LARGE A}&B\\\hline
%\end{tabular}
%\end{quote}
%
% In the left example, two dash-lines are individually drawn in two rows.
% Since the first row is not so tall and deep (8.4\,pt\slash
% 3.6\,pt) as to contain enough number of default dash segments (4\,pt dash and
% 4\,pt gap) to keep |\xleaders| from inserting a large space, the dash-line
% in the first row is {\em sparse}.  On the other hand, the second row is
% enough tall and deep (16.8\,pt\slash 7.2\,pt) and thus the dash-line in the
% row looks better.  Thus the resulting dash-line is awful because it does
% not have a continuous dash/gap sequence.
%
% The right example, which we wish to produce, is much better than the left.
% In this example, the dash line is draw across two rows keeping continuous
% steps of dashes and gaps.  In order to have this result, we have to draw
% the dash-line {\em after} two rows are built because it is necessary to
% know the total hight and depth of two rows.  In general, if we know the
% total hight and depth of rows and whether a column has a dash-line, we can
% draw dash-lines by adding an extra row containing dash-lines.  For
% example, the result shown above is obtained by the following row.
%
% \begin{quote}
% |\omit\hss|\meta{dash-line of 36\,pt high}|&\omit\cr|
% \end{quote}
%
% Note that \meta{dash-line of 36\,pt high} have to be |\smash|-ed.
%
% In addition to this basic scheme, we have to take the following points
% into account.
%
% \begin{itemize}
% \item
% A dash-line drawn by the preamble character `|;|' will have non-default
% dash\slash gap specification.
%
% \item
% A column may have two or more dash-lines separated by spaces of
% |\doublerulesep|.  Mixed sequence of solid- and dash-lines also have to be
% allowed.
%
% \item
% The first column may have dash-lines at both ends, while those of others
% will appear at right ends only.  An exception of this rule is brought by
% |\multicolumn| that may have leading sequence of solid- and/or dash-line
% specifiers in its preamble.
%
% \item
% A |\multicolumn| may break or add a dash-line, or may change the
% dash\slash gap specification of a dash-line.  A sequence of
% |\h(dash)line|-s also break dash-lines.
% 
% \item
% If \textsf{colortbl} is in use, coloring dash\slash gap by
% |\arrayrulecolor| and |\dashgapcolor| gives another possibility of the
% variation of dash\slash gap specification.
% \end{itemize}
%
% In order to cope with them, the following data structure is constructed
% during rows are built.
%
% \begin{enumerate}
% \item
% The list of row information $R=\langle r_1,r_2,\ldots,r_N\rangle$.
%
% \item
% The $i\th$ element of $R$, $r_i$, is one of the following\footnote{
%
% In the real implementation, the structure of $r_i$ is slightly different.}.
%
% \begin{enumerate}
% \item
% A triple $\langle C^L_i,C^R_i,h_i\rangle$, where $C^L_i$ and $C^R_i$ are
% the lists of solid- or dash-line segments drawn at the left and right edge
% of columns respectively, and $h_i$ is the height plus depth of the $i\th$
% row.
%
% \item
% ${\it connect}(h_i)$ for a |\h(dash)line| of $h_i$ wide meaning that $r_i$
% is an empty pseudo row of $h_i$ high and dash-lines are not broken at the
% row.
%
% \item
% \changes{v1.6}{2003/8/25}
%	{Description of \textit{discard} is added.}
% In |longtable| environment, ${\it discard}(h_i)$ for a negative vertical
% space inserted by |\\[|\meta{$h_i$}|]| or |\h(dash)line| meaning $r_i$
% is an empty pseudo row of $h_i$ high and dash-lines are not broken but may 
% be discarded by the page break at the row.
%
% \item
% ${\it disconnect}(h_i)$ for a vertical gap generated by a sequence of
% |\h(dash)line| meaning that $r_i$ is an empty pseudo row of $h_i$ high and
% dash-lines are broken at the row.
% \end{enumerate}
%
% \item
% $C^L_i=\langle e^i_1,e^i_2,\ldots,e^i_m\rangle$ where $e^i_j$ corresponds
% to the $j\th$ (leftmost is first) solid- or dash-line segment.  $C^R_i$ is
% similar but its elements are ordered in reverse, i.e. the rightmost segment
% is the first element.
%
% \item
% \changes{v1.7}{2004/5/21}
% 	{A dash/gap specification $d^i_j/g^i_j$ now has color.}
% The $j\th$ element of $C^L_i$ or $C^R_i$, $e^i_j$, is a triple $\langle
% c^i_j,d^i_j,g^i_j\rangle$ where $c^i_j$ is the column number in which the
% segment appears, and $d^i_j$ and $g^i_j$ are dash\slash gap specification,
% length and color, of the segment.  For a solid line segment, the length
% attributes of both $d^i_j$ and $g^i_j$ are 0.
% \end{enumerate}
%
% Then this data structure is processed to draw solid- and dash-lines at the
% end of the |array|\slash|tabular| as follows.  Let $e^i_j=\langle c^i_j,
% d^i_j, g^i_j\rangle$ be the $j\th$ element of $C^L_i$ of $r_i$.  The {\em
% position} $p^i_j$ of $e^i_j$ in the column $c^i_j$ is defined as follows.
%$$
% p^i_j = \cases{1&		if $j=1 \LOR c^i_j\neq c^i_{j-1}$\cr
%		 p^i_{j-1}+1&	otherwise}.
%$$
% The following defines whether two elements $e^i_j$ and $e^{i'}_{j'}$ are
% {\em connected}, or $e^i_j\sim e^{i'}_{j'}$.
%
% \begin{eqnarray*}
% e^i_j\sim e^{i'}_{j'}&
%	\togets&i<i'\LAND\\
% &&	c^i_j=c^{i'}_{j'}\LAND d^i_j=d^{i'}_{j'}\LAND g^i_j=g^{i'}_{j'}\LAND
%		p^i_j=p^{i'}_{j'}\LAND\\
% &&	^\forall k(i<k<i'\to r_k=\mathit{connect}(h_k)).
% \end{eqnarray*}
%
% With these definitions, we can classify all $e^i_j$ into ordered sets
% $S_1,S_2,\ldots S_n$ as follows.
%
% \def\PH{\phantom{^\forall e^i_j\in S_k,\forall e^{i'}_{j'}\in S_{k'}:{}}}
% \begin{eqnarray*}
% k\neq k'&\togets&
%	S_k\cap S_{k'}=\emptyset\\
% e^i_j\sim e^{i'}_{j'}&\togets&
%	^\exists k:e^i_j,e^{i'}_{j'}\in S_k\LAND
%		   S_k=\{\ldots,e^i_j,e^{i'}_{j'},\ldots\}\\
% k<k'&\togets&
%	^\forall e^i_j\in S_k,\forall e^{i'}_{j'}\in S_{k'}:
%		(c^i_j<c^{i'}_{j'})\LOR\\
% &&	\PH	(c^i_j=c^{i'}_{j'}\LAND p^i_j<p^{i'}_{j'})\LOR\\
% &&	\PH	(c^i_j=c^{i'}_{j'}\LAND p^i_j=p^{i'}_{j'}\LAND i<i').
% \end{eqnarray*}
%
% Now we can draw a dash-line $L_k=\langle\gamma_k, \pi_k, \delta_k, \xi_k,
% \tau_k, \beta_k\rangle$ corresponding to $S_k=\{e^i_j,\ldots,e^{i'}_{j'}\}$
% as follows.
%
% \begin{itemize}
% \item
% $L_k$ is $\pi_k\th$ line in the $\gamma_k\th$ column where $\gamma_k=
% c^i_j=\ldots=c^{i'}_{j'}$ and $\pi_k=p^i_j=\ldots=p^{i'}_{j'}$.
%
% \item
% $L_k$ has the dash specification (size and color)
% $\delta_k=d^i_j=\ldots=d^{i'}_{j'}$ and gap specification
% $\xi_k=g^i_j=\ldots=g^{i'}_{j'}$.
%
% \item
% \changes{v1.73-1}{2016/04/28}
% 	{Add the definition of $\eta_l$ and addition\slash subtraction of it
%	 for $\tau_k$ and $\beta_k$.}
%
% The top and bottom ends of $L_k$ are at $\tau_k$ and $\beta_k$ above the
% bottom of the |array|\slash|tabular|, where;
%$$
% \eta_l=\cases{h_l&$r_l=\mathit{connect(h_l)}$\cr
% 		0&otherwise},\quad
% \tau_k=\eta_{i-1}+\sum_{l=i}^N h_l,\quad
% \beta_k=-\eta_{i'+1}+\sum_{l=i'+1}^N h_l.
%$$
% Note that $\eta_{i-1}$ and $\eta_{i'+1}$ are added\slash subtracted so that
% the top\slash bottom of $L_k$ is at the top\slash bottom edge of the
% horizontal lines above\slash below the set $S_k$.
% \end{itemize}
%
%  The row to draw $L_1,\ldots,L_n$ is;
%$$
% \sigma_1 L_1 \sigma_2 L_2 \ldots L_{n-1}\sigma_n L_n\sigma_{n+1}|\cr|
%$$
% where;
%
% \def\amp{\hbox{\tt\&}}
% \begin{eqnarray*}
% \sigma_1&=&
%	\cs{omit}[\cs{hss}\amp\cs{omit}]^{\gamma_1-1}\\[1ex]
% \sigma_{1<k\leq n}&=&
%	\cases{	\cs{null}&
%			if $\gamma_{k-1}=\gamma_k\LAND \pi_{k-1}=\pi_k$\cr
%		\cs{hskip}\cs{doublerulesep}&
%			if $\gamma_{k-1}=\gamma_k\LAND \pi_{k-1}\neq\pi_k$\cr
%		[\cs{hss}\amp\cs{omit}]^{\gamma_k-\gamma_{k-1}}&
%			if $\gamma_{k-1}\neq\gamma_k$\cr}\\[1ex]
% \sigma_{n+1}&=&
%	[\cs{hss}\amp\cs{omit}]^{{\mit\Gamma}-\gamma_n-1}\cs{hss}.
% \end{eqnarray*}
%
% Note that $[x]^m$ means $m$-times iteration of $x$, and $\mit\Gamma$ is
% the number of columns specified in the preamble.
%
% Dash-lines at the right edges of columns are similarly drawn by processing
% $C^R_i$ with the following modifications.
%
% \begin{eqnarray*}
% k<k'&\togets&
%	^\forall e^i_j\in S_k,\forall e^{i'}_{j'}\in S_{k'}:
%		(c^i_j<c^{i'}_{j'})\LOR\\
% &&	\PH	(c^i_j=c^{i'}_{j'}\LAND p^i_j>p^{i'}_{j'})\LOR\\
% &&	\PH	(c^i_j=c^{i'}_{j'}\LAND p^i_j=p^{i'}_{j'}\LAND i<i')\\
% \sigma_1&=&
%	\cs{omit}\cs{hss}[\amp\cs{omit}\cs{hss}]^{\gamma_1-1}\\[1ex]
% \sigma_{k>1}&=&
%	\cases{	\cs{null}&
%			if $\gamma_{k-1}=\gamma_k\LAND \pi_{k-1}=\pi_k$\cr
%		\cs{hskip}\cs{doublerulesep}&
%			if $\gamma_{k-1}=\gamma_k\LAND \pi_{k-1}\neq\pi_k$\cr
%		[\amp\cs{omit}\cs{hss}]^{\gamma_k-\gamma_{k-1}}&
%			if $\gamma_{k-1}\neq\gamma_k$\cr}\\[1ex]
% \sigma_{n+1}&=&
%	[\amp\cs{omit}\cs{hss}]^{{\mit\Gamma}-\gamma_n-1}
% \end{eqnarray*}
%
% 
%
% \subsection{Another Old Problem}
% \label{sec:imp-ps2}
% 
% \changes{v1.72-2}{1999/06/25}
%	{Change the title of \S4.2 and rephrase sentences according to the
%	 fix of \cs{xleader}'s problem.}
% 
% In the default mode 1, we draw a dash line of dash size $d$ and gap size
% $g$ as follows.  Let $W$ be the length of the line plus 10\,|sp|\footnote{
%
% This small amount is added by \cs{xleaders} in order to, according to the 
% comment in \texttt{tex.web}, compensate floating point rounding error.},
%
% which is unknown for us if horizontal but known for \TeX, and assume
% $W\geq d/2$ (or the line protrude to the column\slash row boarder.)  At
% the both ends of the columns, dashes of $d/2$ long are drawn to make the
% dash-line {\em touched} to the ends.  Then $n=\lfloor(W-d-g)/(d+g)\rfloor$
% dashes are equally distributed in the remaining space.  Thus we will have;
% $$
% D_0(d/2)G_0(g+\epsilon')D_1(d)G_1(g+\epsilon)\ldots
%	G_{n-1}(g+\epsilon)D_n(d)G_n(g+\epsilon')D_{n+1}(d/2)
% $$
% where $D_i(l)$ and $G_i(l)$ are dash and gap of $l$ long,
% $\epsilon=(W-(n+1)(d+g))/(n+1)$ (rounded), and
% $\epsilon'=(W-(n+1)(d+g)-(n-1)\epsilon)/2$ to compensate the rounding
% error on the calculation of $\epsilon$.  For a horizontal line, this
% result will be obtained by |\xleaders| as follows where $G^m_i(\epsilon)$
% and $G^m_i(\epsilon')$ are the spaces inserted by |\xleaders|.
%
% \begin{eqnarray*}
% \rlap{$\displaystyle D_0(d/2)G^l_0(g/2)
%		\hbox{\cs{xleaders}\cs{hbox}{\tt\char`\{}}
%			G^r(g/2)D(d)G^l(g/2)\hbox{\tt\char`\}\cs{hss}\ }
%		G^r_n(g)D_{n+1}(d/2)$}\qquad\\
% &=&D_0(d/2)G^l_0(g/2)G^m_0(\epsilon')
%	\left(G^r_0(g/2)D_1(d)G^l_1(g/2)\right)G^m_1(\epsilon)\\
% &&	\phantom{D_0(d/2)G^l_0(g/2)G^m_0(\epsilon)}
%	\left(G^r_1(g/2)D_2(d)G^l_2(g/2)\right)G^m_2(\epsilon)\\
% &&	\phantom{D_0(d/2)G^l_0(g/2)G^m_0(\epsilon)}\ldots\\
% &&	\phantom{D_0(d/2)G^l_0(g/2)G^m_0(\epsilon)}
% 	\llap{$\displaystyle G^m_{n-1}(\epsilon)$}
% 	\left(G^r_{n-1}(g/2)D_n(d)G^l_n(g/2)\right)
%	G^m_n(\epsilon')G^r_n(g/2)D_{n+1}(d/2)\\
% &=&D_0(d/2)G_0(g+\epsilon')D_1(d)G_1(g+\epsilon)\ldots
%	G_{n-1}(g+\epsilon)D_n(d)G_n(g+\epsilon')D_{n+1}(d/2)
% \end{eqnarray*}
%
% The problem is that $\epsilon'$ could be negative and old \TeX{} mistakingly
% ignored this possibility.  That is, since the \TeX{} older than 3.141592
% did not put |\hbox| beyond the right edge of |\xleaders|, the rightmost
% |\hbox| was omitted if $\epsilon'$ is negative.
%
% Since it is (almost) impossible to know the length of a horizontal line,
% we could not cope with this problem by adding or subtracting its length.
% Thus we introduced {\em drawing mode} to have imperfect solutions.  In the 
% mode 2, we draw a line by the following sequence.
%
% \begin{eqnarray*}
% \rlap{$\displaystyle D_0(d/2)G^l_0(g/2)G^r_{0'}(g/2)D_{1'}(d)G^l_{1'}(g/2)
%		G(-d-g)$}\qquad\\
% &&	\hbox{\cs{xleaders}\cs{hbox}{\tt\char`\{}}
%			G^r(g/2)D(d)G^l(g/2)\hbox{\tt\char`\}\cs{hss}\ }\\
% &&	G(-d-g)G^r_{n'}(g/2)D_{n'}(d)G^l_{n'}(g/2)G^r_n(g)D_{n+1}(d/2)
% \end{eqnarray*}
%
% That is, $n\th$ |\hbox| that could be disappeared is put twice and the
% first one is also overlaid for symmetrization.  Therefore the length of the
% first and $n\th$ dashes is $d+\vert\epsilon'\vert$ and thus could be a
% little bit longer than others.
% 
% On the other hand, we replace |\xleaders| of mode 1 with |\cleaders| for
% the drawing in mode 3.  The result will be;
% $$
% D_0(d/2)G_0(g+R)D_1(d)G_1(g)\ldots
%	G_{n-1}(g)D_n(d)G_n(g+R)D_{n+1}(d/2)
% $$
% where $R=(W-(n+1)(d+g))/2$ to make the first and last gaps considerably wider
% than others.
%
% 
%
% \iffalse
%<*package>
% \fi
%
% \subsection{Register Declaration}
% \label{sec:imp-decl}
%
% Here registers and switches are declared.
%
% \begin{macro}{\dashlinedash}
% \begin{macro}{\dashlinegap}
% \begin{macro}{\hdashlinewidth}
% \begin{macro}{\hdashlinegap}
% First of all, two |\dimen| registers |\dashlinedash| and |\dashlinegap|
% to control the shape of dash-lines are declared, and their default values,
% 4\,pt for both, are assigned to them.  They have aliases,
% |\hdashlinewidth| and |\hdashlinegap| respectively, for the backward
% compatibility.
% \end{macro}\end{macro}\end{macro}\end{macro}
%
%    \begin{macrocode}
%% Register Declaration

\newdimen\dashlinedash \dashlinedash4pt %
\newdimen\dashlinegap \dashlinegap4pt %
\let\hdashlinewidth\dashlinedash
\let\hdashlinegap\dashlinegap

%    \end{macrocode}
%
% Next, the following six switches are declared.
% \begin{itemize}
% \item
% \begin{macro}{\ifadl@leftrule}
% |\ifadl@leftrule| is used in the preamble analysis macro |\@mkpream| and
% is true during it processes leading characters for solid- and dash-lines,
% i.e.\ `\verb!|!', `|:|', and `|;|'.
% \end{macro}
%
% \item
% \begin{macro}{\ifadl@connected}
% |\ifadl@connected| is used to indicate the {\em connection} $e^i_j\sim
% e^{i'}_{j'}$.  When we process $e^{i'}_{j'}$, the switch is true iff
% $^\exists e^i_j(e^i_j\sim e^{i'}_{j'})$.
% \end{macro}
%
% \item
% \begin{macro}{\ifadl@doublerule}
% |\ifadl@doublerule| is used to make $\sigma_k$.  When we are to make
% $\sigma_k L_k$, it is true iff $\gamma_{k-1}=\gamma_k\LAND\pi_{k-1}\neq
% \pi_k$.
% \end{macro}
%
% \item
% \begin{macro}{\ifadl@zwvrule}
% \changes{v1.4-2-2}{1999/06/25}
%	{Introduced to indicate vertical lines have null width.}
% |\ifadl@zwvrule| controls the {\em real} width of vertical lines.  If it
% is true, lines are drawn as if their width is zero following \LaTeX's
% standard.  Otherwise, their width |\arrayrulewidth| contribute to the
% width of columns as \textsf{array} does.
% \end{macro}
%
% \item
% \begin{macro}{\ifadl@zwhrule}
% \changes{v1.7-1}{2004/5/21}
%	{Introduced to indicate horizontal lines have null width.}
% |\ifadl@zwhrule| controls the {\em real} width of horizontal lines.  If it
% is true, a line is drawn as if its width is zero and its bottom edge is
% adjusted to that of the row above by inserting |\vskip-\arrayulewidth|
% before the drawing.  Thus a horizontal dash line is included in the row
% above and its gaps look colored properly if the row is painted.  If it is
% false, the width |\arrayrulewidth| contribute to the height of
% \texttt{array}\slash\texttt{tabular} as usual.
% \end{macro}
%
% \item
% \begin{macro}{\ifadl@usingarypkg}
% \changes{v1.4-2-1}{1999/06/25}
%	{Introduced to know if \textsf{array} is loaded.}
% |\ifadl@usingarypkg| is true iff \textsf{array} has been loaded prior to
% \textsf{arydshln}.  This switch shows us which definitions, by \LaTeX{} or 
% \textsf{array}, we have to modify.  Its value is set by examining if
% |\extrarowheight|, which is introduced by \textsf{array}, is defined.
% \end{macro}
%
% \item
% \begin{macro}{\ifadl@inactive}
% \changes{v1.4-2-3}{1999/06/25}
%	{Introduced to indicate dash-line functions are inactive.}
% |\ifadl@inactive| inactivates dash-line functions if it is true.  Its
% default value is false.
% \end{macro}
% \end{itemize}
%
% \SpecialIndex{\@tempswa}
% We also use a working switch |\@tempswa|.
%
%    \begin{macrocode}
\newif\ifadl@leftrule
\newif\ifadl@connected
\newif\ifadl@doublerule
\newif\ifadl@zwvrule
\newif\ifadl@zwhrule
\newif\ifadl@usingarypkg
\ifx\extrarowheight\undefined \adl@usingarypkgfalse
\else                         \adl@usingarypkgtrue  \fi
\newif\ifadl@inactive \adl@inactivefalse

%    \end{macrocode}
%
% \begin{macro}{\ADLnullwide}
% \changes{v1.7-1}{2004/5/21}
%	{Introduced to make horizontal lines null wide.}
% \begin{macro}{\ADLsomewide}
% \changes{v1.7-1}{2004/5/21}
%	{Introduced to make horizontal lines \cs{arraydashline} wide.}
% The switch |\ifadl@hwvrule| is turned on/off by user interface macros
% |\ADLnullwide| and |\ADLsomewide|.  Its initial value is the complement of
% |\adl@usingarypkg|.
% \end{macro}\end{macro}
%
% \begin{macro}{\ADLnullwidehline}
% \changes{v1.4-2-2}{1999/06/25}
%	{Introduced to make vertical lines null wide.}
% \begin{macro}{\ADLsomewidehline}
% \changes{v1.4-2-2}{1999/06/25}
%	{Introduced to make vertical lines \cs{arraydashline} wide.}
% The switch |\ifadl@zwvrule| is turned on/off by user interface macros
% |\ADLnullwidehline| and |\ADLsomewidehline|.  Its initial value is false.
% \end{macro}\end{macro}
%
% \begin{macro}{\ADLactivate}
% \begin{macro}{\ADLinactivate}
% The switch |\ifadl@inactive| is also turned on/off by user interface macros
% |\ADL#inactivate| and |\ADLactivate|.
% \end{macro}\end{macro}
%
%    \begin{macrocode}
\def\ADLnullwide{\adl@zwvruletrue}
\def\ADLsomewide{\adl@zwvrulefalse}
\ifadl@usingarypkg \ADLsomewide \else \ADLnullwide \fi
\def\ADLnullwidehline{\adl@zwhruletrue}
\def\ADLsomewidehline{\adl@zwhrulefalse}
\ADLsomewidehline

\def\ADLactivate{\adl@inactivefalse}
\def\ADLinactivate{\adl@inactivetrue}

%    \end{macrocode}
%
% The following |\box| register and three |\dimen| registers are used to
% measure the height and depth of a row.
%
% \begin{itemize}
% \item
% \begin{macro}{\adl@box}
% The contents of a column is packed into the |\box| register |\adl@box| to
% measure its height and depth.
% \end{macro}
%
% \item
% \begin{macro}{\adl@height}
% \begin{macro}{\adl@depth}
% The |\dimen| registers |\adl@height| and |\adl@depth| contain the
% height\slash depth of the tallest\slash deepest column in a row.  When a
% column is processed, they are compared to the height and depth of
% |\adl@box| and are updated if they are less.
% \end{macro}\end{macro}
%
% \begin{macro}{\adl@heightsave}
% \begin{macro}{\adl@depthsave}
% Since we have to update these registers |\global|-ly to pass their values
% across |&| and we may have a column containing |array|\slash|tabular|,
% they are saved into |\adl@heightsave|\slash|\adl@depthsave| at the
% beginning of the environment and are restored at its end.
% \end{macro}\end{macro}
% 
% \item
% \begin{macro}{\adl@finaldepth}
% \changes{v1.4-2-6}{1999/06/25}
%	{Introduced to measure the depth of the last row.}
% The other |\dimen| register |\adl@finaldepth| is set to the depth of the
% last row, or zero if the last vertical item is a horizontal line.  This
% value is used to shift |array|\slash|tabular| down because we add extra 
% two |\smash|-ed rows which make the depth of |array|\slash|tabular| zero.
% \end{macro}
% \end{itemize}
%
% \SpecialIndex{\@tempdima}
% \SpecialIndex{\@tempdimb}
% We also use working |\dimen| registers |\@tempdima| and |\@tempdimb|.
%
%    \begin{macrocode}
\newbox\adl@box
\newdimen\adl@height \newdimen\adl@heightsave
\newdimen\adl@depth \newdimen\adl@depthsave
\newdimen\adl@finaldepth
%    \end{macrocode}
%
% Then the following |\count| registers are declared.  Note that some of
% them contain dimensions measured by the unit |sp|.
%
% \begin{itemize}
% \item
% \begin{macro}{\adl@columns}
% \begin{macro}{\adl@ncol}
% \changes{v1.4-2-1}{1999/06/25}
%	{Introduced for new column counting in preamble construction.}
% |\adl@columns| has the number of columns specified in the preamble of the
% environment.  Because of a complicated reason related to the compatibility 
% with \textsf{array}, we cannot count up |\adl@columns| directly but
% increment |\adl@ncol| when each column of preamble is built and move its
% value to |\adl@columns| after the preamble is constructed.
% \end{macro}\end{macro}
%
% \item
% \begin{macro}{\adl@currentcolumn}
% \begin{macro}{\adl@currentcolumnsave}
% To process |\multicolumn|, we have to know the column number where it
% appears.  Thus we have a column counter |\adl@currentcolumn| which is
% |\global|-ly incremented when each column is built.  Because of the
% |\global| assignment, the counter has to be saved\slash restored
% into\slash from |\adl@currentcolumnsave|.
% \end{macro}\end{macro}
%
% \item
% \begin{macro}{\adl@totalheight}
% In the real implementation, $\tau_k$ and $\beta_k$ are calculated by the
% following equations rather than those shown in \S\ref{sec:imp-ps1}.
%$$
% H=\sum_{l=1}^N h_l,\quad
% \tau_k=H-\sum_{l=1}^{i-1} h_l,\quad
% \beta_k=\tau_k-\sum_{l=i'}^i h_l.
%$$
% |\adl@totalheight| contains $\sum_{l=1}^i h_l$ when the $i\th$ row is
% built and thus its final value is $H$.  Since the data structure $R$ are
% represented by a text, we have to pay attention to the precision of its
% dimensional elements, such as $h_i$.  That is, if we append $h_i$ to $R$
% by expanding |\the\dimen|$n$ which has the height plus depth of $i\th$
% row, $h_i$ will be an approximation of |\dimen|$n$ represented by a
% decimal fraction with |pt|.  Although the error of the approximation is
% quite small and may be negligible, the error must be avoided because it is
% avoidable by simply using |\number\dimen|$n$.  Therefore, $h_i$ is an
% integer and thus |\adl@totalheight| is too.
% \end{macro}
%
% \begin{macro}{\adl@totalheightsave}
% Because of the |\global| assignment to |\adl@totalheight| to pass its value
% across rows, it has to be saved\slash restored into\slash from
% |\adl@totalheightsave|.
% \end{macro}
%
% \item
% \begin{macro}{\adl@dash}
% \begin{macro}{\adl@gap}
% In order to check $e^i_j\sim e^{i'}_{j'}$, the size attributes of $d^i_j$
% and $g^i_j$ are kept in
% the registers |\adl@dash| and |\adl@gap| when we process $e^{i'}_{j'}$.
% As explained above, $d^i_j$ and $g^i_j$ are integers and thus |\adl@dash|
% and |\adl@gap| are |\count| registers.
% \end{macro}\end{macro}
%
% \item
% \begin{macro}{\adl@cla}
% \begin{macro}{\adl@clb}
% \SpecialIndex{\@cla}
% \SpecialIndex{\@clb}
% The coding of |\cdashline| is similar to that of |\cline| in \LaTeX-2.09
% which uses two global |\count| registers |\@cla| and |\@clb|.  These
% registers are omitted from \LaTeXe{} because its |\cline| is completely
% recoded.  We could adopt new coding but it requires some other macro
% definitions that \LaTeX-2.09 does not have.  Thus we simply introduce new
% global counters |\adl@cla| and |\adl@clb| for |\cdashline| in order to
% make |\cdashline| work in both \LaTeX-2.09 and \LaTeXe.
% \end{macro}\end{macro}
% \end{itemize}
%
% \SpecialIndex{\@tempcnta}
% \SpecialIndex{\@tempcnta}
% We also use working |\count| registers |\@tempcnta| and |\@tempcntb|.
%
%    \begin{macrocode}
\newcount\adl@columns \newcount\adl@ncol
\newcount\adl@currentcolumn \newcount\adl@currentcolumnsave
\newcount\adl@totalheight \newcount\adl@totalheightsave
\newcount\adl@dash \newcount\adl@gap
\newcount\adl@cla \newcount\adl@clb
%    \end{macrocode}
%
% \begin{macro}{\adl@everyvbox}
% \changes{v1.4-2-1}{1999/06/25}
%	{Introduced for a tricky modification of \cs{@array}.}
% The last register declaration is for a |\toks| register named
% |\adl@everyvbox|.  In order to minimize the copy-and-modify of the codes
% in \LaTeX{} and \textsf{array}, we need to use |\everyvbox| in our own
% definition of |\@array|.  The register is used to save the contents of
% |\everyvbox|.
% \end{macro}
%
%    \begin{macrocode}
\newtoks\adl@everyvbox

%    \end{macrocode}
%
% \begin{macro}{\adl@org@arrayclassz}
% \changes{v1.4-2-3}{1999/06/25}
%	{Introduced to restore \cs{@arrayclassz}.}
% \begin{macro}{\adl@org@tabclassz}
% \changes{v1.4-2-3}{1999/06/25}
%	{Introduced to restore \cs{@tabclassz}.}
% \begin{macro}{\adl@org@classz}
% \changes{v1.4-2-3}{1999/06/25}
%	{Introduced to restore \cs{@classz}.}
% \begin{macro}{\adl@org@@startpbox}
% \changes{v1.4-2-3}{1999/06/25}
%	{Introduced to restore \cs{@@startpbox}.}
% \begin{macro}{\adl@org@@endpbox}
% \changes{v1.4-2-3}{1999/06/25}
%	{Introduced to restore \cs{@@endpbox}.}
% \begin{macro}{\adl@org@endpbox}
% \changes{v1.4-2-3}{1999/06/25}
%	{Introduced to restore \cs{@endpbox}.}
% \begin{macro}{\adl@org@cline}
% \changes{v1.4-2-3}{1999/06/25}
%	{Introduced to restore \cs{cline}.}
% \changes{v1.4-2-6}{1999/06/25}
%	{Introduced to refer original version in modified \cs{cline}.}
% The other declarative stuff consists of the sequence of |\let| to capture
% the original definitions of macros that we will modify afterword.  The
% main purpose of them is to nullify the modification when dash-line
% functions are inactive, while |\adl@org@cline| is also referred to in its
% modified version.
% \end{macro}\end{macro}\end{macro}\end{macro}\end{macro}\end{macro}\end{macro}
%
%    \begin{macrocode}
\let\adl@org@arrayclassz\@arrayclassz
\let\adl@org@tabclassz\@tabclassz
\let\adl@org@classz\@classz
\let\adl@org@@startpbox\@@startpbox
\let\adl@org@@endpbox\@@endpbox
\let\adl@org@endpbox\@endpbox
\let\adl@org@cline\cline

%%^L
%    \end{macrocode}
%
% 
%
% \subsection{Initialization}
% \label{sec:imp-init}
%
% \begin{macro}{\adl@array}
% \changes{v1.4-2-1}{1999/06/25}
%	{Introduced to save original definition of \cs{@array}.}
% \begin{macro}{\@array}
% \changes{v1.4-2-1}{1999/06/25}
%	{Drastically modified to avoid copy-and-modify.}
% \begin{macro}{\adl@noalign}
% \changes{v1.7-2}{2004/5/21}
% 	{Introduced to fix a bug of \textsf{colortbl}.}
%
% \LaTeX's macro |\@array| is modified to save and initialize registers and
% data structures which are |\global|-ly updated in order to allow nested
% |array|\slash|tabular|.  This saving and initializing are performed by
% |\adl@arrayinit| as explained below.  The problem in the modification is
% that the code of |\@array| in \textsf{array} is completely different from
% that of \LaTeX{} original.
%
% The main difference is that \LaTeX{} builds |\@preamble| locally, while
% \textsf{array} does globally exploiting the fact that the lifetime of
% |\@preamble| ends before another |array|\slash|tabular| appears in a
% column.  The latter implementation will work well unless the building
% process in |\@mkpream| produces something referred to after |\@preamble|
% is thrown into \TeX's {\em stomach}.  In our implementation,
% unfortunately, the number of columns has to be counted in
% \break|\@mkpream| and will be referred to by |\hdashline| and the vertical
% line drawing procedure.
%
% Thus we have to change the column counting mechanism depending on whether
% or not \textsf{array} is in use.  The simplest way could be to copy the codes
% of \LaTeX{} and \textsf{array} and modify them appropriately examining the
% value of |\ifadl@usingarypkg|.  However this solution is vulnerable to the
% modification of the original version and thus we wish to refuse it as far
% as possible.
%
% \SpecialIndex{\everyvbox}
% \SpecialIndex{\adl@everyvbox}
% Therefore, we use a trick with |\everyvbox| in which |\adl@arrayinit| is
% temporarily included to initialize registers and locally set
% |\adl@columns| to the number of columns |\global|-ly counted by
% |\adl@ncol|.  This trick works well so far because;
%
% \begin{itemize}
% \item
% the first |\vbox|, |\vtop| or |\vcenter| made by |\@array| is the vertical 
% box surrounding |\halign|, and;
%
% \item
% in |\@array| of \textsf{array} the box is opened {\em after} the preamble
% is constructed;
% \end{itemize}
%
% and will hopefully work in future.
%
% \changes{v1.5-3}{2003/07/11}
%	{Move \cs{adl@inactivate} to \cs{@array} from \cs{adl@arrayinit}.}
% \changes{v1.54-2}{2003/08/25}
%	{Invoke \cs{adl@activate} if not \cs{ifadl@inactive}.}
% \changes{v1.7-2}{2004/5/21}
% 	{Make \cs{adl@noalign} \cs{let}-equal to \cs{noalign}.}
% \SpecialIndex{\arrayrulecolor}
% \SpecialIndex{\doublerulesepcolor}
% \SpecialIndex{\dashgapcolor}
% Next, if |\ifadl@inactive| is true, |\adl@inactivate| is invoked to
% inactivate dash-line functions.  Otherwise, |\adl@activate| is invoked to
% activate them because an inactivated |array|\slash|tabular|
% may have active children in it.  Finally, |\adl@noalign| is made
% |\let|-equal to |\noalign| so that |\arrayrulecolor|,
% |\doublerulesepcolor| and |\dashgapcolor| are expanded with |\noalign| in
% the environment.
% \end{macro}\end{macro}\end{macro}
%
% \begin{macro}{\@@array}
% \changes{v1.4-2-1}{1999/06/25}
%	{Introduced because \textsf{array} uses it.}
%
% \SpecialIndex{\@tabarray}
% Another stuff for the compatibility with \textsf{array} is to |\let| a
% control sequence |\@@array| be equal to |\@array| because it is referred
% in |\@tabarray| in \textsf{array}.
% \end{macro}
%
%    \begin{macrocode}

%% Initialization

\let\adl@array\@array
\def\@array{\adl@everyvbox\everyvbox
        \everyvbox{\adl@arrayinit \the\adl@everyvbox \everyvbox\adl@everyvbox}%
        \ifadl@inactive \adl@inactivate \else \adl@activate \fi
        \let\adl@noalign\noalign
        \adl@array}
\let\@@array\@array

%    \end{macrocode}
%
% \begin{macro}{\adl@arrayinit}
% \changes{v1.4-2-9}{1999/06/25}
%	{The bug of saving \cs{adl@colsR} is fixed.}
% \changes{v1.5-1}{2000/07/12}
%	{Use new macro \cs{adl@arraysave} to save registers/structures.}
% \begin{macro}{\adl@arraysave}
% \changes{v1.5-1}{2000/07/12}
%	{Introduced to use in modified \cs{CC@} of \textsf{colortab}.}
% \SpecialIndex{\adl@height}
% \SpecialIndex{\adl@depth}
% \SpecialIndex{\adl@currentcolumn}
% \SpecialIndex{\adl@totalheight}
%
% As described in \S\ref{sec:imp-decl}, registers updated |\global|-ly,
% which are |\adl@height|, |\adl@depth|, |\adl@currentcolumn| and
% |\adl@totalheight|, are saved in |\adl@arrayinit| by calling
% |\adl@arraysave|, and also given initial values.  The macro also saves the
% following data structures and initializes them to empty lists.
% \end{macro}
%
% \begin{macro}{\adl@rowsL}
% \begin{macro}{\adl@rowsR}
% \begin{macro}{\adl@rowsLsave}
% \begin{macro}{\adl@rowsRsave}
% \begin{macro}{\adl@colsL}
% \begin{macro}{\adl@colsR}
% \begin{macro}{\adl@colsLsave}
% \begin{macro}{\adl@colsRsave}
% \begin{itemize}
% \item
% In the real implementation, the data structure $R$ is split into two
% lists;
%
% \begin{eqnarray*}
% \cs{adl@rowsL}&=&R^L=\langle\langle C^L_1,h_1\rangle,\ldots\rangle\\
% \cs{adl@rowsR}&=&R^R=\langle\langle C^R_1,h_1\rangle,\ldots\rangle
% \end{eqnarray*}
%
% and they are saved into |\adl@rowsLsave| and |\adl@rowsRsave|.
%
% \item
% When the $i\th$ row is building, $C^L_i$ and $C^R_i$ are constructed in
% the macros |\adl@colsL| and |\adl@colsR|.  They are saved
% into |\adl@colsLsave| and |\adl@colsRsave|.
% \end{itemize}
% \end{macro}\end{macro}\end{macro}\end{macro}
% \end{macro}\end{macro}\end{macro}\end{macro}
%
% \begin{macro}{\adl@connect}
% \begin{macro}{\adl@discard}
% \changes{v1.6}{2003/8/25}
%	{Add initializetion of \cs{adl@discard}.}
% \SpecialIndex{\@elt}
% \SpecialOptIndex{longtable}{environment}
%
% In the real implementation, $e^i_j$ is represented by a control sequence
% |\@elt|, and ${\it connect}(i)$ by |\adl@connect|.  They are made
% |\let|-equal to |\relax| to keep them from expansion during $R$ is
% constructed.  In |longtable| environment, ${\it connect}(i)$ for negative
% vertical space inserted by |\\[|\meta{h}|]| or a horizontal line has
% another representation |\adl@discard| to indicate it corresponds to a
% discardable item of page breaking.  Since this representation, however, is
% nonsense in usural |array|\slash|tabular| even if they are included in
% |\longtable|, we define |\adl@discard| as |\adl@connect| so that it
% transforms itself into |\adl@connect| when it is added to
% |\adl@rowsL|\slash|R| by |\xdef|.  Note that |\adl@discard| is made
% |\let|-equal to |\relax| to inhibit the transformation at the beginning of
% |longtable| environment.
% \end{macro}\end{macro}
% 
% \changes{v1.4-2-1}{1999/06/25}
%	{Modified for new column counting in preamble construction.}
%
% Then, we set to |\adl@columns| to the value of |\adl@ncol| {\em locally}.
% As explained above, this has an effect with \textsf{array} because
% |\adl@arrayinit| is called {\em after} the preamble is generated.  Without
% \textsf{array}, on the other hand, this assignment has no effect but safe
% because it is included in a group of |\vbox| etc.
% 
% \changes{v1.4-2-3}{1999/06/25}
%	{Modified to call \cs{adl@inactivate}.}
% \changes{v1.5-3}{2003/07/11}
%	{Move \cs{adl@inactivate} from \cs{adl@arrayinit} to \cs{@array}.}
%
% \end{macro}
% \SpecialIndex{\adl@height}
% \SpecialIndex{\adl@depth}
% \SpecialIndex{\adl@currentcolumn}
% \SpecialIndex{\adl@totalheight}
% \SpecialIndex{\adl@heightsave}
% \SpecialIndex{\adl@depthsave}
% \SpecialIndex{\adl@currentcolumnsave}
% \SpecialIndex{\adl@totalheightsave}
%
%    \begin{macrocode}
\def\adl@arrayinit{%
        \adl@arraysave
        \global\adl@height\z@ \global\adl@depth\z@
        \global\adl@currentcolumn\@ne \global\adl@totalheight\z@
        \gdef\adl@rowsL{}\gdef\adl@rowsR{}\gdef\adl@colsL{}\gdef\adl@colsR{}%
        \let\@elt\relax \let\adl@connect\relax \def\adl@discard{\adl@connect}%
        \adl@columns\adl@ncol}
\def\adl@arraysave{%
        \adl@heightsave\adl@height
        \adl@depthsave\adl@depth
        \adl@currentcolumnsave\adl@currentcolumn
        \adl@totalheightsave\adl@totalheight
        \let\adl@rowsLsave\adl@rowsL
        \let\adl@rowsRsave\adl@rowsR
        \let\adl@colsLsave\adl@colsL
        \let\adl@colsRsave\adl@colsR}

%    \end{macrocode}
%
% \begin{macro}{\adl@inactivate}
% \changes{v1.4-2-3}{1999/06/25}
%	{Introduced to inactivate \cs{@arrayclassz} etc.}
% \changes{v1.5-3}{2003/07/11}
%	{Change \cs{adl@inactivate} caller to \cs{@array}.}
% \changes{v1.7-1}{2004/5/21}
% 	{Remove \cs{cline} because our own version is needed for null-wide.}
% If |\ADLinactivate| has effect and thus |\ifadl@inactive| is true, the
% macro |\adl@inactivate| is called from |\@array|\footnote{
%
% Before v1.53, \cs{adl@inactivate} was called from \cs{adl@arrayinit} and
% thus invokded {\em after} the preamble of \textsf{array} is built.  This
% was incorrect of course and made inactive version of \texttt{p},
% \texttt{m} and \texttt{b} produce nothing.}.
%
% This |\let|-s the following control sequences be equal to their
% counterparts in \LaTeX{} and/or \textsf{array} package.
%
% \changes{v1.54-2}{2003/08/25}
%	{Add \cs{adl@argcr} to inactivation.}
%
% \begin{itemize}\item[]
% \SpecialIndex{\@arrayclassz}
% \SpecialIndex{\@tabclassz}
% \SpecialIndex{\@classz}
% \SpecialIndex{\@@startpbox}
% \SpecialIndex{\@@endpbox}
% \SpecialIndex{\@endpbox}
% \SpecialIndex{\adl@cr}
% \SpecialIndex{\adl@argcr}
% \SpecialIndex{\adl@endarray}
% |\@arrayclassz|\quad
% |\@tabclassz|\quad
% |\@classz|\quad
% |\@@startpbox|\quad
% |\@@endpbox|\\
% |\@endpbox|\quad
% |\adl@cr|\quad
% |\adl@argcr|\quad
% |\adl@endarray|
% \end{itemize}
%
% Note that we have to inactivate both |\@@endpbox| for \LaTeX{} and
% |\@endpbox| for \textsf{array}, while |\@startpbox| for \textsf{array} is
% not necessary because it is unmodified.  Also note that\break|\@classz|
% has to be |\let|-equal to |\adl@org@classz| only if \textsf{array} is in
% use, because \LaTeX{} does not define |\@classz| but refers to it which is
% either |\@arrayclassz| or |\@tabclassz|.  Yet another remark is that we
% have to conceal |\cr| for |\adl@cr|\slash|\adl@argcr| and |\crcr| for
% |\adl@endarray| by bracing them from \TeX's |\halign| mechanism that
% searches them when an |array|\slash|tabular| has an nested
% |array|\slash|tabular|.  This could be done by a tricky |\let|-assignment
% such as;
%
% \begin{quote}
% |\iffalse{\let\adl@cr\cr \iffalse}\fi|
% \end{quote}
%
% but we simply use |\def| instead of |\let| because of clarity.
%
% We also |\let| the following be {\it no-operation} or their inactive
% versions.
%
% \begin{itemize}\item[]
% \SpecialIndex{\adl@hline}
% \SpecialIndex{\adl@ihdashline}
% \SpecialIndex{\adl@cdline}
% \SpecialIndex{\adl@@vlineL}
% \SpecialIndex{\adl@@vlineR}
% \SpecialIndex{\adl@vlineL}
% \SpecialIndex{\adl@vlineR}
% |\adl@hline|\quad
% |\adl@ihdashline|\quad
% |\adl@cdline|\quad
% |\adl@@vlineL|\quad
% |\adl@@vlineR|\\
% |\adl@vlineL|\quad
% |\adl@vlineR|
% \end{itemize}
% Note that we have to inactivate both |\adl@@vlineL| and |\adl@vlineL|,
% because the latter is referred to when \textsf{array} is in use while the
% former is done otherwise.  Their |R| relatives are also inactivated by
% the same reason.
% \end{macro}
%
% \SpecialIndex{\@arrayclassz}
% \SpecialIndex{\@tabclassz}
% \SpecialIndex{\@classz}
% \SpecialIndex{\@@startpbox}
% \SpecialIndex{\@@endpbox}
% \SpecialIndex{\@endpbox}
% \SpecialIndex{\adl@cr}
% \SpecialIndex{\adl@argcr}
% \SpecialIndex{\adl@endarray}
% \SpecialIndex{\adl@hline}
% \SpecialIndex{\adl@ihdashline}
% \SpecialIndex{\adl@cdline}
% \SpecialIndex{\adl@@vlineL}
% \SpecialIndex{\adl@@vlineR}
% \SpecialIndex{\adl@vlineL}
% \SpecialIndex{\adl@vlineR}
% \SpecialIndex{\adl@org@arrayclassz}
% \SpecialIndex{\adl@org@tabclassz}
% \SpecialIndex{\adl@org@classz}
% \SpecialIndex{\adl@org@@startpbox}
% \SpecialIndex{\adl@org@@endpbox}
% \SpecialIndex{\adl@org@endpbox}
% \SpecialIndex{\adl@cr}
% \SpecialIndex{\adl@argcr}
% \SpecialIndex{\adl@org@cline}
% \SpecialIndex{\adl@inactivehdl}
% \SpecialIndex{\adl@inactivecdl}
% \SpecialIndex{\adl@inactivevl}
%    \begin{macrocode}
\def\adl@inactivate{%
        \let\@arrayclassz\adl@org@arrayclassz
        \let\@tabclassz\adl@org@tabclassz
        \ifadl@usingarypkg \let\@classz\adl@org@classz \fi
        \let\@@startpbox\adl@org@@startpbox
        \let\@@endpbox\adl@org@@endpbox
        \let\@endpbox\adl@org@endpbox
        \def\adl@cr{\cr}%
        \def\adl@argcr##1{\cr}%
        \def\adl@endarray{\crcr}%
        \let\adl@hline\@gobbletwo
        \let\adl@ihdashline\adl@inactivehdl
        \let\adl@cdline\adl@inactivecdl
        \let\adl@@vlineL\adl@inactivevl
        \let\adl@@vlineR\adl@inactivevl
        \let\adl@vlineL\adl@inactivevl
        \let\adl@vlineR\adl@inactivevl}
%    \end{macrocode}
% 
% \begin{macro}{\adl@activate}
% \changes{v1.54-2}{2003/08/25}
%	{Introduced to activate \cs{@arrayclassz} etc.\ again.}
%
% On the other hand, if |\ifadl@inactive| is false, the macro |\adl@activate| 
% is called from |\@array| to make inactivated macros active again in order
% to cope with the case in which an inactive \textsf{array}\slash
% \textsf{tabular} has active children in it\footnote{
%
% Before v1.54, an active \textsf{array}\slash\textsf{tabular} in an
% inactive parent was not activated.}.
%
% To do that, |\adl@activate| makes |\@arrayclassz| etc.\ |\let|-equal to
% their active version |\adl@act@arrayclassz| etc.\ which will be defined
% (|\let|-equal to) as our own |\@arrayclassz| etc.\ in \S\ref{sec:imp-sh}.
% \end{macro}
%
% \SpecialIndex{\@arrayclassz}
% \SpecialIndex{\@tabclassz}
% \SpecialIndex{\@classz}
% \SpecialIndex{\@@startpbox}
% \SpecialIndex{\@@endpbox}
% \SpecialIndex{\@endpbox}
% \SpecialIndex{\adl@cr}
% \SpecialIndex{\adl@argcr}
% \SpecialIndex{\adl@endarray}
% \SpecialIndex{\adl@hline}
% \SpecialIndex{\adl@ihdashline}
% \SpecialIndex{\adl@cdline}
% \SpecialIndex{\adl@@vlineL}
% \SpecialIndex{\adl@@vlineR}
% \SpecialIndex{\adl@vlineL}
% \SpecialIndex{\adl@vlineR}
% \SpecialIndex{\adl@act@arrayclassz}
% \SpecialIndex{\adl@act@tabclassz}
% \SpecialIndex{\adl@act@classz}
% \SpecialIndex{\adl@act@@startpbox}
% \SpecialIndex{\adl@act@@endpbox}
% \SpecialIndex{\adl@act@endpbox}
% \SpecialIndex{\adl@act@cr}
% \SpecialIndex{\adl@act@argcr}
% \SpecialIndex{\adl@act@cline}
% \SpecialIndex{\adl@act@endarray}
% \SpecialIndex{\adl@act@hline}
% \SpecialIndex{\adl@act@ihdashline}
% \SpecialIndex{\adl@act@cdline}
% \SpecialIndex{\adl@act@@vlineL}
% \SpecialIndex{\adl@act@@vlineR}
%    \begin{macrocode}
\def\adl@activate{%
        \let\@arrayclassz\adl@act@arrayclassz
        \let\@tabclassz\adl@act@tabclassz
        \ifadl@usingarypkg \let\@classz\adl@act@classz \fi
        \let\@@startpbox\adl@act@@startpbox
        \let\@@endpbox\adl@act@@endpbox
        \let\@endpbox\adl@act@endpbox
        \let\adl@cr\adl@act@cr
        \let\adl@argcr\adl@act@argcr
        \let\adl@endarray\adl@act@endarray
        \let\adl@hline\adl@act@hline
        \let\adl@ihdashline\adl@act@ihdashline
        \let\adl@cdline\adl@act@cdline
        \let\adl@@vlineL\adl@act@@vlineL
        \let\adl@@vlineR\adl@act@@vlineR
        \let\adl@vlineL\adl@act@@vlineL
        \let\adl@vlineR\adl@act@@vlineR}

%%^L
%    \end{macrocode}
%
% \changes{v1.6}{2003/8/25}
%	{Add a summary of activation/inactivation.}
% The summary of the activation and inactivation is shown in
% Table~\ref{tab:act}.
% 
% 
% 
% \begin{table}
% \catcode`\|12
% \caption{Active and Inactive Operations}\label{tab:act}
% \def\!{\texttt{\char`\\\char`\\}}
% \centerline{\begin{tabular}[t]{|l||l|l|}\hline
% command&active&inactive\\\hline\hline
% \tt l c r&&\\
% \qquad with \textsf{array}&	\cs{adl@act@classz}&
%				\cs{adl@org@classz}\\
% \qquad without \textsf{array}&\cs{adl@act@tabclassz}&
%				\cs{adl@org@tabclassz}\\
% &				\cs{adl@act@arrayclassz}&
%				\cs{adl@org@arrayclassz}\\\hline
% {\tt p m b} (open)&&\\
% \qquad with \textsf{array}&	\cs{adl@act@classz}&
%				\cs{adl@org@classz}\\
% \qquad without \textsf{array}&\cs{adl@act@@startpbox}&
%				\cs{adl@org@@startpbox}\\\hline
% {\tt p m b} (close)&		\cs{adl@act@@endpbox}&
%				\cs{adl@org@@endpbox}\\\hline
% \tt \char`\|/:/;&		\cs{adl@act@@vlineL/R}&
%				\cs{adl@inactivevl}\\\hline
% \!&				$\to$\cs{adl@act@(arg)cr}&
%				$\to$\cs{cr}\\\hline
% \cs{hline}&			$\to$\cs{adl@act@hline}&
%				$\to$\cs{@gobbletwo}\\\hline
% \cs{hdashline}&		$\to$\cs{adl@act@ihdashline}&
%				$\to$\cs{adl@inactivehdl}\\\hline
% \cs{cdashline}&		$\to$\cs{adl@act@cdline}&
%				$\to$\cs{adl@inactivecdl}\\\hline
% \end{tabular}}
% \end{table}
% 
% 
%
% \subsection{Making Preamble}
% \label{sec:imp-pream}
%
% Each preamble character is converted to a part of |\halign|'s preamble as
% follows.
%
% \begin{itemize}
% \item
% \begin{macro}{\adl@colhtdp}
% \changes{v1.5-2}{2002/06/26}
%	{The pseudo-formal description of $\langle\hbox{\it put-lrc}\rangle$
%	 is modified.}
% \SpecialIndex{\adl@box}
% \SpecialIndex{\adl@height}
% \SpecialIndex{\adl@depth}
% `|l|', `|r|' and `|c|' are converted to the following \meta{lrc}.
%
% \begingroup \catcode`\|12
% \begin{eqnarray*}
% \meta{lrc}&::=&
%	[\cs{hfil}]\meta{put-lrc}[\cs{hfil}]\\
% \meta{put-lrc}&::=&
%	\cs{setbox}\cs{adl@box}\cs{hbox}\Meta{lrc-contents}\\
% &&	\cs{adl@colhtdp }\cs{unhbox}\cs{adl@box}\\
% \meta{lrc-contents}&::=&
%	\hbox{\texttt{\$\cs{relax}\#\$}}\ |\\
% &&	\hbox{\texttt{\#\cs{unskip}}}\\
% \end{eqnarray*}
% \endgroup
%
% That is, the content of a column is at first packed into the |\box|
% register |\adl@box|, then its height and depth are compared to
% |\adl@height| and |\adl@depth| by the macro |\adl@colhtdp|, and finally
% the box is put with leading and/or trailing |\hfil|.
% \end{macro}
%
% \item
% \begin{macro}{\adl@vlineL}
% \begin{macro}{\adl@vlineR}
% \def\G{{\mit\Gamma}}
% `\verb!|!', `|:|' and |;{|\meta{dash}|/|\meta{gap}|}| are converted to the
% following \meta{vline}.
%
% \begingroup \catcode`\|12
% \begin{eqnarray*}
% \meta{vline}&::=&
%	[\cs{hskip}\cs{doublerulesep}]\meta{vline-LR}\\
% \meta{vline-LR}&::=&
%	\cs{adl@vlineL}\Meta{$\G_d$}\Meta{$\G_g$}\Meta{c}
%		\hbox{{\tt\char`\{}\meta{d}/\meta{g}{\tt\char`\}}}\ |\\
% &&	\cs{adl@vlineR}\Meta{$\G_d$}\Meta{$\G_g$}\Meta{c}
%		\hbox{{\tt\char`\{}\meta{d}/\meta{g}{\tt\char`\}}}\\
% \meta{d}&::=&
%	0\ |&\hbox to.3\textwidth{\ldots\ for `\texttt{|}'\hfil}\cr
% &&	\cs{dashlinedash}\ |&
%		\hbox to.3\textwidth{\ldots\ for `\texttt{:}'\hfil}\cr
% &&	\meta{dash}&
%		\hbox to.3\textwidth{\ldots\ for `\texttt{;}'\hfil}\cr
% \meta{g}&::=&
%	0\ |&\hbox to.3\textwidth{\ldots\ for `\texttt{|}'\hfil}\cr
% &&	\cs{dashlinegap}\ |&
%		\hbox to.3\textwidth{\ldots\ for `\texttt{:}'\hfil}\cr
% &&	\meta{gap}&
%		\hbox to.3\textwidth{\ldots\ for `\texttt{;}'\hfil}\cr
% \end{eqnarray*}
% \endgroup
%
% \SpecialIndex{\CT@arc@}
% \SpecialIndex{\adl@dashgapcolor}
% Note that \meta{c} is the column number (leftmost is 1) where the
% character appears, and \meta{$\G_d$} and \meta{$\G_g$} is the
% color of dashs and gaps specified in |\CT@arc@| and |\adl@dashgapcolor|.
% \end{macro}\end{macro}
% \end{itemize}
%
% \SpecialIndex{\adl@currentcolumn}
% Additionally, each column except for the last one has;
%
% \begin{quote}
% |\global\advance\adl@currentcolumn\@ne|
% \end{quote}
%
% just before |&| to increment |\adl@currentcolumn|.  Other features, such
% as inserting spaces of |\arraycolsep|\slash|\tabcolsep|, are as same as
% original scheme.  This means that |@|\Meta{text} and |!|\Meta{text} of
% \textsf{array} are {\em not} handled specially although it could interfere
% with drawing vertical lines.  Therefore, we have the problem 1 shown in
% \S\ref{sec:prob}, which is very hard to solve.  Note that the measurement
% of the column of `|p|' of \LaTeX{} original is done by (modified)
% |\@@startpbox| and |\@@endpbox| and thus the preamble for `|p|' is not
% modified.  In the case with \textsf{array}, however, the preambles for
% `|p|' and its relatives `|m|' and `|b|' are modified to set |\adl@box| to
% the box for them.
%
% \begin{macro}{\adl@mkpream}
% \begin{macro}{\@mkpream}
% \changes{v1.4-2-1}{1999/06/25}
%	{Modified for new column counting and control sequence redefinition.}
% \changes{v1.71}{2004/08/31}
%	{Modified to nullify \cs{adl@startmbox} and \cs{adl@endmbox} for
%	\textsf{array's} \texttt{m}-columns.}
% \SpecialIndex{\adl@colhtdp}
% \SpecialIndex{\adl@vlineL}
% \SpecialIndex{\adl@vlineR}
% \SpecialIndex{\adl@startmbox}
% \SpecialIndex{\adl@endmbox}
% \SpecialIndex{\adl@preaminit}
% \SpecialIndex{\afterassignment}
%
% To make the preamble shown above, |\@mkpream| is modified to |\let|
% control sequences |\adl@colhtpd|, |\adl@vlineL| and |\adl@vlineR| be
% |\relax| in order to keep them from being expanded by |\edef|\slash
% |\xdef| for the preamble construction.  The control sequences
% |\adl@startmbox| and |\adl@endmbox| for \texttt{m}-columns of
% \textsf{array} are also made |\let|-equal to |\relax|.
%
% Giving them their own definition is done by |\adl@preaminit| that is
% called using |\afterassignment| after |\@preamble| is made by
% |\adl@mkpream|, the original version of |\@mkpream|.  If \textsf{array} is
% not in use, |\@mkpream| is followed by an |\edef| of |\@preamble| to add
% |\ialign| etc.\ and thus |\adl@preaminit| is properly called {\em after}
% this final {\em assignment} to make |\@preamble|.
%
% With \textsf{array}, on the other hand, calling |\adl@preaminit| is safe
% because |\@mkpream| is followed by |\xdef| for |\@preamble| too, but has
% no effect because it is in the group for |\@mkpream|.  This grouping,
% however, gives us an easier way to give those control sequences their own
% definition.  That is, we simply initiate them with the definitions that
% will be regained when the group is closed.
% 
% \SpecialIndex{\adl@ncol}
% \SpecialIndex{\adl@columns}
% \SpecialIndex{\ifadl@leftrule}
% The modified |\@mkpream| also initializes |\adl@ncol| and
% |\ifadl@leftrule|, and set |\adl@columns| to the value of |\adl@ncol|
% locally after the preamble is made.  This has an effect in the case
% without \textsf{array} because the body of |array|\slash|tabular| is in
% the same grouping context of |\@mkpream|.  With \textsf{array}, on the
% other hand, this assignment has no effect but safe because it is included
% in a group of |\@mkpream|'s own.
% \end{macro}\end{macro}
%
%    \begin{macrocode}

%% Making Preamble

\let\adl@mkpream\@mkpream
\def\@mkpream#1{\let\adl@colhtdp\relax
        \let\adl@vlineL\relax \let\adl@vlineR\relax
        \let\adl@startmbox\relax \let\adl@endmbox\relax
        \global\adl@ncol\@ne \adl@leftruletrue
        \adl@mkpream{#1}\adl@columns\adl@ncol \afterassignment\adl@preaminit}

%    \end{macrocode}
%
% \begin{macro}{\@addamp}
% \changes{v1.4-2-1}{1999/06/25}
%	{Modified for new column counting in preamble construction.}
% \SpecialIndex{\adl@currentcolumn}
% \SpecialIndex{\adl@ncol}
%
% The macro |\@addamp| is also modified to add the code for incrementing the
% counter |\adl@currentcolumn| to |\@preamble| with |&|.  The counter
% |\adl@ncol| is also incremented by |\@addamp| so that we can refer to its
% value as \meta{c} of |\adl@vlineL|\slash|R|\@.  This increment is done
% |\global|-ly in order that we locally set |\adl@columns| to the counting
% result outside of the group for |\@mkpream| of \textsf{array}.  Therefore,
% whether or not \textsf{array} is in use, |\adl@columns| will have a
% correct value and will be correctly referred to by |\hdashline| to know how
% many columns are specified in the preamble.  Note that this |\global|
% assignment is safe because the life time of |\adl@ncol| is same as that of
% |\@preamble|.
% \end{macro}
%
%    \begin{macrocode}
\def\@addamp{\if@firstamp\@firstampfalse \else
        \@addtopreamble{\global\advance\adl@currentcolumn\@ne &}%
        \global\advance\adl@ncol\@ne \fi}

%    \end{macrocode}
%
% \SpecialIndex{\ifadl@usingarypkg}
% Since the implementation of |\@testpach| and macros for class-0 characters 
% (i.e. |l|, |r| and |c|) is completely different between \LaTeX{} and
% \textsf{array}, we have to have two versions switched by
% |\adl@usingarypkg|.
% 
% \subsubsection*{With \textsf{array}}
% 
% \begin{macro}{\@testpach}
% \changes{v1.4-2-1}{1999/06/25}
%	{The version for \textsf{array} is introduced.}
% \SpecialIndex{\@arrayrule}
% \SpecialIndex{\adl@arrayrule}
% \SpecialIndex{\adl@arraydashrule}
% \SpecialIndex{\adl@argarraydashrule}
%
% Although we introduced two preamble characters `|:|' and `|;|', we did not
% introduce new character {\em class} because we want to minimize the
% modification of original codes.  Therefore, `|:|' and `|;|' is classified
% into class-1 together with `\verb!|!'\@.  Since these characters obviously
% have their own appropriate operations, |\@testpach| is modified so that
% |\@arrayrule|, which is invoked from |\@mkpream| in the case of class-1
% character, is |\let|-equal to the macro corresponding to each character.
% \end{macro}
%
%    \begin{macrocode}
\ifadl@usingarypkg
\def\@testpach{\@chclass
 \ifnum \@lastchclass=6 \@ne \@chnum \@ne \else
  \ifnum \@lastchclass=7 5 \else
   \ifnum \@lastchclass=8 \tw@ \else
    \ifnum \@lastchclass=9 \thr@@
   \else \z@
   \ifnum \@lastchclass = 10 \else
   \edef\@nextchar{\expandafter\string\@nextchar}%
   \@chnum
   \if \@nextchar c\z@ \else
    \if \@nextchar l\@ne \else
     \if \@nextchar r\tw@ \else
   \z@ \@chclass
   \if\@nextchar |\@ne \let\@arrayrule\adl@arrayrule \else
   \if\@nextchar :\@ne \let\@arrayrule\adl@arraydashrule \else
   \if\@nextchar ;\@ne \let\@arrayrule\adl@argarraydashrule \else
    \if \@nextchar !6 \else
     \if \@nextchar @7 \else
      \if \@nextchar <8 \else
       \if \@nextchar >9 \else
  10
  \@chnum
  \if \@nextchar m\thr@@\else
   \if \@nextchar p4 \else
    \if \@nextchar b5 \else
   \z@ \@chclass \z@ \@preamerr \z@ \fi \fi \fi \fi \fi \fi
   \fi \fi  \fi  \fi  \fi  \fi  \fi \fi \fi \fi \fi \fi}

%    \end{macrocode}
% 
% \begin{macro}{\@classz}
% \changes{v1.4-2-1}{1999/06/25}
%	{Introduced because \textsf{array} uses it.}
% \changes{v1.71}{2004/08/31}
%	{Modified to call \cs{adl@startmbox} and \cs{adl@endmbox} for
%	\textsf{array}'s \texttt{m}-columns.}
% \SpecialIndex{\adl@putlrc}
% \SpecialIndex{\adl@box}
% \SpecialIndex{\adl@startmbox}
% \SpecialIndex{\adl@endmbox}
%
% In \textsf{array}, |array| and |tabular| share common macro for class-0
% named |\@classz|, which also generates the preamble for `|p|', `|m|' and
% `|b|'.  Thus we modify it to measure the height and depth of the
% class-0 column by the macro |\adl@putlrc|, and to set |\adl@box| to the
% box for `|p|' and its relatives.  Note that a |m|-type preamble
% (|@chnum|${}=3$) has to be generated to have |\adl@startmbox| and
% |\adl@endmbox| in it because a |\vcenter| construct cannot be assigned
% to |\adl@box| by |\setbox| directly.
% \end{macro}
%
%    \begin{macrocode}
\def\@classz{\@classx
   \@tempcnta \count@
   \prepnext@tok
   \@addtopreamble{\ifcase \@chnum
      \hfil
      \adl@putlrc{\d@llarbegin \insert@column \d@llarend}\hfil \or
      \hskip1sp\adl@putlrc{\d@llarbegin \insert@column \d@llarend}\hfil \or
      \hfil\hskip1sp\adl@putlrc{\d@llarbegin \insert@column \d@llarend}\or
   \setbox\adl@box\hbox \adl@startmbox{\@nextchar}\insert@column
        \adl@endmbox\or
   \setbox\adl@box\vtop \@startpbox{\@nextchar}\insert@column \@endpbox \or
   \setbox\adl@box\vbox \@startpbox{\@nextchar}\insert@column \@endpbox
  \fi}\prepnext@tok}
%    \end{macrocode}
% 
% \begin{macro}{\adl@class@start}
% \changes{v1.4-2-1}{1999/06/25}
%	{Introduced for class number identification.}
% \begin{macro}{\adl@class@iiiorvii}
% \changes{v1.4-2-1}{1999/06/25}
%	{Introduced for class number identification.}
%
% Another stuff for compatibility is to refer to the class number for the
% beginning of preamble which is different between \LaTeX{} and
% \textsf{array}, and that for `|p|' or `|@|' to get the argument of `|;|' as 
% explained later.  In the case with \textsf{array}, the former is class-4
% and we use `|@|' (class-7) for the latter.
% \end{macro}\end{macro}
%
%    \begin{macrocode}
\def\adl@class@start{4}
\def\adl@class@iiiorvii{7}

%    \end{macrocode}
% 
% \subsubsection*{Without \textsf{array}}
%
% \begin{macro}{\@testpach}
% \SpecialIndex{\@arrayrule}
% \SpecialIndex{\adl@arrayrule}
% \SpecialIndex{\adl@arraydashrule}
% \SpecialIndex{\adl@argarraydashrule}
%
% The reason why and how we modify |\@testpach| of \LaTeX{} is same as those 
% of \textsf{array}.
% \end{macro}
%
%    \begin{macrocode}
\else
\def\@testpach#1{\@chclass \ifnum \@lastchclass=\tw@ 4\relax \else
        \ifnum \@lastchclass=\thr@@ 5\relax \else
                \z@ \if #1c\@chnum \z@ \else
                    \if #1l\@chnum \@ne \else
                    \if #1r\@chnum \tw@ \else
                \@chclass
                    \if #1|\@ne \let\@arrayrule\adl@arrayrule \else
                    \if #1:\@ne \let\@arrayrule\adl@arraydashrule \else
                    \if #1;\@ne \let\@arrayrule\adl@argarraydashrule \else
                    \if #1@\tw@ \else
                    \if #1p\thr@@ \else \z@ \@preamerr 0\fi
        \fi  \fi  \fi  \fi  \fi  \fi  \fi  \fi  \fi}

%    \end{macrocode}
%
% \begin{macro}{\@arrayclassz}
% \begin{macro}{\@tabclassz}
% \SpecialIndex{\adl@putlrc}
%
% Since \LaTeX{} has two macros for class-0, one for |array| and the other
% for |tabular|, we have to modify both.  Since the box for `|p|' is opened
% by |\@@startpbox|, however, we may not worry about it.
% \end{macro}\end{macro}
%
%    \begin{macrocode}
\def\@arrayclassz{\ifcase \@lastchclass \@acolampacol \or \@ampacol \or
                \or \or \@addamp \or
                \@acolampacol \or \@firstampfalse \@acol \fi
        \edef\@preamble{\@preamble
                \ifcase \@chnum
                    \hfil\adl@putlrc{$\relax\@sharp$}\hfil
                \or \adl@putlrc{$\relax\@sharp$}\hfil
                \or \hfil\adl@putlrc{$\relax\@sharp$}\fi}}
\def\@tabclassz{\ifcase \@lastchclass \@acolampacol \or \@ampacol \or
                \or \or \@addamp \or
                \@acolampacol \or \@firstampfalse \@acol \fi
        \edef\@preamble{\@preamble
        \ifcase \@chnum
                    \hfil\adl@putlrc{\@sharp\unskip}\hfil
                \or \adl@putlrc{\@sharp\unskip}\hfil
                \or \hfil\hskip\z@ \adl@putlrc{\@sharp\unskip}\fi}}
%    \end{macrocode}
%
% \begin{macro}{\adl@class@start}
% \changes{v1.4-2-1}{1999/06/25}
%	{Introduced for class number identification.}
% \begin{macro}{\adl@class@iiiorvii}
% \changes{v1.4-2-1}{1999/06/25}
%	{Introduced for class number identification.}
% In \LaTeX{}, the beginning of preamble is class-6 and we use `|p|'
% (class-3) to get the argument of `|;|'.
% \end{macro}\end{macro}
%
%    \begin{macrocode}
\def\adl@class@start{6}
\def\adl@class@iiiorvii{3}
\fi

%    \end{macrocode}
% 
% Hereafter, codes for \LaTeX{} and \textsf{array} are common again.
%
% \begin{macro}{\adl@putlrc}
% \changes{v1.5-2}{2002/06/26}
%	{\cs{adl@putlrc} must do \cs{unhbox}\cs{adl@box} to make glues
%	 effective.} 
% \SpecialIndex{\adl@box}
% \SpecialIndex{\adl@colhtdp}
%
% The macro |\adl@putlrc| is for class-0 preamble characters to set
% |\adl@box| to the contents of a column, measure its height\slash depth by
% |\adl@colhtdp| and put the box by |\unhbox| (not by |\box|) in order to
% make the glues in the contents effective.
% \end{macro}
%
%    \begin{macrocode}
\def\adl@putlrc#1{\setbox\adl@box\hbox{#1}\adl@colhtdp \unhbox\adl@box}

%    \end{macrocode}
% 
% \begin{macro}{\adl@arrayrule}
% \changes{v1.4-2-1}{1999/06/25}
%	{Modified to replace \cs{adl@columns} with \cs{adl@ncol}.}
% \begin{macro}{\adl@arraydashrule}
% \changes{v1.4-2-1}{1999/06/25}
%	{Modified to replace \cs{adl@columns} with \cs{adl@ncol}.}
% \begin{macro}{\adl@argarraydashrule}
% \changes{v1.4-2-1}{1999/06/25}
%	{Modified to replace \cs{adl@columns} with \cs{adl@ncol}.}
% \changes{v1.4-2-1}{1999/06/25}
%	{Modified to pretend \texttt{p} or \texttt{@} depending on if
%	\textsf{array} is in use.}
% \begin{macro}{\adl@xarraydashrule}
% \changes{v1.4-2-1}{1999/06/25}
%	{Modified to refer \cs{adl@class@start} rather than \LaTeX's 6.}
% \changes{v1.4-2-2}{1999/06/25}
%	{Modified to add invisible rule of \cs{arrayrulewidth} wide if
%	\cs{ADLsome}\penalty\exhyphenpenalty\texttt{wide}.}
% \changes{v1.7-3}{2004/5/21}
%	{Modified to add color arguments to \cs{adl@vline}\texttt{L}\slash
%	\texttt{R}}
% \SpecialIndex{\@lastchclass}
% \SpecialIndex{\adl@class@start}
% \SpecialIndex{\ifadl@leftrule}
% \SpecialIndex{\adl@vlineL}
% \SpecialIndex{\adl@vlineR}
% \SpecialIndex{\adl@ncol}
% \SpecialIndex{\CT@arc@}
% \SpecialIndex{\adl@dashgapcolor}
% \SpecialIndex{\color}
% \def\G{{\mit\Gamma}}
%
% The preamble parts for vertical solid- and dash-lines are constructed by
% the macros |\adl@arrayrule| for `\verb!|!', |\adl@arraydashrule| for
% `|:|', and |\adl@argarraydashrule| for `|;|'.\@  The macro;
%
% \begin{quote}
% |\adl@xarraydashrule|\Meta{$c^L$}\Meta{$c^R$}|{|\meta{d}|/|\meta{g}|}|
% \end{quote}
%
% is invoked by them to perform common operations.  It at first checks the
% preamble character is the first element of the preamble
% (|\@lastchclass|${}=|\adl@class@start|$) or it follows another character
% for vertical line (|\@lastchclass|${}=1$).  If this is not satisfied, the
% vertical line is put at the right edge of a column and thus
% |\ifadl@leftrule| is set to false.  Then it adds
% |\adl@vlineL|\Meta{$\G_d$}\Meta{$\G_g$}
% \Meta{$c^L$}|{|\meta{d}|/|\meta{g}|}| if |\ifadl@leftrule| is
% true indicating the vertical line will appear at the left edge of the
% column \meta{$c^L$}, or |\adl@vlineR|\Meta{$\G_d$}\Meta{$\G_g$}
% \Meta{$c^R$}|{|\meta{d}|/|\meta{g}|}| otherwise.  Note that \meta{$c^L$}
% is always 1 for {\em main} preamble while \meta{$c^R$} is the column
% number given by |\adl@ncol|, but \meta{$c^L$} may not be 1 for the
% preamble of |\multicolumn| as described in \S\ref{sec:imp-mc}.  Also note
% that $\G_d$ and $\G_g$ are |\CT@arc@| and |\adl@dashgapcolor|
% respectively whose bodies are |\color| for dashes and gaps specified by
% |\arrayrulecolor| and |\dashgapcolor|, or |\relax| if they are not colored.
% 
% In addition, an invisible |\vrule| of |\arrayrulewidth| wide is added if
% both |\ADL#some#wide| and |\ADLactivate| are in effect, i.e.\ both
% |\ifadl@zwrule| and |\ifadl@inactive| are false, to keep a space for
% the vertical line having {\em real} width.
% \end{macro}\end{macro}\end{macro}\end{macro}
%
% \begin{macro}{\adl@classv}
% \begin{macro}{\adl@classvfordash}
% \SpecialIndex{\@nextchar}
% \SpecialIndex{\@chclass}
% \SpecialIndex{\@classv}
% \SpecialIndex{\adl@class@iiiorvii}
%
% The argument of `|;|' is not provided by |\adl@argarraydashrule| but is
% directly passed from the preamble text through |\@nextchar|.  This direct
% passing is implemented by the following trick.  The macro
% |\adl@argarraydashrule| set |\@chclass| to |\adl@class@iiiorvii| to 
% pretend it is for `|p|' if \textsf{array} is not in use, or `|@|'
% otherwise.  Then it temporally changes the definition of |\@classv|, which
% is incidentally for the argument of `|p|' and `|@|' in the case without/with
% \textsf{array} respectively, to |\adl@classvfordash| to process the
% argument of `|;|' rather than that of `|p|' or `|@|'\@.  Then
% |\adl@classvfordash| is invoked by |\@mkpream| and it adds the argument to
% |\@preamble|.  Finally, it restores the definition of |\@classv| and sets
% |\@chclass| to 1 to indicate that the last item is a vertical line
% specification.
% \end{macro}\end{macro}
%
%    \begin{macrocode}
\def\adl@arrayrule{%
        \adl@xarraydashrule
                {\@ne}{\adl@ncol}{{\z@/\z@}}}
\def\adl@arraydashrule{%
        \adl@xarraydashrule
                {\@ne}{\adl@ncol}%
                {{\dashlinedash/\dashlinegap}}}
\def\adl@argarraydashrule{%
        \adl@xarraydashrule
                {\@ne}{\adl@ncol}{}%
        \@chclass\adl@class@iiiorvii \let\@classv\adl@classvfordash}
\def\adl@xarraydashrule#1#2#3{%
        \ifnum\@lastchclass=\adl@class@start\else
        \ifnum\@lastchclass=\@ne\else
                \adl@leftrulefalse \fi\fi
        \ifadl@zwvrule\else \ifadl@inactive\else
                \@addtopreamble{\vrule\@width\arrayrulewidth
                        \@height\z@ \@depth\z@}\fi \fi
        \ifadl@leftrule
                \@addtopreamble{\adl@vlineL{\CT@arc@}{\adl@dashgapcolor}%
                        {\number#1}#3}%
        \else   \@addtopreamble{\adl@vlineR{\CT@arc@}{\adl@dashgapcolor}%
                        {\number#2}#3}\fi}
\let\adl@classv\@classv
\def\adl@classvfordash{\@addtopreamble{{\@nextchar}}\let\@classv\adl@classv
        \@chclass\@ne}

%%^L
%    \end{macrocode}
% 
%
% \subsection{Building Columns}
% \label{sec:imp-col}
%
% \begin{macro}{\adl@preaminit}
% \begin{macro}{\adl@colhtdp}
% \begin{macro}{\adl@vlineL}
% \begin{macro}{\adl@vlineR}
%
% If \textsf{array} is not in use, after the |\@preamble| is completed, the
% control sequences for macros in it should regain their own definition.
% The macro |\adl@preaminit| performs this operation for macros we
% introduced, |\adl@colhtdp|, |\adl@vlineL| and |\adl@vlineR|.  For the case 
% with \textsf{array}, we will call |\adl@preaminit| in \textsf{arydshln} to 
% initiate them with the definitions as described later.
% \end{macro}\end{macro}\end{macro}\end{macro}
%
%    \begin{macrocode}

%% Building Columns

\def\adl@preaminit{\let\adl@colhtdp\adl@@colhtdp
        \let\adl@vlineL\adl@@vlineL \let\adl@vlineR\adl@@vlineR}

%    \end{macrocode}
%
% \begin{macro}{\adl@@colhtdp}
% For the measurement of the height and depth of a row, |\adl@@colhtdp|
% compares |\adl@height| and |\adl@depth| to the height and depth of
% |\adl@box| which contains the main part of the column to be built, and
% |\global|-ly updates the registers if they are less.
% \end{macro}
%
%    \begin{macrocode}
\def\adl@@colhtdp{%
        \ifdim\adl@height<\ht\adl@box \global\adl@height\ht\adl@box \fi
        \ifdim\adl@depth<\dp\adl@box \global\adl@depth\dp\adl@box\fi}

%    \end{macrocode}
%
% \begin{macro}{\adl@@vlineL}
% \changes{v1.7-3}{2004/5/21}
% 	{Color arguments are added.}
% \begin{macro}{\adl@@vlineR}
% \changes{v1.7-3}{2004/5/21}
% 	{Color arguments are added.}
% \begin{macro}{\adl@@ivline}
% \changes{v1.7-3}{2004/5/21}
% 	{Invocations of \cs{adl@setcolor} are added.}
% \begin{macro}{\adl@setcolor}
% \changes{v1.7-3}{2004/5/21}
% 	{Introduced to color vertical lines.}
% \begin{macro}{\adl@nocolor}
% \changes{v1.7-3}{2004/5/21}
% 	{Introduced to examine if coloring is specified.}
% \begin{macro}{\adl@dashcolor}
% \changes{v1.7-3}{2004/5/21}
% 	{Introduced as the temporary variable of color specification of
%	 dashes.}
% \begin{macro}{\adl@gapcolor}
% \changes{v1.7-3}{2004/5/21}
% 	{Introduced as the temporary variable of color specification of
%	 gaps.}
% \SpecialIndex{\color}
% \SpecialIndex{\current@color}
% \def\G{{\mit\Gamma}} \let\g\gamma
% The macro
% |\adl@@vlineL|\meta{$\G_d$}\meta{$\G_g$}\meta{c}|{|\meta{d}|/|\meta{g}|}|
% adds the element
% $e=\langle c,d,g\rangle={}$
% |\@elt|\Meta{c}\Meta{d}\Meta{g}\Meta{$\g_d$}\Meta{$\g_g$} to the tail of
% the list |\adl@colsL| to construct $C^L_i$, where $\g_d$ and $\g_g$ are
% the color specifications given by  |\color| macros in $\G_d$ and $\G_g$.
% The macro |\add@@vlineR| performs similar operation but the element is
% added to the head of |\adl@colsR| for $C^R_i$ because it is processed
% right-to-left manner. The argument \meta{d} and \meta{g} are extracted by
% the macro |\adl@ivline| which converts given dimensional values of them to
% integers. It also sets \meta{d} and \meta{g} to 0 (i.e. solid-line) if one
% of given values are not positive, in order to make it sure that one dash
% segment has positive length.  Then it invokes |\adl@setcolor| to define
% |\adl@dashcolor| and |\adl@gapcolor| with the color specification of
% $\G_d$ and $\G_g$.  Since |\adl@setcolor| locally expands |\color| macro
% in $\G_d$ and $\G_g$ to define |\currnet@color| that becomes the body of
% |\adl@dashcolor| ($\g_d$) and |\adl@gapcolor| ($\g_g$) with expansion,
% different |\color| specifications of a color, such as |\color{red}|
% and |\color[rgb]{1,0,0}|, will produce a unified result such as
% |{rgb 1 0 0}|.  If $\G_d$ or $\G_g$ is |\relax| which is the body of
% |\adl@nocolor|, $\g_d$ or $\g_g$ is also |\relax| to indicate dashes are
% colored (or not colored) as done in outer world and gaps are transparent.
% \end{macro}\end{macro}\end{macro}\end{macro}\end{macro}\end{macro}\end{macro}
%
%    \begin{macrocode}
\def\adl@@vlineL#1#2#3#4{\adl@ivline#4\@nil{#1}{#2}%
        \xdef\adl@colsL{\adl@colsL
                \@elt{#3}{\number\@tempcnta}{\number\@tempcntb}%
                        {\adl@dashcolor}{\adl@gapcolor}}}
\def\adl@@vlineR#1#2#3#4{\adl@ivline#4\@nil{#1}{#2}%
        \xdef\adl@colsR{%
                \@elt{#3}{\number\@tempcnta}{\number\@tempcntb}%
                        {\adl@dashcolor}{\adl@gapcolor}%
                \adl@colsR}}
\def\adl@ivline#1/#2\@nil#3#4{%
        \@tempdima#1\relax \@tempcnta\@tempdima
        \@tempdima#2\relax \@tempcntb\@tempdima
        \ifnum\@tempcnta>\z@ \else \@tempcnta\z@ \@tempcntb\z@ \fi
        \ifnum\@tempcntb>\z@ \else \@tempcnta\z@ \@tempcntb\z@ \fi
        \adl@setcolor\adl@dashcolor{#3}\adl@setcolor\adl@gapcolor{#4}}
\def\adl@setcolor#1#2{\def\@tempa{#2}\ifx\@tempa\adl@nocolor \def#1{\relax}%
        \else{#2\xdef#1{\current@color}}\fi}
\def\adl@nocolor{\relax}
%    \end{macrocode}
%
% \begin{macro}{\adl@colhtdp}
% \changes{v1.4-2-1}{1999/06/25}
%	{Initialized by calling \cs{adl@preaminit}.}
% \begin{macro}{\adl@vlineL}
% \changes{v1.4-2-1}{1999/06/25}
%	{Initialized by calling \cs{adl@preaminit}.}
% \begin{macro}{\adl@vlineR}
% \changes{v1.4-2-1}{1999/06/25}
%	{Initialized by calling \cs{adl@preaminit}.}
% \SpecialIndex{\adl@preaminit}
%
% After |\adl@@colhtdp|, |\adl@@vlineL| and |\adl@@vlineR| are defined, 
% we call |\adl@preaminit| to |\let| their single |@| counterparts be equal
% to them.  Therefore, in the case with \textsf{array}, |\adl@colhtdp| etc.\
% are temporarily |\relax| when |\@preamble| is being generated in the group
% of |\@mkpream|, and regain their own definitions outside the group where
% the completed |\@preamble| is referred to.
% \end{macro}\end{macro}\end{macro}
%
%    \begin{macrocode}
\adl@preaminit

%    \end{macrocode}
% 
% \begin{macro}{\adl@inactivevl}
% \changes{v1.4-2-3}{1999/06/25}
%	{Introduced to emulate `\texttt{:}' and \texttt{;} by
%	\protect\VERTBAR.} 
% \changes{v1.7-3}{2004/5/21}
% 	{Modified to color the \cs{vline} by the first argument.}
% \SpecialIndex{\ifadl@zwvrule}
%
% If |\ADLinactivate| is in effect, |\adl@vlineL|\slash|R| and
% |\adl@@vlineL|\slash|R| are |\let|-equal
% to |\adl@inactivevl|.  This macro simply puts a |\vrule| by |\vline| with
% |\color| (or |\relax|) in its first argument and
% with\slash without negative |\hskip| of a half of |\arrayrulewidth| wide
% depending on |\ifadl@zwvrule|, discarding other arguments.
% \end{macro}
%
%    \begin{macrocode}
\def\adl@inactivevl#1#2#3#4{\ifadl@zwvrule \hskip-.5\arrayrulewidth \fi
        {#1\vline}\ifadl@zwvrule \hskip-.5\arrayrulewidth \fi}

%    \end{macrocode}
% 
% 
% \begin{macro}{\@@startpbox}
% \begin{macro}{\@@endpbox}
% \begin{macro}{\@endpbox}
% \changes{v1.4-2-1}{1999/06/25}
%	{Introduced because \textsf{array} uses it.}
% \begin{macro}{\adl@startmbox}
% \changes{v1.71}{2004/08/31}
%	{Introduced to the bug fix of \textsf{array}'s \texttt{m}-columns.}
% \begin{macro}{\adl@endmbox}
% \changes{v1.71}{2004/08/31}
%	{Introduced to the bug fix of \textsf{array}'s \texttt{m}-columns.}
% \SpecialIndex{\@finalstrut}
% \SpecialIndex{\@startpbox}
% \SpecialIndex{\adl@box}
% \SpecialIndex{\adl@colhtdp}
%
% The macros to make |\parbox| for `|p|' (and `|b|' of
% \textsf{array}), |\@@startpbox| and \hbox{|\@@endpbox|}, are modified for
% height\slash depth measurement.  The code for |\@@endpbox| is based on
% that of \LaTeXe{} to fix the bug of |\strut|-ing in \LaTeX-2.09, but
% |\@finalstrut| is manually expanded because it is not available in
% \LaTeX-2.09.
% 
% In \textsf{array}, |\@@endpbox| is not used but |\@endpbox| is.
% Therefore, we |\let| them be equal.  As for |\@startpbox|, however, we may 
% not worry about it because we have modified \hbox{|\@classz|} in
% \S\ref{sec:imp-pream} for the measurement.  However, we have to take care
% of |m|-type columns specially because its body |\vcenter| cannot be
% assigned directly to |\adl@box| by |\setbox|\footnote{
%
% The author had forgotten this fact until Morten H\o gholm pointed out it.
% Thanks Morten.}.
%
% Thus we enclose a |$\vcenter{|\ldots|}$| construct in a |\hbox| and assign
% it to |\adl@box|.  The macro |\adl@startmbox| opens the construct with
% \textsf{array}'s |\@startpbox|, while |\adl@endmbox| closes it calling
% |\adl@org@endpbox| which is the unmodified |\@endpbox| of \textsf{array}
% and measures the height and depth of the |\hbox| by |\adl@colhtdp|.
% \end{macro}\end{macro}\end{macro}\end{macro}\end{macro}
%
%    \begin{macrocode}
\def\@@startpbox#1{\setbox\adl@box\vtop\bgroup \hsize#1\@arrayparboxrestore}
\def\@@endpbox{\unskip \ifhmode \nobreak
        \vrule\@width\z@\@height\z@\@depth\dp\@arstrutbox \fi
        \par \egroup \adl@colhtdp \box\adl@box \hfil}
\let\@endpbox\@@endpbox
\def\adl@startmbox{\bgroup $\vcenter\@startpbox}
\def\adl@endmbox{\adl@org@endpbox $\egroup \adl@colhtdp \box\adl@box \hfil}

%%^L
%    \end{macrocode}
%
% 
%
% \subsection{Multi-columns}
% \label{sec:imp-mc}
%
% \begin{macro}{\multicolumn}
% \changes{v1.4-2-1}{1999/06/25}
%	{Modified for several reason.}
% \begin{macro}{\adl@preamble}
% \begin{macro}{\adl@mcaddamp}
% \changes{v1.4-2-1}{1999/06/25}
%	{Introduced for the complaint on multiple columns if with
%	\textsf{array}.}
% \begin{macro}{\adl@activatepbox}
% \changes{v1.4-2-1}{1999/06/25}
%	{Introduced to do nothing if with \textsf{array}.}
%
% The macro |\multicolumn| is modified for the followings.
%
% \begin{itemize}
% \item
% The macros to construct the parts of |\@preamble| for vertical lines,
% |\adl@arrayrule|, |\adl@arraydashrule| and |\adl@argarraydashrule|, have
% to perform operations slightly different from those for main preamble.
% Thus they are |\def|-ined to multi-column version |\adl@mcarrayrule|, etc.
% \changes{v1.3}{1998/10/08}
% 	{\cs{def}-s for \cs{adl@mcarrayrule} etc. are enclosed in a group.}
% These |\def|-initions are enclosed in a group so that they are not
% affected to |array| or |tabular| which may occur in the third argument of
% |\multicolumn|.  In order to make |\@preamble| work well outside of the
% group containing |\@makepream|, |\adl@preamble| is |\global|-ly |\let|-equal
% to |\@preamble| just after |\@makepream| in the group and then
% reverse |\let|-assignment is performed just after the group is closed.
% These global assignment is unnecessary with \textsf{array} because
% |\@preamlbe| is constructed |\global|-ly, but safe.
% 
% Since this grouping nullifies the effect of |\adl@preaminit| called in
% |\@mkpream|, we call |\adl@preaminit| again after the group closing.
%
% \item
% In \textsf{array}, |\@addamp| to make |\@preamble| for |\multicolumn| has
% a different definition from that for main one.  Thus it is |\let|-equal
% to |\adl@mcaddamp| whose definition is switched by |\ifadl@usingarypkg|.
% 
% \item
% If \textsf{array} is in use, |\@preamble| has to be |\xdef|-ed once again
% by |\@addpreamble| with an |\@empty| argument after |\@mkpreamble| to
% expand the contents of |\toks| registers.  This is performed whether or
% not with \textsf{array} because it is safe.
%
% \item
% As done in |\@array|, |\set@typeset@protect| is replaced with direct
% |\let|.
%
% \item
% If without \textsf{array}, |\@startpbox| and |\@endpbox| should be
% |\let|-equal to their |@@| counterparts, while should not with
% \textsf{array}.  Thus we define |\adl@activatepbox| to do or not to do so
% depending on |\ifadl@usingarypkg|.
% 
% \item
% The counter |\adl@currentcolumn| is |\global|-ly incremented by the first
% argument of |\multicolumn| (number of columns to be |\span|-ned).
% \end{itemize}
%
% Note that |\adl@columns| is modified by |\@mkpream|, but it is not
% referred to by |\adl@mcarrayrule| etc., and its value is restored before
% referred to by |\hdashline|, etc.
% \end{macro}\end{macro}\end{macro}\end{macro}
%
% \SpecialIndex{\adl@arrayrule}
% \SpecialIndex{\adl@arraydashrule}
% \SpecialIndex{\adl@argarraydashrule}
% \SpecialIndex{\adl@mcarrayrule}
% \SpecialIndex{\adl@mcarraydashrule}
% \SpecialIndex{\adl@mcargarraydashrule}
% \SpecialIndex{\ifadl@usingarypkg}
% \SpecialIndex{\adl@currentcolumn}
%
%    \begin{macrocode}

%% Multi-Columns

\def\multicolumn#1#2#3{\multispan{#1}\begingroup \begingroup
        \def\adl@arrayrule{\adl@mcarrayrule{#1}}%
        \def\adl@arraydashrule{\adl@mcarraydashrule{#1}}%
        \def\adl@argarraydashrule{\adl@mcargarraydashrule{#1}}%
        \let\@addamp\adl@mcaddamp
        \@mkpream{#2}\@addtopreamble\@empty
        \global\let\adl@preamble\@preamble \endgroup
        \let\@preamble\adl@preamble
        \def\@sharp{#3}\let\protect\relax
        \adl@activatepbox
        \adl@preaminit
        \@arstrut \@preamble\hbox{}\endgroup
        \global\advance\adl@currentcolumn#1\ignorespaces}
\ifadl@usingarypkg
        \def\adl@mcaddamp{\if@firstamp\@firstampfalse \else\@preamerror5\fi}
        \let\adl@activatepbox\relax
\else
        \let\adl@mcaddamp\@addamp
        \def\adl@activatepbox{\let\@startpbox\@@startpbox
                \let\@endpbox\@@endpbox}
\fi

%    \end{macrocode}
%
% \begin{macro}{\adl@mcarrayrule}
% \begin{macro}{\adl@mcarraydashrule}
% \begin{macro}{\adl@mcargarraydashrule}
% \changes{v1.4-2-1}{1999/06/25}
%	{Modified to pretend \texttt{p} or \texttt{@} depending on if
%	\textsf{array} is in use.}
% \SpecialIndex{\adl@arrayrule}
% \SpecialIndex{\adl@arraydashrule}
% \SpecialIndex{\adl@argarraydashrule}
% \SpecialIndex{\adl@currentcolumn}
%
% The preamble parts for vertical lines are constructed by the macros
% |\adl@mcarrayrule|, |\adl@mcarraydashrule| and |\adl@mcargarraydashrule|
% to which the first argument \meta{$n$} of |\multicolumn| is passed to know
% the number of columns to be |\span|-ned.  They are similar to their
% relatives for main preamble, |\adl@arrayrule|, etc., but the arguments
% \meta{$c^L$} and \meta{$c^R$} passed to |\adl@xarraydashrule| are;
% $$
% c^L=c,\qquad c^R=c+n-1
% $$
% where $c={}$|\adl@currentcolumn|.  This makes leading vertical lines drawn
% at the left edge of the leftmost |\span|-ned column and trailing ones at
% the right edge of the rightmost column.
% \end{macro}\end{macro}\end{macro}
%
%    \begin{macrocode}
\def\adl@mcarrayrule#1{\@tempcnta#1\advance\@tempcnta\adl@currentcolumn
        \advance\@tempcnta\m@ne
        \adl@xarraydashrule
                {\adl@currentcolumn}{\@tempcnta}{{\z@/\z@}}}
\def\adl@mcarraydashrule#1{\@tempcnta#1\advance\@tempcnta\adl@currentcolumn
        \advance\@tempcnta\m@ne
        \adl@xarraydashrule
                {\adl@currentcolumn}{\@tempcnta}%
                {{\dashlinedash/\dashlinegap}}}
\def\adl@mcargarraydashrule#1{\@tempcnta#1\advance\@tempcnta\adl@currentcolumn
        \advance\@tempcnta\m@ne
        \adl@xarraydashrule
                {\adl@currentcolumn}{\@tempcnta}{}%
        \@chclass\adl@class@iiiorvii \let\@classv\adl@classvfordash}

%%^L
%    \end{macrocode}
%
% 
%
% \subsection{End of Rows}
% \label{sec:imp-eor}
%
% \begin{macro}{\@xarraycr}
% \changes{v1.4-2-1}{1999/06/25}
%	{The version for \textsf{array} is introduced.}
% \begin{macro}{\@xtabularcr}
% \begin{macro}{\@xargarraycr}
% \begin{macro}{\@yargarraycr}
% \SpecialIndex{\ifadl@usingarypkg}
% 
% At the end of the $i\th$ row, we have to calculate $h_i$ which is the height
% plus depth of the row, and add elements \meta{$C^L_i,h_i$} and
% \meta{$C^R_i,h_i$} to $R^L$ and $R^R$.  To do this, |\cr|-s in the macros
% \hbox{|\@xarraycr|}, |\@xtabularcr|, |\@xargarraycr| are replaced with our
% own |\adl@cr|.  The macro |\@yargarraycr|\meta{dimen} is also modified but
% its |\cr| is replaced with |\adl@argcr|\meta{dimen} to add (negative)
% |\dimen| to $h_i$.  Note that |\@xargarraycr|\meta{dimen} uses ordinary
% |\adl@cr| because the extra vertical space of \meta{dimen} is inserted to
% the last column.
% 
% Note that the implementation of |\@xarraycr| is slightly different between
% \LaTeX{} and \textsf{array}, we have to have two versions and choose one.
% \end{macro}\end{macro}\end{macro}\end{macro}
%
%    \begin{macrocode}

%% End of row

\ifadl@usingarypkg
\def\@xarraycr{\@ifnextchar[{\@argarraycr}{\ifnum0=`{}\fi\adl@cr}}
\else
\def\@xarraycr{\@ifnextchar[{\@argarraycr}{\ifnum0=`{\fi}${}\adl@cr}}
\fi
\def\@xtabularcr{\@ifnextchar[{\@argtabularcr}{\ifnum0=`{\fi}\adl@cr}}
\def\@xargarraycr#1{\@tempdima#1\advance\@tempdima\dp\@arstrutbox
        \vrule\@height\z@\@depth\@tempdima\@width\z@
        \adl@cr}
\def\@yargarraycr#1{\adl@argcr{#1}\noalign{\vskip #1}}

%    \end{macrocode}
%
% \begin{macro}{\adl@cr}
% \begin{macro}{\adl@argcr}
% The macro |\adl@cr| and |\adl@argcr| perform |\cr| and then invoke the
% common macro |\adl@@cr|\meta{x}.  The argument \meta{x} is the extra
% (negative) vertical space for |\adl@argcr|, while it is 0 for |\adl@cr|.
% \end{macro}\end{macro}
%
% \begin{macro}{\adl@@cr}
% \changes{v1.4-2-6}{1999/06/25}
%	{Modified to set \cs{adl@finaldepth}.}
% \changes{v1.6}{2003/8/25}
%	{Modified to insert \cs{adl@discard}.}
% \SpecialIndex{\@arstrutbox}
% \SpecialIndex{\adl@totalheight}
% \SpecialIndex{\adl@colsL}
% \SpecialIndex{\adl@colsR}
% \SpecialIndex{\adl@rowsL}
% \SpecialIndex{\adl@rowsR}
% \SpecialIndex{\adl@connect}
% \SpecialIndex{\adl@discard}
% \SpecialIndex{\adl@finaldepth}
% \SpecialIndex{\adl@height}
% \SpecialIndex{\adl@depth}
% \SpecialIndex{\adl@currentcolumn}
%
% The macro |\adl@@cr|\meta{x} at first calculate $h_i$ as follows.  The
% registers |\adl@height|${}=\eta$ and |\adl@depth|${}=\delta$ have the
% maximum height and depth of the columns in the row.  However, they could
% be smaller than the height and/or depth of |\@arstrutbox|, $\eta_s$ and
% $\delta_s$.  If so, the height and/or depth of the row are $\eta_s$ and
% $\delta_s$.  Therefore, $h_i$ is caluclated by;
% $$
% h_i = \max(\eta,\eta_s) + \max(\delta,\delta_s).
% $$
% Additionally, if the extra space \meta{x} is negative, a vertical space of
% $x$ is inserted below the row\footnote{
%
% Before v1.54, negative \meta{x} shrinks the hight of the row by $\vert
% x\vert$.  Although the former result may be more appropriate if the row
% has vertical lines than the current because lines extrude to the next row
% now, new feature is considered compatible with original
% \texttt{array}\slash \texttt{tabular}.}.
%
% Thus the integer value of $h_i+x$ is
% |\global|-ly added to |\adl@totalheight|, and the elements
% \meta{$C^L_i{=}\cs{adl@colsL},h_i$} and
% \meta{$C^R_i{=}\cs{adl@colsR},h_i$} are added to the tail of
% $R^L={}$|\adl@rowsL| and $R^C={}$|\adl@rowsR|\@.  If $x$ is not 0
% (negative), {\it discard}$(x)$ or {\it connect}$(x)$ is also added after
% \meta{$C^{L/R}_i,h_i$} according to the current environment (|longtable|
% or not).  In the real implementation, $R^L$ and $R^C$ has the following
% format of \meta{rows}.
% 
%
% \begin{eqnarray*}
% \meta{rows}&::=&
%	[\meta{row}\hbox{\tt;}]^*\\
% \meta{row}&::=&
%	\hbox{{\tt(}\meta{cols}{\tt/}\meta{$h_i$}{\tt)}}\\
% \meta{cols}&::=&
%	[\hbox{\cs{elt}\Meta{c}\Meta{d}\Meta{g}}]^*\ \vert\qquad\qquad&
%	    \hbox to.3\textwidth{\ldots\ $C^L$ or $C^R$\hss}\cr
% &&	\cs{adl@connect}\ \vert&
%	    \hbox to.3\textwidth{\ldots\ for {\it connect}$(h_i)$\hss}\cr
% &&	\cs{adl@discard}\ \vert&
%	    \hbox to.3\textwidth{\ldots\ for {\it discard}$(h_i)$\hss}\cr
% &&	\cs{relax}&
%	    \hbox to.3\textwidth{\ldots\ for {\it disconnect}$(h_i)$\hss}\cr
% \noalign{\vskip-\baselineskip}
% \end{eqnarray*}
%
% Since |\adl@discard| is |\def|-ined as |\adl@connect| by |\adl@arrayinit|, 
% added |\adl@discard| transforms itself into |\adl@connect| if current
% envrionment is not |longtable|.  Otherwise, as we make |\adl@discard|
% |\let|-equal to |\relax| when a |longtable| environment starts, it keeps
% its own form.
% 
% Then, |\adl@finaldepth| is set to |\adl@depth| if $x$ is zero, or to zero
% otherwise (negative), in order to make the depth of |array|\slash|tabular|
% equal to that of the last row.  Finally, |\adl@colsL|, |\adl@colsR|,
% |\adl@currentcolumn|, |\adl@height| and |\adl@depth| are reinitialized to
% process the next row.
% \end{macro}
%
%    \begin{macrocode}
\def\adl@cr{\cr\noalign{\adl@@cr\z@}}
\def\adl@argcr#1{\cr\noalign{\adl@@cr{#1}}}
\def\adl@@cr#1{
        \ifdim\adl@height<\ht\@arstrutbox \adl@height\ht\@arstrutbox\fi
        \ifdim\adl@depth<\dp\@arstrutbox \adl@depth\dp\@arstrutbox\fi
        \advance\adl@height\adl@depth
        \global\advance\adl@totalheight\adl@height
        \@tempdima#1\relax \global\advance\adl@totalheight\@tempdima
        \xdef\adl@rowsL{\adl@rowsL
                (\adl@colsL/\number\adl@height);%
                \ifdim#1=\z@\else (\adl@discard/\number\@tempdima);\fi}%
        \xdef\adl@rowsR{\adl@rowsR
                (\adl@colsR/\number\adl@height);%
                \ifdim#1=\z@\else (\adl@discard/\number\@tempdima);\fi}%
        \gdef\adl@colsL{}\gdef\adl@colsR{}
        \global\adl@currentcolumn\@ne
        \ifdim#1=\z@ \global\adl@finaldepth\adl@depth
        \else        \global\adl@finaldepth\z@\fi
        \global\adl@height\z@ \global\adl@depth\z@}

%%^L
%    \end{macrocode}
%
% 
%
% \subsection{Horizontal Lines}
% \label{sec:imp-hline}
%
% \begin{macro}{\hline}
% \changes{v1.4-2-6}{1999/06/25}
%	{Modified to set \cs{adl@finaldepth} to zero.}
% \changes{v1.7-1}{2004/5/21}
% 	{Modified to shift up if null-wide.}
% \begin{macro}{\cline}
% \changes{v1.4-2-6}{1999/06/25}
%	{Modified to set \cs{adl@finaldepth} to zero.}
% \changes{v1.7-1}{2004/5/21}
% 	{Modified to shift up if null-wide.}
% \SpecialIndex{\adl@hline}
% \SpecialIndex{\adl@xhline}
% \SpecialIndex{\adl@connect}
% \SpecialIndex{\adl@finaldepth}
% \SpecialIndex{\adl@org@cline}
%
% The macro |\hline| is modified to insert |\vskip-\arrayrulewidth| before
% drawing if |\ADLnullwidehline| is in effect, or to add the element ${\it
% connect}(w)={}$|(\adl@connect/|\allowbreak|\number\arrayrulewidth)| to the
% end of $R^L$ and $R^R$ by |\adl@hline| otherwise.  The other modifications 
% are to set |\adl@finaldepth| to zero for the case that the last vertical
% item is |\hline|, and to check if it is followed by not only |\hline| but
% also |\hdashline| by |\adl@xhline|.
% 
% The macro |\cline| is also modified to set |\adl@finaldepth| to zero.  As
% for the feature of |\ADLnullwidehline|, it inserts |\vskip-\arrayrulewidth| 
% to shift the line up befor drawing, and |\vskip\arrayrulewidth| after
% drawing to cancel the negative skip inserted by |\adl@org@cline|.
% \end{macro}\end{macro}
%
%    \begin{macrocode}

%% Horizontal Lines

\def\hline{\noalign{\ifnum0=`}\fi
        \ifadl@zwhrule \vskip-\arrayrulewidth
        \else \adl@hline\adl@connect\arrayrulewidth \fi
        \hrule\@height\arrayrulewidth
        \global\adl@finaldepth\z@
        \futurelet\@tempa\adl@xhline}
\def\cline#1{\noalign{\global\adl@finaldepth\z@
                \ifadl@zwhrule \vskip-\arrayrulewidth\fi}
        \adl@org@cline{#1}%
        \noalign{\ifadl@zwhrule \vskip\arrayrulewidth\fi}}

%    \end{macrocode}
%
% \begin{macro}{\hdashline}
% \changes{v1.4-2-5}{1999/06/25}
%	{Modified to make \cs{adl@hdashline} usable for
%	\cs{first}\slash\texttt{lasthdashline}.}
% \begin{macro}{\adl@hdashline}
% \changes{v1.4-2-5}{1999/06/25}
%	{Modified to be usable for
%	\cs{first}\slash\texttt{lasthdashline}.}
% \changes{v1.7-1}{2004/5/21}
% 	{Modified for null-wide horizontal lines.}
% \begin{macro}{\adl@ihdashline}
% \changes{v1.4-2-5}{1999/06/25}
%	{Introduced as the substitute of old \cs{adl@hdashline}.}
% \changes{v1.7-1}{2004/5/21}
% 	{\cs{adl@hline} is moved to \cs{adl@hdashline} for null-wide lines.}
% \SpecialIndex{\adl@hcline}
% \SpecialIndex{\dashlinedash}
% \SpecialIndex{\dashlinegap}
%
% The macro |\hdashline| calls |\adl@hdashline| to open the |\noalign|
% construct by the well-known trick |{\ifnum0=`}\fi| and then to invoke
% |\adl@ihdashline| checking the existence of its optional argument
% |[|\meta{dash}|/|\meta{gap}|]|.  Before the invocation, it inserts
% |\vskip-\arrayrulewidth| if |\ADLnullwidehline| is in effect, or adds
% ${\it connect}(w)$ to the end of $R^L$ and $R^R$.  Then |\adl@ihdashline|
% closes the |\noalign| by |\ifnum0=`{\fi}| to start the pseudo row for the
% horizontal dash-line.  Before the dash-line is drawn by |\adl@hcline|
% which is also used for |\cdashline|, all the columns are |\span|-ned by
% giving |\adl@columns| to |\multispan|.  Finally, the |\noalign| is opened
% again and |\adl@xhline| is invoked to check whether |\h(dash)line| is
% followed.
% \end{macro}\end{macro}\end{macro}
% 
% \begin{macro}{\adl@inactivehdl}
% \changes{v1.4-2-3}{1999/06/25}
%	{Introduced to emulate \cs{hdashline} by \cs{hline}.}
% \changes{v1.7-1}{2004/5/21}
%	{Modified to shift up if null-wide.}
%
% If |\ADLinactivate| is in effect, |\adl@ihdashline| is |\let|-equal
% to |\adl@inactivehdl|.  This macro simply puts a |\hrule| discarding its
% arguments after inserting |\vskip|\allowbreak|-\arrayrulewidth| if
% |\ADLnullwidehline| is in effect.
% \end{macro}
%
%    \begin{macrocode}
\def\hdashline{\adl@hdashline\adl@ihdashline}
\def\adl@hdashline#1{\noalign{\ifnum0=`}\fi
        \ifadl@zwhrule \vskip-\arrayrulewidth
        \else \adl@hline\adl@connect\arrayrulewidth \fi
        \@ifnextchar[%]
                     {#1}%
                     {#1[\dashlinedash/\dashlinegap]}}
\def\adl@ihdashline[#1/#2]{\ifnum0=`{\fi}%
        \multispan{\adl@columns}\unskip \adl@hcline\z@[#1/#2]%
        \noalign{\ifnum0=`}\fi
        \futurelet\@tempa\adl@xhline}
\def\adl@inactivehdl[#1/#2]{\ifadl@zwhrule \vskip-\arrayrulewidth \fi
        \hrule\@height\arrayrulewidth
        \futurelet\@tempa\adl@xhline}

%    \end{macrocode}
%
% \begin{macro}{\adl@xhline}
% \SpecialIndex{\@xhline}
%
% The macro |\adl@xhline| is the counterpart of the original |\@xhline|.
% This is introduced to check the mixed sequence of |\hline| and
% |\hdashline|, and to add the element ${\it
% disconnect}(s)={}$|(\relax/\doublerulesep)| to the end of $R^L$ and $R^R$
% by |\adl@hline| if a pair of |\h(dash)line| is found.
% \end{macro}
%
%    \begin{macrocode}
\def\adl@xhline{\ifx\@tempa\hline \adl@ixhline\fi
        \ifx\@tempa\hdashline \adl@ixhline\fi
        \ifnum0=`{\fi}}
\def\adl@ixhline{\vskip\doublerulesep \adl@hline\relax\doublerulesep}

%    \end{macrocode}
%
% \begin{macro}{\adl@hline}
% \SpecialIndex{\adl@totalheight}
% \SpecialIndex{\adl@rowsL}
% \SpecialIndex{\adl@rowsR}
%
% The macro |\adl@hline|\meta{cs}\meta{dimen} |\global|-ly adds the integer
% value of \meta{dimen} to |\adl@totalheight| and adds the element
% |(|\meta{cs}|/\number|\meta{dimen}|)| to the tail of $R^L$ and $R^R$.  The
% arguments \meta{cs}\meta{dimen} are |\adl@connect\arrayrulewidth| for
% ${\it connect}(w)$ or |\relax\doublerulesep| for ${\it disconnect}(s)$.
% \end{macro}
%
%    \begin{macrocode}
\def\adl@hline#1#2{\@tempcnta#2
        \global\advance\adl@totalheight\@tempcnta
        \xdef\adl@rowsL{\adl@rowsL
                (#1/\number\@tempcnta);}%
        \xdef\adl@rowsR{\adl@rowsR
                (#1/\number\@tempcnta);}}

%    \end{macrocode}
%
% \begin{macro}{\cdashline}
% \changes{v1.4-2-7}{1999/06/25}
%	{Modified to call renamed \cs{adl@cdline}.}
% \begin{macro}{\adl@cdline}
% \changes{v1.4-2-7}{1999/06/25}
% 	{Renamed and modified to call renamed \cs{adl@cdlinea}\slash
%	\texttt{b}.} 
% \changes{v1.7-1}{2004/5/21}
% 	{Modified to shift up if null-wide.}
% \begin{macro}{\adl@cdlinea}
% \changes{v1.4-2-7}{1999/06/25}
% 	{Renamed.}
% \begin{macro}{\adl@cdlineb}
% \changes{v1.4-2-7}{1999/06/25}
% 	{Renamed.}
% \SpecialIndex{\@cline}
% \SpecialIndex{\adl@cla}
% \SpecialIndex{\adl@clb}
% \SpecialIndex{\adl@org@cline}
% \SpecialIndex{\adl@hcline}
%
% The macro |\cdashline| at first opens |\noalign| and then invokes
% |\adl@cdline| checking the existence of its optional argument
% |[|\meta{dash}|/|\meta{gap}|]|.  The macro |\adl@cdline| first inserts
% |\vskip-\arrayrulewidth| if |\ADLnullwidehline| is in effect.  Then it
% performs column |\span|-ing by the code based on that of |\@cline| in
% \LaTeX-2.09 because \LaTeXe's version will not work with \LaTeX-2.09.  The
% main job is done by |\adl@hcline| after the target columns are |\span|-ned
% by |\adl@cdlinea| or |\adl@cdlineb|.
% \end{macro}\end{macro}\end{macro}\end{macro}
%
% \begin{macro}{\adl@inactivecdl}
% \changes{v1.4-2-3}{1999/06/25}
%	{Introduced to emulate \cs{cdashline} by \cs{cline}.}
% \changes{v1.7-1}{2004/5/21}
% 	{Modified to invoke \cs{cline} rather than \cs{adl@orgcline} for
%	 null-wide.}
%
% If |\ADLinactivate| is in effect, |\adl@cdline| is |\let|-equal
% to |\adl@inactivecdl|.  This macro simply calls our own |\cline|, after
% closing the |\noalign| opened by |\cdashline|.
% \end{macro}
%
%    \begin{macrocode}
\def\cdashline#1{\noalign{\ifnum0=`}\fi
        \@ifnextchar[%]
                     {\adl@cdline[#1]}%
                     {\adl@cdline[#1][\dashlinedash/\dashlinegap]}}
\def\adl@cdline[#1-#2]{\ifadl@zwhrule \vskip-\arrayrulewidth \fi
        \global\adl@cla#1\relax
        \global\advance\adl@cla\m@ne
        \ifnum\adl@cla>\z@ \global\let\@gtempa\adl@cdlinea
        \else              \global\let\@gtempa\adl@cdlineb\fi
        \global\adl@clb#2\relax
        \global\advance\adl@clb-\adl@cla \ifnum0=`{\fi}
        \@gtempa{-\arrayrulewidth}}
\def\adl@cdlinea{\multispan\adl@cla &\multispan\adl@clb \unskip \adl@hcline}
\def\adl@cdlineb{\multispan\adl@clb \unskip \adl@hcline}

\def\adl@inactivecdl[#1-#2][#3]{\ifnum0=`{\fi}\cline{#1-#2}}

%    \end{macrocode}
%
% \begin{macro}{\adl@hcline}
% \changes{v1.4-2-4}{1999/06/25}
%	{Modified to use \cs{adl@draw}.}
% \changes{v1.54-3}{2003/08/25}
%	{Omit \cs{vskip} if the space is 0.}
% \changes{v1.7-1}{2004/5/21}
%	{Modified not to shift null-wide \cs{cdashline} down.}
% \SpecialIndex{\adl@draw}
% \SpecialIndex{\adl@vrule}
% \SpecialIndex{\adl@finaldepth}
%
% The macro |\adl@hcline|\meta{w}|[|\meta{d}|/|\meta{g}|]| draws a
% horizontal dash-line of dash size $d$ and gap size $g$ for |\hdashline|
% and |\cdashline| in the |\span|-ned columns by |\adl@draw|.  As we will
% discuss in \S\ref{sec:imp-draw}, the macro requires $d$ and $g$ are
% passed through |\@tempdima| and \hbox{|\@tempdimb|}, and control sequences
% \meta{rule}, \meta{skip} and \meta{box} are passed through its arguments
% to make it usable for both horizontal and vertical lines.  Then the
% vertical space of $w$, $-$|\arrayrulewidth| for |\cdashline|, is
% inserted if it is not 0 (for |\hdashline|) and |\ADLnullwidehline| is not
% in effect.
% \end{macro}
%
%    \begin{macrocode}
\def\adl@hcline#1[#2/#3]{\@tempdima#2\relax \@tempdimb#3\relax
        \adl@draw\adl@vrule\hskip\hbox \cr
        \noalign{\global\adl@finaldepth\z@ \ifdim#1=\z@\else
                \ifadl@zwhrule\else \vskip#1\fi\fi}}

%    \end{macrocode}
%
% \begin{macro}{\firsthdashline}
% \changes{v1.4-2-5}{1999/06/25}
%	{Introduced as the dashed version of \cs{firsthline}.}
% \begin{macro}{\lasthdashline}
% \changes{v1.4-2-5}{1999/06/25}
%	{Introduced as the dashed version of \cs{lasthline}.}
% \SpecialIndex{\firsthline}
% \SpecialIndex{\lasthline}
% \SpecialIndex{\adl@hdashline}
% \SpecialIndex{\adl@ihdashline}
%
% If \textsf{array} is in use, we wish to have dashed counterparts of
% |\first|\slash|lasthline| named |\first|\slash|lasthdashline|, which
% simply call |\adl@hdashline| with an argument to call
% |\adl@first|\slash|lasthdashline| after closing |\noalign| opened by
% |\adl@hdashline|.
% \end{macro}\end{macro}
%
% \begin{macro}{\adl@defflhdl}
% \changes{v1.4-2-5}{1999/06/25}
%	{Introduced for the tricky definition of \cs{adl@first}\slash
%	\texttt{lasthdashline}.}
% \begin{macro}{\adl@idefflhdl}
% \changes{v1.4-2-5}{1999/06/25}
%	{Introduced for the tricky definition of \cs{adl@first}\slash
%	\texttt{lasthdashline}.}
% \begin{macro}{\adl@firsthdashline}
% \changes{v1.4-2-5}{1999/06/25}
%	{Introduced as the body of \cs{firsthdashline}.}
% \begin{macro}{\adl@lasthdashline}
% \changes{v1.4-2-5}{1999/06/25}
%	{Introduced as the body of \cs{lasthdashline}.}
% The macros |\adl@first|\slash|lasthdashline|, however, are defined in a
% tricky manner to replace |\hline| in |\first|\slash|lasthline| with;
%
% \begin{quote}
% |\adl@hdashline\adl@ihdashline[|\meta{dash}|/|\meta{gap}|]|
% \end{quote}
%
% in order to avoid copy-and-replace.  To do that, we define |\adl@defflhdl| 
% and |\adl@idefflhdl| in which the body of |\first|\slash|lasthline| is
% expanded by |\exapndafter| and the parts preceding and following |\hline|
% are extracted.  Then the preceding part \meta{p}, the calling sequence of
% |\adl@hdashline|, and the following part \meta{f} are connected to be the
% body of |\adl@first|\slash|lasthdashline|.  Thus we define
% |\adl@firsthdashline| as follows.
%
% \begin{quote}
% |\def\adl@firsthdashline|\texttt{[\#1/\#2]}|{%|\\
% \mbox{\qquad}\meta{p}\\
% \mbox{\qquad}|\adl@hdashline\adl@ihdashline|\texttt{[\#1/\#2]}\\
% \mbox{\qquad}\meta{f}|}|
% \end{quote}
% \end{macro}\end{macro}\end{macro}\end{macro}
%
%    \begin{macrocode}
\ifadl@usingarypkg
\def\firsthdashline{\adl@hdashline{\ifnum0=`{\fi}\adl@firsthdashline}}
\def\lasthdashline{\adl@hdashline{\ifnum0=`{\fi}\adl@lasthdashline}}

\def\adl@defflhdl#1{\def\@tempa{#1}
        \expandafter\adl@idefflhdl}
\def\adl@idefflhdl#1\hline#2\@nil{%
        \@namedef\@tempa[##1/##2]{#1\adl@hdashline\adl@ihdashline[##1/##2]#2}}
\adl@defflhdl{adl@firsthdashline}\firsthline\@nil
\adl@defflhdl{adl@lasthdashline}\lasthline\@nil
\fi

%%^L
%    \end{macrocode}
% 
%
% \subsection{End of Environment}
% \label{sec:imp-eoe}
%
% \begin{macro}{\endarray}
% \begin{macro}{\endtabular}
% \changes{v1.7}{2004/5/21}
%	{Modified to refer proper \cs{endarray} depending on the
%	exsistance of \textsf{colortbl}.}
% \begin{macro}{\endtabular*}
% \SpecialIndex{\adl@arrayrestore}
%
% The macros to close the |array|/|tabular| environment, |\endarray| and
% |\endtabular(*)|, are modified so that they invoke |\adl@endarray| to draw
% vertical lines just before closing |\halign|, and |\adl@arrayrestore| to
% restore registers and data structures |\global|-ly modified in the
% environment.
% \end{macro}\end{macro}\end{macro}
%
%    \begin{macrocode}

%% End of Environment

\def\endarray{\adl@endarray \egroup \adl@arrayrestore \egroup}
\def\endtabular{\endarray $\egroup}
\expandafter\let\csname endtabular*\endcsname\endtabular

%    \end{macrocode}
%
% \begin{macro}{\adl@endarray}
% \changes{v1.4-2-6}{1999/06/25}
%	{Modified to set the depth of \texttt{array}\slash\texttt{tabular}
%	to \cs{adl@finaldepth}.}
% \begin{macro}{\adl@rows}
% \begin{macro}{\adl@addvl}
% \begin{macro}{\adl@vlrowL}
% \begin{macro}{\adl@vlrowR}
% \begin{macro}{\adl@vlrow}
% \SpecialIndex{\adl@height}
% \SpecialIndex{\adl@depth}
% \SpecialIndex{\adl@@cr}
% \SpecialIndex{\adl@currentcolumn}
% \SpecialIndex{\adl@rowsL}
% \SpecialIndex{\adl@rowsR}
% \SpecialIndex{\adl@makevlrL}
% \SpecialIndex{\adl@makevlrR}
% \SpecialIndex{\adl@drawvl}
% \SpecialIndex{\adl@finaldepth}
%
% The macro |\adl@endarray| at first closes the last row by |\crcr|.  If this
% |\crcr| has real effect, we have to invoke |\adl@@cr| to perfrom our own
% end-of-row operations.  We assume that the |\crcr| is effective if either
% |\adl@height| or |\adl@depth| has a non-zero value\footnote{
%
% The author confesses that this rule is not strict and the introduction of a
% switch could improve the strictness.}.
%
% Then the rows to draw vertical lines $L_1$, \ldots, $L_n$;
% $$
% \sigma_1 L_1 \sigma_2 L_2 \ldots L_{n-1}\sigma_n L_n\sigma_{n+1}
% $$
% are created in |\adl@vlrowL| and |\adl@vlrowR| by |\adl@makevlrL| and
% |\adl@makevlrR|.  In the real implementation,
% $L_k=\meta{$\gamma_k,\pi_k,\delta_k,\xi_k,\tau_k,\beta_k$}$ is represented
% as;
%
% \begin{quote}
% |\adl@vl{|$\beta_k$|}{|$\tau_k-\beta_k$|}{|$\delta_k$|}{|$\xi_k$|}|.
% \end{quote}
%
% Thus |\adl@vl| is made |\let|-equal to |\relax| when the rows are
% constructed and to \hbox{|\adl@@vl|} when the rows are put.
%
% Since |\adl@makevlrL| and |\adl@makevlrR| shares common macros, they
% conceptually have the following interface.
%
% \begin{quote}
% \begin{tabbing}
% \cs{adl@vlrow}${}={}$\cs{adl@makevlrL/R}(\=
%	\cs{adl@rows}\,:\,\meta{$R^L$ or $R^R$},\\
% \>	\cs{adl@currentcolumn}\,:\,\meta{start column},\\
% \>	\cs{adl@addvl}\,:\,\meta{macro to add an element})
% \end{tabbing}
% \end{quote}
%
% Thus they are invoked as;
%
% \begin{quote}\leavevmode
% \cs{adl@vlrowL}${}={}$\cs{adl@makevrL}(
%	\cs{adl@rowsL},
% 	1,
% 	\cs{adl@addvlL})\\
% \cs{adl@vlrowR}${}={}$\cs{adl@makevrR}(
%	\cs{adl@rowsR},
% 	\cs{adl@columns},
% 	\cs{adl@addvlR})
% \end{quote}
%
% \changes{v1.54-1}{2003/08/25}
%	{Rows for vertical lines are replaced by \cs{adl@drawvl}.}
%
% Finally, after constructed rows for vertical lines are put by
% |\adl@drawvl|, a vertical skip of $-{}$|\adl@finaldepth| is inserted to
% move back to the last baseline, and then an invisible |\vrule| of
% |\adl@finaldepth| deep is put to make |array|\slash|tabular| has the depth
% of the last real row or zero if it ends with a horizontal line.
% \end{macro}\end{macro}\end{macro}\end{macro}\end{macro}\end{macro}
%
%    \begin{macrocode}
\def\adl@endarray{\crcr \noalign{
        \ifdim\adl@height=\z@
        \ifdim\adl@depth=\z@  \else \adl@@cr\z@ \fi
                              \else \adl@@cr\z@ \fi
        \let\adl@vl\relax
        \def\adl@vlrow{}\adl@currentcolumn\@ne
                \let\adl@rows\adl@rowsL
                \let\adl@addvl\adl@addvlL
                \adl@makevlrL \global\let\adl@vlrowL\adl@vlrow
        \def\adl@vlrow{}\adl@currentcolumn\adl@columns
                \let\adl@rows\adl@rowsR
                \let\adl@addvl\adl@addvlR
                \adl@makevlrR \global\let\adl@vlrowR\adl@vlrow
        \global\let\adl@vl\adl@@vl}%
        \adl@drawvl
        \noalign{\vskip-\adl@finaldepth}%
        \omit\vrule\@width\z@\@height\z@\@depth\adl@finaldepth\cr}

%    \end{macrocode}
%
% \begin{macro}{\adl@arrayrestore}
% \SpecialIndex{\adl@height}
% \SpecialIndex{\adl@depth}
% \SpecialIndex{\adl@currentcolumn}
% \SpecialIndex{\adl@totalheight}
% \SpecialIndex{\adl@heightsave}
% \SpecialIndex{\adl@depthsave}
% \SpecialIndex{\adl@currentcolumnsave}
% \SpecialIndex{\adl@totalheightsave}
% \SpecialIndex{\adl@rowsL}
% \SpecialIndex{\adl@rowsR}
% \SpecialIndex{\adl@rowsLsave}
% \SpecialIndex{\adl@rowsRsave}
% \SpecialIndex{\adl@colsL}
% \SpecialIndex{\adl@colsR}
% \SpecialIndex{\adl@colsLsave}
% \SpecialIndex{\adl@colsRsave}
%
% The macro |\adl@arrayrestore| restores the values of registers and data
% structures, |\adl@height|, |\adl@depth|, |\adl@currentcolumn|,
% |\adl@totalheight|, |\adl@rowsL|, |\adl@rowsR|, |\adl@colsL| and
% |\adl@colsR|, saved by |\adl@arrayinit|.
% \end{macro}
%
%    \begin{macrocode}
\def\adl@arrayrestore{%
        \global\adl@height\adl@heightsave
        \global\adl@depth\adl@depthsave
        \global\adl@currentcolumn\adl@currentcolumnsave
        \global\adl@totalheight\adl@totalheightsave
        \global\let\adl@rowsL\adl@rowsLsave
        \global\let\adl@rowsR\adl@rowsRsave
        \global\let\adl@colsL\adl@colsLsave
        \global\let\adl@colsR\adl@colsRsave}

%%^L
%    \end{macrocode}
%
% 
%
% \def\defprog{
% \def\WHILE{\hbox{\bf while}}
% \def\DO{\hbox{\bf do}}
% \def\BEGIN{\hbox{\bf begin}}
% \def\END{\hbox{\bf end}}
% \def\IF{\hbox{\bf if}}
% \def\THEN{\hbox{\bf then}}
% \def\ELSE{\hbox{\bf else}}
% \def\TRUE{\hbox{\bf true}}
% \def\FALSE{\hbox{\bf false}}
% \let\_\gets
% \def\GAMMA{{\mit\Gamma}}
% \def\LAMBDA{{\mit\Lambda}}
% \def\!{\stepcounter{enumi}\llap{\footnotesize(\arabic{enumi})\quad}}
% }
% \begin{figure}[p]\defprog
% \begin{itemize}\item[]
% \let\B\bot
% \setcounter{enumi}{0}
% \begin{tabbing}
%\!$\LAMBDA\_\meta{}$; $R\_R^L$; $\gamma\_1$;\\
%\!\WHILE\ $\gamma\leq\GAMMA$ \DO\ \BEGIN\\
%\!\qquad\=
%	$\tau\_H$; $\beta\_H$; $\eta\_0$; $\delta\_ \meta{$-1,\B$}$;
%			       $\xi\_ \meta{$-1,\B$}$;\\
%\!\>	${\it conn}\_\FALSE$; ${\it double}\_\FALSE;$; $R'\_\meta{}$\\
%\!\>	\WHILE\ $R\neq\meta{}$ \DO\ \BEGIN\\
%\!\>\qquad\=	$\meta{$r,R$}\_R$;\\
%\!\>	\>	$\meta{$C,h$}\_r$;\\
%\!\>	\>	\IF\ $C=\meta{}$ \THEN\ \BEGIN\quad
%	${\it add}(\tau,\beta,\delta,\xi)$; $\eta\_0$;\quad\END;\\
%\!\>	\>	\ELSE\IF\ $C=\meta{connect}$ \THEN\quad $\eta\_h$;\\
%\!\>	\>	\ELSE\ \BEGIN\\
%\!\>	\>\qquad\=	$\meta{$e,C'$}=C$; $\meta{$c,d,g$}=e$;\\
%\!\>	\>	\>	\IF\ $c=\gamma$ \THEN\ \BEGIN\\
%\!\>	\>	\>\qquad\=	\IF\ $d=\delta\LAND g=\xi$ \THEN\ \BEGIN\\
%\!\>	\>	\>	\>\qquad\=	\IF\ $\lnot{\it conn}$ \THEN\ \BEGIN\\
%\!\>	\>	\>	\>	\>\qquad\=	$\tau\_\beta+\eta$;
%						${\it conn}\_\TRUE$;\\
%\!\>	\>	\>	\>	\>	\END;\\
%\!\>	\>	\>	\>	\END;\\
%\!\>	\>	\>	\>	\ELSE\ \BEGIN\\
%\!\>	\>	\>	\>	\>	${\it add}(\tau,\beta,\delta,\xi)$;\\
%\!\>	\>	\>	\>	\>	$\delta\_d$; $\xi\_g$;
%					$\tau\_\beta+\eta$; 
%					${\it conn}\_\TRUE$;\\
%\!\>	\>	\>	\>	\END;\\
%\!\>	\>	\>	\>	\IF\ $C'=\meta{\meta{$\gamma,?,?$},$?$}$ \THEN
% 					\quad ${\it double}\_\TRUE$;\\
%\!\>	\>	\>	\>	$C\_C'$;\\
%\!\>	\>	\>	\END;\\
%\!\>	\>	\>	\ELSE\quad ${\it add}(\tau,\beta,\delta,\xi)$;\\
%\!\>	\>	\>	$\eta\_0$;\\
%\!\>	\>	\END;\\
%\!\>	\>	$\beta\_\beta-h$; $R'\_\meta{$R',\meta{$C,h$}$}$\\
%\!\>	\END;\\
%\!\>	${\it add}(\tau,\beta,\delta,\xi)$; $R\_R'$;\\
%\!\>	\IF\ {\it double} \THEN\quad
%		$\LAMBDA\_\meta{$\LAMBDA,\cs{hskip}\cs{doublerulesep}$}$;\\
%\!\>	\ELSE\ \BEGIN\\
%\!\>	\>	$\gamma\_\gamma+1$;\\
%\!\>	\>	\IF\ $\gamma>\GAMMA$ \THEN\quad\=
%			$\LAMBDA\_\meta{$\LAMBDA,\cs{hfil}$}$;\\
%\!\>	\>	\ELSE\>
%			$\LAMBDA\_
%			    \meta{$\LAMBDA,\cs{hfil}\hbox{\tt\&}\cs{omit}$}$;\\
%\!\> 	\END;\\
%\!\END;\\
%\!\\
%\!{\bf procedure} ${\it add}(\tau,\beta,\delta,\xi)$ \BEGIN\\
%\!\>	\IF\ {\it conn} \THEN\ \BEGIN\\
%\!\>	\>	$\LAMBDA\_
%		    \meta{$\LAMBDA,\meta{$\beta,\tau-\beta,\delta,\xi$}$}$;
%		${\it conn}\_\FALSE$;\\
%\!\>	\END;\\
%\!\END;
% \end{tabbing}
% \end{itemize}
% \caption{Conceptual Code of \cs{adl@makevlrL}}
% \label{fig-makevlr}
% \end{figure}
%
% 
%
% \subsection{Drawing Vertical Lines}
% \label{sec:imp-vl}
%
% \SpecialIndex{\adl@rowsL}
% \SpecialIndex{\adl@rowsR}
% \SpecialIndex{\adl@rows}
% \SpecialIndex{\adl@vlrowL}
% \SpecialIndex{\adl@vlrowR}
% \SpecialIndex{\adl@vlrow}
% \SpecialIndex{\adl@columns}
% \SpecialIndex{\adl@currentcolumn}
% \SpecialIndex{\adl@lastconn}
% \SpecialIndex{\adl@dash}
% \SpecialIndex{\adl@gap}
% \SpecialIndex{\adl@vlrow}
% \SpecialIndex{\ifadl@connected}
% \SpecialIndex{\ifadl@doublerule}
% \SpecialIndex{\adl@totalheight}
%
% \changes{v1.73-1}{2016/04/28}
% 	{Add $\eta\EQ\cs{adl@lastconn}$, its initialization and updates, and
% 	 the addition to $\tau$.}
%
% Figure~\ref{fig-makevlr} shows the conceptual code of |\adl@makevlrL|.
% The correspondance of variables in the code and control sequences in the
% real implementation is as follows.
% \begin{itemize}\item[]
% $\begin{array}[t]{*4{r@{\;{:}\;}l}}
% R^L&			\cs{adl@rowsL}&
% R&			\cs{adl@rows}&
% R'&			\cs{@tempb}&
% \mit\Lambda&		\cs{adl@vlrowL}\\
% \mit\Gamma&		\cs{adl@columns}&
% \gamma&		\rlap{\cs{adl@currentcolumn}}\\
% \tau&			\cs{@tempcnta}&
% \beta&		\cs{@tempcntb}&
% \eta&			\rlap{\cs{adl@lastconn}}\\
% \delta&		\rlap{\cs{adl@dash}/\cs{adl@dashcolor}}&
% 			\multicolumn2c{}&
% \xi&			\rlap{\cs{adl@gap}/\cs{adl@gapcolor}}&\multicolumn2c{}\\
% H&			\rlap{\cs{adl@totalheight}}\\
% {\it conn}&		\rlap{\cs{ifadl@connected}}&\multicolumn2c{}&
% {\it double}&		\rlap{\cs{ifadl@doublerule}}\\
% \end{array}$
% \end{itemize}
%
% \begin{macro}{\adl@makevlrL}
% \changes{v1.4-2-8}{1999/06/25}
%	{Modified to replace \cs{hfil} with \cs{hss} to prevent drawing
%	vertical lines widen columns.}
% \begin{macro}{\adl@makevlrR}
% \changes{v1.4-2-8}{1999/06/25}
%	{Modified to replace \cs{hfil} with \cs{hss} to prevent drawing
%	vertical lines widen columns.}
% The macro |\adl@makevlrL| corresponds to the line (2) and (31)--(36).  Its
% right-edge counterpart |\adl@makevlrR| has the same correspondance but the
% lines (1)--(2) are;
%
% \begin{quote}\defprog\setcounter{enumi}{0}
% \begin{tabbing}
%\!$\LAMBDA\_\meta{}$; $R\_R^R$; $\gamma\_\GAMMA$;\\
%\!\WHILE\ $\gamma>0$ \DO\ \BEGIN
% \end{tabbing}
% \end{quote}
%
% and (31)--(36) are;
%
% \begin{quote}\defprog\setcounter{enumi}{30}
% \begin{tabbing}
%\!\qquad\=
%	\IF\ {\it double} \THEN\quad
%		$\LAMBDA\_\meta{$\cs{hskip}\cs{doublerulesep},\LAMBDA$}$;\\
%\!\>	\ELSE\ \BEGIN\\
%\!\>\qquad\=	$\gamma\_\gamma-1$;\\
%\!\>	\>	\IF\ $\gamma=0$ \THEN\quad\=
%			$\LAMBDA\_\meta{$\cs{hss},\LAMBDA$}$;\\
%\!\>	\>	\ELSE\>
%			$\LAMBDA\_
%			    \meta{$\hbox{\tt\&}\cs{omit}\cs{hss},\LAMBDA$}$;\\
%\!\> 	\END;\\
% \end{tabbing}
% \end{quote}
% \end{macro}\end{macro}
%    \begin{macrocode}

%% Drawing Vertical Lines

\def\adl@makevlrL{\adl@makevlr
        \ifadl@doublerule
                \edef\adl@vlrow{\adl@vlrow \hskip\doublerulesep}%
                \let\next\adl@makevlrL
        \else
                \advance\adl@currentcolumn\@ne
                \ifnum\adl@currentcolumn>\adl@columns \let\next\relax
                        \edef\adl@vlrow{\adl@vlrow \hss}%
                \else \let\next\adl@makevlrL
                        \edef\adl@vlrow{\adl@vlrow \hss &\omit}%
        \fi\fi\next}
\def\adl@makevlrR{\adl@makevlr
        \ifadl@doublerule
                \edef\adl@vlrow{\hskip\doublerulesep \adl@vlrow}%
                \let\next\adl@makevlrR
        \else
                \advance\adl@currentcolumn\m@ne
                \ifnum\adl@currentcolumn=\z@ \let\next\relax
                        \edef\adl@vlrow{\hss \adl@vlrow}%
                \else \let\next\adl@makevlrR
                        \edef\adl@vlrow{&\omit \hss \adl@vlrow}%
        \fi\fi\next}

%    \end{macrocode}
%
% \begin{macro}{\adl@makevlr}
% \changes{v1.7-3}{2004/5/21}
% 	{Modified to initialize \cs{adl@dashcolor} and \cs{adl@gapcolor}.}
% \changes{v1.73-1}{2016/04/28}
% 	{Add $\eta\EQ\cs{adl@lastconn}\gets0$.}
% \SpecialIndex{\adl@addvl}
% \SpecialIndex{\adl@lastconn}
%
% The macro |\adl@makevlr| corresponds to the lines (3)--(4) and (30).
% \end{macro}
%
%    \begin{macrocode}
\def\adl@makevlr{\@tempcnta\adl@totalheight \@tempcntb\adl@totalheight
        \let\adl@lastconn\z@ \adl@dash\m@ne \adl@gap\m@ne
        \let\adl@dashcolor\relax \let\adl@gapcolor\relax
        \adl@connectedfalse \adl@doublerulefalse \def\@tempb{}%
        \expandafter\adl@imakevlr\adl@rows\@nil;%
        \adl@addvl
        \edef\adl@rows{\@tempb}}

%    \end{macrocode}
%
% \begin{macro}{\adl@imakevlr}
% \begin{macro}{\adl@iimakevlr}
%
% \changes{v1.73-1}{2016/04/28}
% 	{Modify the definition of \cs{adl@connect} to pass $h$ to
%	 \cs{adl@@connect}.}
%
% \begin{macro}{\adl@endmakevlr}
% \SpecialIndex{\@elt}
% \SpecialIndex{\adl@connect}
%
% The macro |\adl@imakevlr|\meta{$r$}|;| corresponds to the lines (5)--(6),
% and the macro |\adl@iimakevlr(|\meta{$C$}|/|\meta{$h$}|)| to (7) and (28).
% \end{macro}\end{macro}\end{macro}
%
%    \begin{macrocode}
\def\adl@imakevlr#1;{\def\@tempa{#1}\ifx\@tempa\@nnil \let\next\relax
        \else \adl@iimakevlr#1\let\next\adl@imakevlr \fi \next}
\def\adl@iimakevlr(#1/#2){\let\@elt\adl@iiimakevlr
        \def\adl@connect{\adl@@connect#2}%
        \let\adl@endmakevlr\adl@endmakevlrcut
        #1\adl@endmakevlr
        \let\@elt\relax \let\adl@connect\relax
        \advance\@tempcntb-#2\edef\@tempb{\@tempb(\@tempc/#2);}}

%    \end{macrocode}
%
% \begin{macro}{\adl@iiimakevlr}
% \changes{v1.7-3}{2004/5/21}
% 	{Modified to check color indentity.}
% \changes{v1.73-1}{2016/04/28}
% 	{Replace two occurences of $\tau\gets\beta$ with
%	 $\tau\gets\beta+\eta$ and add $\eta\gets0$, where
%	 $\eta\EQ\cs{adl@lastconn}$.}
% \begin{macro}{\adl@ivmakevlr}
% \changes{v1.7-3}{2004/5/21}
% 	{Modified not to see $d$ and $g$ which now have colors.}
% \begin{macro}{\adl@vmakevlr}
% \begin{macro}{\adl@endmakevlrcut}
% \changes{v1.73-1}{2016/04/28}
% 	{Add $\eta\EQ\cs{adl@lastconn}\gets0$.}
% \begin{macro}{\adl@endmakevlrconn}
% \begin{macro}{\adl@@connect}
% \changes{v1.73-1}{2016/04/28}
% 	{Add $\eta\EQ\cs{adl@lastconn}\gets h$ with the added
%	 argument $h$.}
%
% The correspondance of the lines (8)--(30) is a little bit complicated.  As
% shown above, |\adl@iimakevlr| expands $C$ attaching the sentinel
% |\adl@endmakevlr|.
%
% \begin{enumerate}
% \item
% If $C\neq\meta{}$ and $C\neq\meta{connect}$, $C$ has at least one
% |\@elt|\meta{$c$}\meta{$d$}\meta{$g$} which is made |\let|-equal to
% |\adl@iiimakevlr| by |\adl@iimakevlr|.  Thus the lines (10)--(21) and
% (25)--(26) are performed by |\adl@iiimakevlr|.
%
% Then;
% \begin{enumerate}
% \item
% if $c=\gamma$, |\@elt| becomes |\let|-equal to |\adl@ivmakevlr| which
% corresponds to (22) in the case of $C'\neq\meta{}$.  Then |\adl@vmakevlr|
% is invoked for (23) and to eat the sentinel |\adl@endmakevlr|.  If
% $C'=\meta{}$, |\adl@endmakevlrconn| is invoked, because the sentinel
% |\adl@endmakevlr| is made |\let|-equal to it by |\adl@iiimakevlr|, for
% (23) (i.e.\ $C\gets\meta{}$).
%
% \item
% if $c\neq\gamma$, |\adl@vmakevlr| is invoked to perform implicit $C\gets
% C$ operation and to eat the sentinel.
% \end{enumerate}
%
% \item
% If $C=\meta{connect}$, i.e.\ it has only one element |\adl@connect|, the
% macro |\adl@@connect| is invoked with $h$ because it is |\define|-dl to be
% $|\adl@@connect|\meta{h}$.  The macro performs (9) and implict $C\gets C\ {}
% (=\meta{connect})$ eating the sentinel.
%
% \item
% If $C=\meta{}$, |\adl@endmakevlrcut| that is |\let|-equal to the sentinel
% |\adl@endmakevlr| is invoked to perform (8) and implicit $C\gets C
% \ (=\meta{})$.
% \end{enumerate}
% \end{macro}\end{macro}\end{macro}\end{macro}\end{macro}\end{macro}
% \SpecialIndex{\adl@addvl}
%
%    \begin{macrocode}
\def\adl@iiimakevlr#1#2#3#4#5{\let\@elt\adl@ivmakevlr \let\next\relax
        \ifnum#1=\adl@currentcolumn\relax
                \let\adl@endmakevlr\adl@endmakevlrconn
                \@tempswafalse
                \ifnum#2=\adl@dash\relax
                \ifnum#3=\adl@gap\relax
                \def\@tempa{#4}\ifx\@tempa\adl@dashcolor
                \def\@tempa{#5}\ifx\@tempa\adl@gapcolor
                        \@tempswatrue
                \fi\fi\fi\fi
                \if@tempswa
                        \ifadl@connected\else
                                \@tempcnta\@tempcntb
                                \advance\@tempcnta\adl@lastconn\relax
                                \adl@connectedtrue
                        \fi
                \else
                        \adl@addvl
                        \adl@dash#2\relax \adl@gap#3\relax
                        \def\adl@dashcolor{#4}\def\adl@gapcolor{#5}%
                        \@tempcnta\@tempcntb
                        \advance\@tempcnta\adl@lastconn\relax
                        \adl@connectedtrue
                \fi
        \else
                \adl@addvl
                \def\next{\adl@vmakevlr\@elt{#1}{#2}{#3}{#4}{#5}}%
        \fi
        \let\adl@lastconn\z@ \next}
\def\adl@ivmakevlr#1{%
        \ifnum#1=\adl@currentcolumn \adl@doubleruletrue \fi
        \adl@vmakevlr\@elt{#1}}
\def\adl@vmakevlr#1\adl@endmakevlr{\def\@tempc{#1}}
\def\adl@endmakevlrcut{\adl@addvl \let\adl@lastconn\z@ \def\@tempc{}}
\def\adl@endmakevlrconn{\def\@tempc{}}
\def\adl@@connect#1\adl@endmakevlr{\def\adl@lastconn{#1}%
        \def\@tempc{\adl@connect}}

%    \end{macrocode}
%
% \begin{macro}{\adl@addvlL}
% \changes{v1.7-3}{2004/5/21}
% 	{Modified to add colors to $\delta$ and $\xi$.}
% \begin{macro}{\adl@addvlR}
% \changes{v1.7-3}{2004/5/21}
% 	{Modified to add colors to $\delta$ and $\xi$.}
% The macro |\adl@addvlL| corresonds to the lines (38)--(42), i.e.\ the
% procedure {\it add}.  The macro |\adl@addvlR| performs simlar operations,
% but its conceptual code is the following.
%
% \begin{quote}\defprog\setcounter{enumi}{37}
% \begin{tabbing}
%\!{\bf procedure} ${\it add}(\tau,\beta,\delta,\xi)$ \BEGIN\\
%\!\qquad\=
%	\IF\ {\it conn} \THEN\ \BEGIN\\
%\!\>\qquad\=	$\LAMBDA\_
%		    \meta{$\meta{$\beta,\tau-\beta,\delta,\xi$},\LAMBDA$}$;
%		${\it conn}\_\FALSE$;\\
%\!\>	\END;\\
%\!\END;
% \end{tabbing}
% \end{quote}
% \end{macro}\end{macro}
% \SpecialIndex{\ifadl@connected}
% \SpecialIndex{\adl@vlrow}
% \SpecialIndex{\adl@vl}
% \SpecialIndex{\adl@dash}
% \SpecialIndex{\adl@gap}
%
%    \begin{macrocode}
\def\adl@addvlL{\ifadl@connected
        \advance\@tempcnta-\@tempcntb
        \edef\adl@vlrow{\adl@vlrow
                \adl@vl{\number\@tempcntb}{\number\@tempcnta}%
                       {\number\adl@dash}{\number\adl@gap}%
                       {\adl@dashcolor}{\adl@gapcolor}}%
        \adl@connectedfalse \fi}
\def\adl@addvlR{\ifadl@connected
        \advance\@tempcnta-\@tempcntb
        \edef\adl@vlrow{\adl@vl{\number\@tempcntb}{\number\@tempcnta}%
                               {\number\adl@dash}{\number\adl@gap}%
                               {\adl@dashcolor}{\adl@gapcolor}\adl@vlrow}%
        \adl@connectedfalse \fi}

%    \end{macrocode}
%
% \begin{macro}{\adl@drawvl}
% \changes{v1.54-1}{2003/08/25}
%	{Introduced to draw vertical lines correctly if \cs{ADLsomewide}.}
% \begin{macro}{\adl@@vl}
% \changes{v1.4-2-2}{1999/06/25}
%	{Modified to make vertical line null wide only if \cs{ADLnullwide}.}
% \changes{v1.4-2-4}{1999/06/25}
%	{Modified to use \cs{adl@draw}.}
% \changes{v1.54-1}{2003/08/25}
%	{Insert a negative skip to left/right of the line if \cs{ADLsomewide}.}
% \changes{v1.7-3}{2004/5/21}
% 	{Modified to color dashs and gaps.}
% \begin{macro}{\adl@vl@leftskip}
% \begin{macro}{\adl@vl@rightskip}
% \SpecialIndex{\adl@vlrowL}
% \SpecialIndex{\adl@vlrowL}
% \SpecialIndex{\adl@vl}
% \SpecialIndex{\adl@draw}
% \SpecialIndex{\set@color}
%
% After the the macros |\adl@vlrowL| and |\adl@vlrowR| are constructed, they
% are expanded to draw vertical lines by |\adl@drawvl|.  Prior to the
% expansion, the macro |\adl@drawvl| globally defines |\adl@vl@leftskip| and
% |\adl@vl@rightskip|, which are the amount of negative spaces inserted to the
% left\slash right of a vertical line, as follows.
%
% \begin{eqnarray*}
% \cs{adl@vl@leftskip}&=&\cases{
%	\cs{arrayrulewidth}/2&	if \cs{ifadl@zwrule}\cr
%	0&			else if leftside\cr
%	\cs{arrayrulewidth}&	otherwise}\\
% \cs{adl@vl@rightskip}&=&\cases{
%	\cs{arrayrulewidth}/2&	if \cs{ifadl@zwrule}\cr
%	0&			else if rightside\cr
%	\cs{arrayrulewidth}&	otherwise}
% \end{eqnarray*}
%
% That is, if |\ADLnulwide| is in effect, a vertical line is surrounded by
% horizontal spaces of $-|\arrayrulewidth|/2$ to adjust the center of the
% line to the left or right edge of its column.  Otherwise, a horizontal
% space $-|\arrayrulewidth|$ is inserted after (before) the line is drawn to 
% adjust its left (right) edge to the left (right) edge of the column
% \footnote{
% Before v1.54, the horizontal spaces was not inserted if \cs{ADLsomewide}
% and thus disconnected lines were not aligned vertically.}.
% 
% Then the macros |\adl@vlrowL| and |\adl@vlrowR| are expanded.  These
% macros will have |\adl@vl|, which is made |\let|-equal to
% \hbox{|\adl@@vl|} prior to the expansion, to draw a vertical line.  The
% macro
% |\adl@@vl|\meta{$\beta$}\meta{$\lambda$}\meta{$\delta_l$}\meta{$\gamma_l$}
% \meta{$\delta_c$}\meta{$\gamma_c$} ($x_l$ and $x_c$ are length and color)
% draws a sloid line if $\gamma_l=0$ or a dash-line otherwise in a |\vbox| of
% $\lambda=\tau-\beta$ high and |\raise|-s it by $\beta$.  The method to
% draw a dash line in the |\vbox| is analogous to that for horizontal line
% shown in \S\ref{sec:imp-hline}, except that a line is surrounded by
% horizontal spaces of |\adl@vl@leftskip| and |\adl@vl@rightskip|.  Coloring 
% gaps is done by drawing a vertical rule setting $\gamma_c$ by |\set@color|
% prior to dash line drawing if $\gamma_c$ is not |\relax|.  To color
% dashes or solid line, |\set@color| with $\delta_c$ is done if it is not
% |\relax| before line drawing.
% \end{macro}\end{macro}\end{macro}\end{macro}
%
%    \begin{macrocode}
\def\adl@drawvl{%
        \omit \relax \ifadl@zwvrule
                        \gdef\adl@vl@leftskip{.5\arrayrulewidth}%
                        \global\let\adl@vl@rightskip\adl@vl@leftskip
                \else   \global\let\adl@vl@leftskip\z@
                        \global\let\adl@vl@rightskip\arrayrulewidth
                \fi \adl@vlrowL \cr
        \omit \relax \ifadl@zwvrule
                        \gdef\adl@vl@leftskip{.5\arrayrulewidth}%
                        \global\let\adl@vl@rightskip\adl@vl@leftskip
                \else   \global\let\adl@vl@leftskip\arrayrulewidth
                        \global\let\adl@vl@rightskip\z@
                \fi \adl@vlrowR \cr}

\def\adl@@vl#1#2#3#4#5#6{\vbox to\z@{\vss\hbox{%
        \hskip-\adl@vl@leftskip
        \ifnum#3=\z@\else \def\@tempa{#6}\ifx\@tempa\adl@nocolor\else
                \raise#1sp\hbox{\let\current@color\@tempa \set@color
                        \vrule height#2sp width\arrayrulewidth}%
                \hskip-\arrayrulewidth \fi \fi
        \raise#1sp\vbox to#2sp{
                \def\@tempa{#5}\ifx\@tempa\adl@nocolor\else
                        \let\current@color\@tempa \set@color \fi
                \ifnum#3=\z@
                        \hrule height#2sp depth\z@ width\arrayrulewidth
                \else   \@tempdima#3sp \@tempdimb#4sp
                        \adl@draw\adl@hrule\vskip\vbox
                \fi}%
        \hskip-\adl@vl@rightskip}}}

%%^L
%    \end{macrocode}
%
% 
% 
% \subsection{Drawing Dash-lines}
% \label{sec:imp-draw}
% \changes{v1.4-1}{1999/06/25}
%	{Section 4.12 is added.}
% 
% \begin{macro}{\adl@vrule}
% \changes{v1.4-2-4}{1999/06/25}
%	{Introduced to draw a dash for horizontal lines in \cs{adl@draw}.}
% \begin{macro}{\adl@hrule}
% \changes{v1.4-2-4}{1999/06/25}
%	{Introduced to draw a dash for vertical lines in \cs{adl@draw}.}
% As explained later, horizontal and vertical lines are drawn by a common
% macro |\adl@draw| to which the length of a dash segment, $d$, is passed
% through |\@tempdima|.  The macro also has an argument that is either
% |\adl@vrule| to draw a dash for {\em horizontal} lines or |\adl@hrule| for 
% {\em vertical}.  These two macros commonly have one argument \meta{f} to
% draw a dash of $f\times d$ long and of |\arrayrulewidth| wide.
% \end{macro}\end{macro}
%
%    \begin{macrocode}

%% Draw Dash Lines (\adl@vrule/\adl@hrule, \hskip/\vskip, \hbox/\vbox)

\def\adl@vrule#1{\vrule\@width#1\@tempdima\@height\arrayrulewidth\relax}
\def\adl@hrule#1{\hrule\@height#1\@tempdima\@width\arrayrulewidth\relax}
%    \end{macrocode}
% 
% \begin{macro}{\adl@drawi}
% \changes{v1.4-2-4}{1999/06/25}
%	{Introduced as \cs{adl@draw} in mode 1.}
% \begin{macro}{\adl@drawii}
% \changes{v1.4-2-4}{1999/06/25}
%	{Introduced as \cs{adl@draw} in mode 2.}
% \begin{macro}{\adl@drawiii}
% \changes{v1.4-2-4}{1999/06/25}
%	{Introduced as \cs{adl@draw} in mode 3.}
% \begin{macro}{\adl@draw}
% \changes{v1.4-2-4}{1999/06/25}
%	{Introduced as the mode and axis independent line drawing macro.}
% \SpecialIndex{\xleaders}
% \SpecialIndex{\cleaders}
%
% The macro |\adl@draw| is to draw a horizontal or vertical line.  It is
% |\let|-equal to one of |\adl@drawi|, |\adl@drawii| and |\adl@drawiii|
% according to the drawing mode specified by |\ADLdrawingmode|.  These three
% macros have common interface, |\@tempdima| and \hbox{|\@tempdimb|} for the
% length of dash and gap, $d$ and $g$, and three arguments \meta{rule},
% \meta{skip} and \meta{box} with which |\adl@draw| is called in the
% following manner.
%
% \begin{eqnarray*}
% \hbox{|\adl@draw\adl@vrule\hskip\hbox|}&\ldots&\hbox{horizontal}\\
% \hbox{|\adl@draw\adl@hrule\vskip\vbox|}&\ldots&\hbox{vertical}
% \end{eqnarray*}
%
% The drawing methods in three modes have been explained in
% \S\ref{sec:imp-ps2}.  More specifically, |\adl@drawi| for mode 1, to which
% |\adl@draw| is |\let|-equal by default, conceptually performs the
% following operations.
%
% \begin{quote}
% \def\arg#1{\hbox{{\tt\char`\{}#1{\tt\char`\}}}}
% \meta{rule}\arg{1/2}\quad\meta{skip}$(g/2)$\\
% |\xlearders|\meta{box}
%	\arg{\meta{skip}$(g/2)$ \meta{rule}\arg{1} \meta{skip}$(g/2)$}\\
% \phantom{\texttt{.xleaders}}\meta{skip}(0 |plus| 1|fil minus| 1|fil|)\\
% \meta{skip}$(g/2)$\quad\meta{rule}\arg{1/2}
% \end{quote}
%
% The conceptual operations of |\adl@drawii| for mode 2 are as follows.
%
% \begin{quote}
% \def\arg#1{\hbox{{\tt\char`\{}#1{\tt\char`\}}}}
% \meta{rule}\arg{1/2}\quad\meta{skip}$(g/2)$\\
% \meta{box}\arg{\meta{skip}$(g/2)$ \meta{rule}\arg{1} \meta{skip}$(g/2)$}\quad
%	\meta{skip}$(-d-g)$\\
% |\xlearders|\meta{box}
%	\arg{\meta{skip}$(g/2)$ \meta{rule}\arg{1} \meta{skip}$(g/2)$}\\
% \phantom{\texttt{.xleaders}}\meta{skip}(0 |plus| 1|fil minus| 1|fil|)\\
% \meta{skip}$(-d-g)$\quad
%    \meta{box}\arg{\meta{skip}$(g/2)$ \meta{rule}\arg{1} \meta{skip}$(g/2)$}\\
% \meta{skip}$(g/2)$\quad\meta{rule}\arg{1/2}
% \end{quote}
%
% The macro |\adl@drawiii| for mode 3 is quite similar to |\adl@drawi|
% except that |\xleaders| is replaced by |\cleaders|.  This replacement is
% done by temporarily |\let|-ing |\xleaders| be equal to |\cleaders|.
% \end{macro}\end{macro}\end{macro}\end{macro}
%
% \SpecialIndex{\adl@box}
%
%    \begin{macrocode}
\def\adl@drawi#1#2#3{%
        #1{.5}#2.5\@tempdimb
        \xleaders#3{#2.5\@tempdimb #1{1}#2.5\@tempdimb}%
                #2\z@ plus1fil minus1fil\relax
        #2.5\@tempdimb #1{.5}}
\def\adl@drawii#1#2#3{%
        \setbox\adl@box#3{#2.5\@tempdimb #1{1}#2.5\@tempdimb}%
        #1{.5}#2.5\@tempdimb
        \copy\adl@box #2-\@tempdima #2-\@tempdimb
        \xleaders\copy\adl@box#2\z@ plus1fil minus1fil\relax
        #2-\@tempdima #2-\@tempdimb \copy\adl@box
        #2.5\@tempdimb #1{.5}}
\def\adl@drawiii#1#2#3{{\let\xleaders\cleaders \adl@drawi#1#2#3}}
\let\adl@draw\adl@drawi

%    \end{macrocode}
% 
% \begin{macro}{\ADLdrawingmode}
% \changes{v1.4-2-4}{1999/06/25}
%	{Introduced to specify drawing mode.}
% The macro |\ADLdrawingmode|\Meta{m} defines the drawing mode by |\let|-ing 
% |\adl@draw| be equal to |\adl@drawi| if $m=1$, and so on.  If \meta{m} is
% neither 1, 2 nor 3, it is assumed as 1.
% \end{macro}
%
%    \begin{macrocode}
\def\ADLdrawingmode#1{\ifcase #1%
        \let\adl@draw\adl@drawi \or
        \let\adl@draw\adl@drawi \or
        \let\adl@draw\adl@drawii \or
        \let\adl@draw\adl@drawiii \else
        \let\adl@draw\adl@drawi \fi}

%%^L
%    \end{macrocode}
% 
% 
% 
% \subsection{Shorthand Activation}
% \label{sec:imp-sh}
% \changes{v1.4-1}{1999/06/25}
%	{Section 4.13 is added.}
%
% \begin{macro}{\adl@Array}
% \changes{v1.4-2-3}{1999/06/25}
%	{Introduced as the body of \cs{Array}.}
% \begin{macro}{\adl@Tabular}
% \changes{v1.4-2-3}{1999/06/25}
%	{Introduced as the body of \cs{Tabular}.}
% \begin{macro}{\adl@Tabularstar}
% \changes{v1.4-2-3}{1999/06/25}
%	{Introduced as the body of \cs{Tabular*}.}
% \begin{macro}{\adl@Longtable}
% \changes{v1.6}{2003/8/25}
%	{Introduced as the body of \cs{Longtable}.}
% \SpecialIndex{\array}
% \SpecialIndex{\tabular}
% \SpecialIndex{\tabular*}
% \SpecialIndex{\longtable}
%
% The macros|\adl@Array|, |\adl@Tabular|, |\adl@Tabular*| and
% |\adl@Longtable| start environments |array|, |tabular|, |tabular*| and
% |longtable| respectively, turning |\ifadl@inactive| false to activate
% dash-line functions.  We will |\let| macros |\Array| etc.\ be equal to
% them for shorthand activation.
% \end{macro}\end{macro}\end{macro}\end{macro}
%
%    \begin{macrocode}

%% Shorthand Activation

\def\adl@Array{\adl@inactivefalse \array}
\def\adl@Tabular{\adl@inactivefalse \tabular}
\def\adl@Tabularstar{\adl@inactivefalse \@nameuse{tabular*}}
\def\adl@Longtable{\adl@inactivefalse \longtable}

%    \end{macrocode}
% 
% \begin{macro}{\@notdefinable}
% \begin{macro}{\adl@notdefinable}
% \changes{v1.4-2-3}{1999/06/25}
%	{Introduced to check if \cs{Array} etc. are definable.}
% \SpecialIndex{\@ifdefinable}
%
% Before making |\Array| etc.\ |\let|-equal to |\adl@Array| etc., we have to 
% check if these macros having too natural names have already used.  This
% check is done by |\@ifdefinable| that will call |\@notdefinable| for the
% complaint if undefinable.  Since we want to complain with our own
% warning message, |\@notdefinable| is temporarily |\def|-ined so that it
% simply |\def|-ines a macro |\adl@notdefinable| as empty.  Therefore,
% |\adl@notdefinebale| will have some definition if one of |\Array|,
% |\Tabular|, |\Tabular*| and |\Longtable| (if \textsf{longtable} is loaded) 
% cannot be defined, while it will stay undefined otherwise.
% \end{macro}\end{macro}
%
%    \begin{macrocode}
\begingroup
\def\@notdefinable{\gdef\adl@notdefinable{}}
\@ifdefinable\Array\relax
\@ifdefinable\Tabular\relax
\expandafter\@ifdefinable\csname Tabular*\endcsname\relax
\ifx\longtable\undefined\else \@ifdefinable\Longtable\relax \fi
\endgroup

%    \end{macrocode}
% 
% \begin{macro}{\Array}
% \changes{v1.4-2-3}{1999/06/25}
%	{Introduced as the always-active \cs{array}.}
% \begin{macro}{\Tabular}
% \changes{v1.4-2-3}{1999/06/25}
%	{Introduced as the always-active \cs{tabular}.}
% \begin{macro}{\Tabular*}
% \changes{v1.4-2-3}{1999/06/25}
%	{Introduced as the always-active \cs{tabular*}.}
% \begin{macro}{\Longtable}
% \changes{v1.6}{2003/8/25}
%	{Introduced as the always-active \cs{longtable}.}
% \begin{macro}{\endArray}
% \changes{v1.4-2-3}{1999/06/25}
%	{Introduced to \cs{end} the environment \texttt{Array}.}
% \begin{macro}{\endTabular}
% \changes{v1.4-2-3}{1999/06/25}
%	{Introduced to \cs{end} the environment \texttt{Tabular}.}
% \begin{macro}{\endTabular*}
% \changes{v1.4-2-3}{1999/06/25}
%	{Introduced to \cs{end} the environment \texttt{Tabular*}.}
% \begin{macro}{\endLongtable}
% \changes{v1.6}{2003/8/25}
%	{Introduced to \cs{end} the environment \texttt{Longtable}.}
% \SpecialIndex{\PackageWarning}
% \SpecialIndex{\@warning}
% \SpecialIndex{\MessageBreak}%
%
% If |\adl@notdefinable| is |\undefined| indicating that all |\Array| etc.\
% are definable, we |\let| them be equal to |\adl@Array| etc.  We also
% |\let| ending macros |\endArray| etc.\ be equal to |\endarray| etc.
% Note that |\Longtable| and |\endLongtable| are defined only when
% \textsf{longtable} is loaded, and |\endLongtable| is |\def|-ined as (not
% being |\let|-equal to) |\endlongtable| because its definition of our own
% is not given yet.
% 
% Otherwise, we complain with a warning message put by |\PackageWarning| if
% it is defined (i.e.\ \LaTeXe) or |\@warning| otherwise (i.e.\ \LaTeX-2.09).
% \end{macro}\end{macro}\end{macro}\end{macro}
% \end{macro}\end{macro}\end{macro}\end{macro}
%
%    \begin{macrocode}
\ifx\adl@notdefinable\undefined
        \let\Array\adl@Array
        \let\Tabular\adl@Tabular
        \expandafter\let\csname Tabular*\endcsname\adl@Tabularstar
        \let\endArray\endarray
        \let\endTabular\endtabular
        \expandafter\let\csname endTabular*\endcsname\endtabular
        \ifx\longtable\undefined\else
                \let\Longtable\adl@Longtable
                \def\endLongtable{\endlongtable}
        \fi
\else
\begingroup
\ifx\longtable\undefined
\def\@tempa{Array and Tabular are not defined because one of them\MessageBreak
        has been defined}
\else
\def\@tempa{Array/Tabular/Longtable are not defined because \MessageBreak
        one of them has been defined}
\fi
\ifx\PackageWarning\undefined
        \def\MessageBreak{^^J}
        \@warning\@tempa
\else
        \let\on@line\empty
        \PackageWarning{arydshln}\@tempa
\fi
\endgroup
\fi

%    \end{macrocode}
% 
% \begin{macro}{\ADLnoshorthanded}
% \changes{v1.4-2-3}{1999/06/25}
%	{Introduced to nullify macros for shorthand activation.}
% \changes{v1.6}{2003/8/25}
%	{\cs{Longtable} and \cs{endLongtable} are added.}
% If a user wishes to define an environment named |Array| or |Tabular(*)|
% (or |Longtable| if \textsf{longtable} is in use) by him/herself or by
% loading other packages {\em after} \textsf{arydshln} is loaded,
% |\newenvironment| for |Array| etc.\ will fail because they have already
% been undefinable.  The macro |\ADLnoshorthanded| makes them definable
% again by |\let|-ing them and their ending counterparts be equal to
% |\relax|.
% \end{macro}
%
%    \begin{macrocode}
\def\ADLnoshorthanded{%
        \let\Array\relax
        \let\Tabular\relax
        \expandafter\let\csname Tabular*\endcsname\relax
        \let\endArray\relax
        \let\endTabular\relax
        \expandafter\let\csname endTabular*\endcsname\relax
        \ifx\longtable\undefined\else
                \let\Longtable\relax
                \let\endLongtable\relax \fi}

%    \end{macrocode}
% 
% \vskip15\baselineskip\endgraf\vskip-15\baselineskip\vskip0pt
% \begin{macro}{\adl@act@arrayclassz}
% \changes{v1.54-2}{2003/08/25}
%	{Introduced to activate \cs{@arrayclassz} etc. again.}
% \begin{macro}{\adl@act@tabclassz}
% \begin{macro}{\adl@act@classz}
% \begin{macro}{\adl@act@@startpbox}
% \begin{macro}{\adl@act@@endpbox}
% \begin{macro}{\adl@act@endpbox}
% \begin{macro}{\adl@act@cr}
% \begin{macro}{\adl@act@argcr}
% \begin{macro}{\adl@act@cline}
% \begin{macro}{\adl@act@endarray}
% \begin{macro}{\adl@act@hline}
% \begin{macro}{\adl@act@ihdashline}
% \begin{macro}{\adl@act@cdline}
% \begin{macro}{\adl@act@@vlineL}
% \begin{macro}{\adl@act@@vlineR}
% \SpecialIndex{\@arrayclassz}
% \SpecialIndex{\@tabclassz}
% \SpecialIndex{\@classz}
% \SpecialIndex{\@@startpbox}
% \SpecialIndex{\@@endpbox}
% \SpecialIndex{\@endpbox}
% \SpecialIndex{\adl@cr}
% \SpecialIndex{\adl@argcr}
% \SpecialIndex{\cline}
% \SpecialIndex{\adl@endarray}
% \SpecialIndex{\adl@hline}
% \SpecialIndex{\adl@ihdashline}
% \SpecialIndex{\adl@cdline}
% \SpecialIndex{\adl@@vlineL}
% \SpecialIndex{\adl@@vlineR}
%
% Finally here we define {\em active} version of |\@arrayclassz| named
% |\adl@act@arrayclassz| etc. for |\adl@activate| (see \S\ref{sec:imp-init}).
% The definitions are simply done by |\let|-ing |\adl@act@arrayclassz| equal 
% to |\@arrayclassz| etc\footnote{%
%
% Alternatively, we may define \cs{adl@act@arrayclassz} in place of
% \cs{@arrayclassz} but the author chose this way to minimize the possiblity 
% of {\em enbug}.}.
% \end{macro}\end{macro}\end{macro}\end{macro}\end{macro}
% \end{macro}\end{macro}\end{macro}\end{macro}\end{macro}
% \end{macro}\end{macro}\end{macro}\end{macro}\end{macro}
%
%    \begin{macrocode}
\let\adl@act@arrayclassz\@arrayclassz
\let\adl@act@tabclassz\@tabclassz
\ifadl@usingarypkg \let\adl@act@classz\@classz \fi
\let\adl@act@@startpbox\@@startpbox
\let\adl@act@@endpbox\@@endpbox
\let\adl@act@endpbox\@endpbox
\let\adl@act@cr\adl@cr
\let\adl@act@argcr\adl@argcr
\let\adl@act@endarray\adl@endarray
\let\adl@act@hline\adl@hline
\let\adl@act@ihdashline\adl@ihdashline
\let\adl@act@cdline\adl@cdline
\let\adl@act@@vlineL\adl@@vlineL
\let\adl@act@@vlineR\adl@@vlineR

%%^L
%    \end{macrocode}
% 
% 
% 
% \subsection{Compatibility with \textsf{colortab}}
% \label{sec:imp-colortab}
%
% \begin{macro}{\adl@CC@}
% \begin{macro}{\CC@}
% \changes{v1.5-1}{2000/07/12}
%	{Modified to save/restore globals before/after height measurement.}
% \SpecialIndex{\LCC}
% \SpecialIndex{\adl@arraysave}
% \SpecialIndex{\adl@arrayrestore}
%
% The package \textsf{colortab} has a macro;
%
% \begin{quote}
% |\LCC|\meta{colorspec}|\\|\meta{rows}|\ECC|
% \end{quote}
%
% to color \meta{rows} referring \meta{colorspec}.  The macro |\CC@|, the
% heart of the coloring function, first makes a box with \meta{rows} using
% |\@preamble| to measure the height of \meta{rows}, then makes a row putting
% a heavy rule of the height in each column with a color command for the
% column specified by \meta{colorspec}, and finally puts \meta{rows}
% overlaying them on the colored rule.  Therefore \meta{rows} is processed
% twice by |\CC@| to update |\global| registers/structures incorrectly.
% 
% Thus we modify |\CC@|, if the package \textsf{colortab} is provided, to 
% save |\global| stuff by |\adl@arraysave| before the height measurement and 
% restore them by |\adl@arrayrestore| after that.
% \end{macro}\end{macro}
%
%    \begin{macrocode}

%% Compatibility with colortab

\def\adl@CC@#1#2#3{%
  \ifcolortab
    \noalign{%
      \adl@arraysave
      \setbox\CT@box=\vbox{#1#3\crcr\egroup}%
      \adl@arrayrestore
      \CT@dim=\ht\CT@box
      \global\advance\CT@dim by \dp\CT@box
      \def\CT@next{}%
      \futurelet\next\CT@columncolor#2&\@nil}%
    \CT@next\cr
    \noalign{\vskip-\CT@dim}%
  \fi
  #3}
\ifx\ColortabLoaded\undefined\else
\let\CC@\adl@CC@
\fi

%%^L
%    \end{macrocode}
% 
% 
% 
% \subsection{Compatibility with \textsf{longtable}}
% \label{sec:imp-longtable}
%
% \changes{v1.6}{2003/8/25}
%	{\S\ref{sec:imp-longtable} is added.}
% 
% \SpecialOptIndex{longtable}{package}
% \SpecialOptIndex{longtable}{environment}
% \SpecialIndex{\output}
% \SpecialIndex{\adl@rowsL}
% \SpecialIndex{\adl@rowsR}
%
% Making \textsf{arydshln} compatible with \textsf{longtable} is a hard job
% because a |longtable| consists of multiple {\em chunks} and each chunk is
% a distinct |\halign|.  We could draw vertical lines in each chunks as we
% do with ordinary |array|\slash |table|.  However this straightforward
% solution should {\em break} dash-lines at invisible borders of chunks and
% produce awful results.
%
% Therefore, this implementation draws dash-lines in |\output| routine in
% which we have all the rows to be put in a page.  The hard part is to know
% which rows are being put in |\output|.  This problem is solved by
% extracting the leading part of $R^L$ (|\adl@rowsL|) and $R^R$
% (|\adl@rowsR|) by the height\slash depth of the table fraction to be put
% and removing the part from $R^{L/R}$.
%
% 
%
% \subsubsection{Initialization}
% \label{sec:imp-lt-init}
% \changes{v1.7}{2003/05/21}
%	{Codes for \textsf{longtable} is surrounded by \cs{ifx}/\cs{fi}}
% 
% First of all, the following switch and |\dimen| register are declared.
% \begin{itemize}
% \item
% \begin{macro}{\ifadl@LTfirstpage}
% |\ifadl@LTfirstpage| is tested in |\output| routine to examine if the page 
% being put has the first fraction of a |longtable|.
% \end{macro}
% 
% \item
% \begin{macro}{\adl@LTpagetotal}
% |\adl@LTpagetotal| is set to |\pagetotal| just before the first portion of a
% |longtable| is added to the main vertical list.  Since the |\box255| has
% items preceding the |\longtable| and its first fraction, we can obtain the 
% height of the first fraction by subtracting |\adl@LTpagetotal| from the
% height plus depth of |\box255|.
% \end{macro}
% \end{itemize}
% 
%
%    \begin{macrocode}

%% Compatibility with longtable: initialization

\newif\ifadl@LTfirstpage
\newdimen\adl@LTpagetotal

%    \end{macrocode}
%
% Next, we skip everything if \textsf{longtable} is not in use, or 
% we have undefined-error when we refer to the definitions in it.  Note that
% since |\newif| cannot be in the |\ifx|\slash|\fi| construct, the
% declarations above are excluded.
%
%    \begin{macrocode}
\ifx\longtable\undefined\else

%    \end{macrocode}
%
% 
% \begin{macro}{\adl@LT@array}
% \begin{macro}{\LT@array}
% \changes{v1.73-2}{2016/04/28}
% 	{Add \cs{let}-assignment of \cs{adl@LThdashline} to
%	 \cs{adl@hdashline} so that the \textsf{longtable} version of
%	 \cs{adl@hdashline} is effective only in \texttt{longtable}
%	 environment rather than globally.}
%
% \begin{macro}{\adl@discard}
% \SpecialIndex{\adl@arrayinit}
% \SpecialIndex{\adl@ihdashline}
% \SpecialIndex{\adl@LTihdashline}
% \SpecialIndex{\ifadl@inactive}
%
% Then we redefine the macro |\LT@array|, which is the heart of |\longtable|,
% saving its original definition in |\adl@LT@array|.  The modified
% |\LT@array| first calls |\adl@arrayinit| to initialize the global data
% structures, and sets |\ifadl@LTfirstpage| to true.  Then |\adl@dashline|,
% |\adl@idashline| and |\adl@discard| are made |\let|-equal to the
% \textsf{longtable} versions |\adl@LTdashline| and |\adl@LTidashline|, and
% |\relax| (to inhibit expansion) respectively.  Then the macro calls
% |\adl@LTinactivate| if |\adl@inactive| is true, and finally calls its
% original version |\adl@LT@array|.  Note that since |longtable| cannot be
% nested;
%
% \begin{itemize}
% \item
% |\adl@arraysave| in |\adl@arrayinit| is unnecessary but safe, and thus
% its invocation timing is not so sensitive; and
% 
% \item
% activator is not required.
% \end{itemize}
%
% Also note that the assignment |\adl@ncol| to |\adl@columns| in
% |\adl@arrayinit| is void and thus we will do it afterward.
% \end{macro}\end{macro}\end{macro}
%
% \begin{macro}{\adl@LTinactivate}
% \SpecialIndex{\adl@inactivate}
% \SpecialIndex{\adl@ihdashline}
% \SpecialIndex{\adl@LTinactivehdl}
% \SpecialIndex{\endlongtable}
% \SpecialIndex{\LT@make@row}
% \SpecialIndex{\LT@echunk}
% \SpecialIndex{\LT@end@hd@ft}
% \SpecialIndex{\LT@kill}
% \SpecialIndex{\LT@output}
% \SpecialIndex{\adl@org@endlongtable}
% \SpecialIndex{\adl@org@LT@make@row}
% \SpecialIndex{\adl@org@LT@echunk}
% \SpecialIndex{\adl@org@LT@end@hd@ft}
% \SpecialIndex{\adl@org@LT@kill}
% \SpecialIndex{\adl@org@LT@output}
%
% The macro |\adl@LTinactivate| first calls |\adl@inactivate| to do basic
% inactivation and then |\let|-s the following control sequences be equal to
% their counterparts in \textsf{longtable}.
% \begin{itemize}\item[]
% |\endlongtable|\quad
% |\LT@make@row|\quad
% |\LT@echunk|\quad
% |\LT@end@hd@ft|\quad
% |\LT@kill|\\
% |\LT@output|
% \end{itemize}
% It also make |\adl@idashline| |\let|-equal to its inactive version because 
% we need the macro to find mixed |\hline| and |\hdasnline| sequence.
% \end{macro}
%
%    \begin{macrocode}
\let\adl@LT@array\LT@array
\def\LT@array{\adl@arrayinit \adl@LTfirstpagetrue
        \let\adl@discard\relax \let\adl@hdashline\adl@LThdashline
        \let\adl@ihdashline\adl@LTihdashline
        \ifadl@inactive \adl@LTinactivate \fi
        \adl@LT@array}
\def\adl@LTinactivate{\adl@inactivate
        \let\endlongtable\adl@org@endlongtable
        \let\LT@make@row\adl@org@LT@make@row
        \let\LT@echunk\adl@org@LT@echunk
        \let\LT@end@hd@ft\adl@org@LT@end@hd@ft
        \let\LT@kill\adl@org@LT@kill
        \let\LT@output\adl@org@LT@output
        \let\adl@ihdashline\adl@LTinactivehdl}

%    \end{macrocode}
%
% \begin{macro}{\adl@org@LT@make@row}
% \begin{macro}{\LT@make@row}
% \SpecialIndex{\adl@ncol}
% \SpecialIndex{\adl@columns}
% \SpecialIndex{\adl@LTstartpbox}
% \SpecialIndex{\@startpbox}
% \SpecialIndex{\@@startpbox}
% \SpecialIndex{\adl@LTendpbox}
% \SpecialIndex{\@endpbox}
% \SpecialIndex{\@@endpbox}
% \SpecialIndex{\LT@startpbox}
% \SpecialIndex{\adl@endmbox}
% \SpecialIndex{\adl@@endmbox}
% \SpecialIndex{\adl@LTendmbox}
%
% \changes{v1.72-1}{2016/03/19}
% 	{Modified to add \cs{let}-assignments to \cs{adl@@endmbox} and
%	 \cs{adl@endbmox} so that footnotes are correclty processed at the
%	 closing of a \texttt{m}-type column.}
% 
% The macro |\LT@make@row| is redefined for additional initialization which
% must be done after the original |\LT@array| performs its own
% initialization.  First, |\LT@make@row| itself is reset to its original
% version |\adl@org@LT@make@row| to initialize stuff only once, since
% |\LT@make@row| is called repeatedly at each chunk.  Next |\adl@ncol| is
% assigned to |\adl@columns| to give its value calculated in |\@mkpream|.
% Then macros to begin\slash end |p|-boxes are made |\let|-equal to our own
% version because the original |\LT@array| has done it with its own version.
% Note that |\@@startpbox| and |\@statpbox| are |\let|-equal to our own
% |\adl@LTstartpbox| if \textsf{array} is not in use because with
% \textsf{array} opening a |p|-box is not done by |\@startpbox| but is
% embedded in |\@preamble|.  Also note that we need |\adl@LTendmbox| to
% close |m|-boxes through our own closing macro |\adl@endmbox|, whose
% definition is kept in |\adl@@endmbox|, for \textsf{longtable}-specific
% operations for footnotes.  Finally, the original version
% |\adl@org@LT@make@row| is called.
% \end{macro}\end{macro}
%
%    \begin{macrocode}
\let\adl@org@LT@make@row\LT@make@row
\def\LT@make@row{\let\LT@make@row\adl@org@LT@make@row
        \adl@columns\adl@ncol
        \ifadl@usingarypkg\else
                \let\@@startpbox\adl@LTstartpbox
                \let\@startpbox\adl@LTstartpbox \fi
        \let\@@endpbox\adl@LTendpbox
        \let\@endpbox\adl@LTendpbox
        \let\adl@@endmbox\adl@endmbox
        \let\adl@endmbox\adl@LTendmbox
        \adl@org@LT@make@row}

%%^L
%    \end{macrocode}
% 
% The summary of the activation and inactivation specific to
% \textsf{longtable} is shown in Table~\ref{tab:lt-act}.
% 
% 
% 
% \begin{table}
% \catcode`\|12
% \caption{Active and Inactive \textsf{longtable} Operations}\label{tab:lt-act}
% \def\!{\texttt{\char`\\\char`\\}}
% \centerline{\begin{tabular}[t]{|l||l|l|}\hline
% command&active&inactive\\\hline\hline
% {\tt p b} (open)&&\\
% \qquad with \textsf{array}&	\cs{adl@act@classz}&
%				\cs{adl@org@classz}\\
% &				\quad$\to$\cs{LT@startpbox}&
%				\quad$\to$\cs{LT@startpbox}\\
% \qquad without \textsf{array}&\cs{adl@LTstartpbox}&
%				\cs{LT@startpbox}\\\hline
% {\tt m} (open)&		\cs{adl@act@classz}&
%				\cs{adl@org@classz}\\
% &				\quad$\to$\cs{adl@startmbox}&
%				\quad$\to$\cs{LT@startpbox}\\
% &				\quad$\to$\cs{LT@startpbox}&\\\hline
% {\tt p b} (close)&		\cs{adl@LTendpbox}&
%				\cs{LT@endpbox}\\\hline
% {\tt m} (close)&		\cs{adl@LTendmbox}&
%				\cs{LT@endpbox}\\\hline
% \cs{hline}&			$\to$\cs{adl@act@hline}&
%				$\to$\cs{@gobbletwo}\\\hline
% \cs{hdashline}&		$\to$\cs{adl@LTihdashline}&
%				$\to$\cs{adl@LTinactivehdl}\\
% &				$\to$\cs{adl@act@hline}&
%				$\to$\cs{@gobbletwo}\\\hline
% \cs{endlongtable}&		modified version&
%				\cs{adl@org@endlongtable}\\
% \cs{LT@make@row}&&		\cs{adl@org@LT@make@row}\\
% \cs{LT@echunk}&&		\cs{adl@org@LT@echunk}\\
% \cs{LT@end@hd@ft}&&		\cs{adl@org@LT@end@hd@ft}\\
% \cs{LT@kill}&&		\cs{adl@org@LT@kill}\\
% \cs{LT@output}&&		\cs{adl@org@LT@output}\\\hline
% \end{tabular}}
% \end{table}
% 
% 
% 
% \subsubsection{Ending Chunks}
% \label{sec:imp-lt-echunk}
% 
% \begin{macro}{\adl@org@endlongtable}
% \begin{macro}{\endlongtable}
% \begin{macro}{\adl@org@LT@echunk}
% \begin{macro}{\LT@echunk}
% \begin{macro}{\adl@LTlastrow}
% \SpecialIndex{\adl@height}
% \SpecialIndex{\adl@depth}
% \SpecialIndex{\adl@@cr}
%
% When a chunk is closed with |\crcr|, we have to add the information of the
% last row to $R^{L/R}=|\adl@rowsL|/|R|$ if the row is not finished by an
% explicit |\\|.  This is done by |\adl@LTlastrow| as we did at the first
% job of |\adl@endarray|.  Two chunk closing macros, |\endlongtable| and
% |\LT@echunk|, are modified to call |\adl@LTlastrow| before its original
% job done by |\adl@org@endlongtable| and |\adl@org@LT@echunk| respectively.
% Note that |\adl@LTlastrow| only has |\crcr| and |\noalign| and thus
% another |\crcr| in original |\endlongtable| and |\LT@echunk| is
% no-operation as desired.  Also note that |\adl@LTlastrow| is called twice
% from |\endlongtable|, once from |\LT@echunk| in the original version, but
% it is safe because the first call makes |\adl@height| and |\adl@depth|
% zero and thus the second become no-operation.
% \end{macro}\end{macro}\end{macro}\end{macro}\end{macro}
% 
%
%    \begin{macrocode}

%% Compatibility with longtable: end chunk

\let\adl@org@endlongtable\endlongtable
\def\endlongtable{\adl@LTlastrow \adl@org@endlongtable}

\let\adl@org@LT@echunk\LT@echunk
\def\LT@echunk{\adl@LTlastrow \adl@org@LT@echunk}

\def\adl@LTlastrow{\crcr \noalign{
        \ifdim\adl@height=\z@
        \ifdim\adl@depth=\z@  \else \adl@@cr\z@ \fi
                              \else \adl@@cr\z@ \fi}}

%    \end{macrocode}
%
% \begin{macro}{\adl@org@LT@end@hd@ft}
% \begin{macro}{\LT@end@hd@ft}
% \begin{macro}{\adl@LThfsave}
% \begin{macro}{\adl@LTth}
% \begin{Macro}{\\adl@LTth\LT@firsthead}
% \begin{Macro}{\\adl@LTth\LT@head}
% \begin{Macro}{\\adl@LTth\LT@lastfoot}
% \begin{Macro}{\\adl@LTth\LT@foot}
% \begin{Macro}{\\adl@rowsL\LT@firsthead}
% \begin{Macro}{\\adl@rowsL\LT@head}
% \begin{Macro}{\\adl@rowsL\LT@lastfoot}
% \begin{Macro}{\\adl@rowsL\LT@foot}
% \begin{Macro}{\\adl@rowsR\LT@firsthead}
% \begin{Macro}{\\adl@rowsR\LT@head}
% \begin{Macro}{\\adl@rowsR\LT@lastfoot}
% \begin{Macro}{\\adl@rowsR\LT@foot}
% \SpecialIndex{\endfirsthead}
% \SpecialIndex{\endhead}
% \SpecialIndex{\endlastfoot}
% \SpecialIndex{\endfoot}
% \SpecialIndex{\LT@firsthead}
% \SpecialIndex{\LT@head}
% \SpecialIndex{\LT@lastfoot}
% \SpecialIndex{\LT@foot}
% \SpecialIndex{\adl@rowsL}
% \SpecialIndex{\adl@rowsR}
% \SpecialIndex{\adl@totalheight}
%
% Another chunk ending macro is |\LT@end@hd@ft|\meta{box} to close a
% header\slash footer called by |\endfirsthead|, |\endhead|, |\endlastfoot|
% and |\endfoot| with an argument \meta{box} being |\LT@firsthead|,
% |\LT@head|, |\LT@lastfoot| and |\LT@foot| respectively.
% In order to maintain the information of rows
% $R^{L/R}=|\adl@rowsL|/|R|$ of headers/footers separately from the main one, 
% the modified |\LT@end@hd@ft| saves them together with
% |\adl@totalheight| to weirdly named macros;
%
% \begin{quote}
% |\\adl@LTth|\meta{box}\\
% |\\adl@rowsL|\meta{box}\\
% |\\adl@rowsR|\meta{box}
% \end{quote}
%
% after closing the last row by |\adl@LTlastrow|.  The |\string|
% representation of the macros looks like;
%
% \begin{quote}
% |\\adl@LTth\LT@firsthead|
% \end{quote}
% and so on.  The saving operation is done by 
% the macro |\adl@LThfsave|\meta{box}\meta{info} and is equivalent to;
%
% \begin{quote}
% |\global\let\|\meta{info}\meta{box}=\meta{info}
% \end{quote}
%
% After the saving, three global variables are reinitialized.  Calling
% |\adl@LTlastrow| twice, once from the original version through
% |\LT@echunk| is safe as described above.
% \end{Macro}\end{Macro}\end{Macro}\end{Macro}
% \end{Macro}\end{Macro}\end{Macro}\end{Macro}
% \end{Macro}\end{Macro}\end{Macro}\end{Macro}
% \end{macro}\end{macro}\end{macro}\end{macro}
%
%    \begin{macrocode}
\let\adl@org@LT@end@hd@ft\LT@end@hd@ft
\def\LT@end@hd@ft#1{\adl@LTlastrow
        \noalign{\edef\adl@LTth{\number\adl@totalheight}%
                \adl@LThfsave#1\adl@LTth \global\adl@totalheight\z@
                \adl@LThfsave#1\adl@rowsL\gdef\adl@rowsL{}%
                \adl@LThfsave#1\adl@rowsR\gdef\adl@rowsR{}}
        \adl@org@LT@end@hd@ft#1}
\def\adl@LThfsave#1#2{\expandafter\global\expandafter\let
        \csname\string#2\string#1\endcsname#2}

%    \end{macrocode}
% 
% \begin{macro}{\adl@org@LT@kill}
% \begin{macro}{\LT@kill}
% \begin{macro}{\adl@LTkill}
% \begin{macro}{\adl@LTkillend}
% The additional job for yet another chunk closer |\LT@kill| to kill a
% template row is a little bit harder.  Since the row information might have 
% been added by an explicit |\\| preceding |\kill|, we have to remove it
% from the tail of |\adl@rowsL|\slash|R|, and subtract its $h_i$ from
% |\adl@totalheight| because |\kill|-ed row may be in header\slash footer
% definition.  To do that, modified |\LT@kill| first ensures the information
% addition by |\adl@LTlastrow|, then traverses |\adl@rowsL|\slash|R| adding
% its non-last elements to |\@tempb| by the loop of |\adl@LTkill|, and
% assigns |\@tempb| to |\adl@rowsL|\slash|R| globally by
% |\adl@LTkillend| when |\adl@LTkill| finds the tail.  The macro
% |\adl@LTkillend| also sets the $h_i$ of the last element to |\@tempcnta|,
% which is subtracted from |\adl@totalheight| globally.  Finally, the
% original version |\adl@org@LT@kill| is called.
% \end{macro}\end{macro}\end{macro}\end{macro}
%
%    \begin{macrocode}
\let\adl@org@LT@kill\LT@kill
\def\LT@kill{\adl@LTlastrow \noalign{
            \def\@tempb{}\expandafter\adl@LTkill\adl@rowsL\@nil\adl@rowsL
            \def\@tempb{}\expandafter\adl@LTkill\adl@rowsR\@nil\adl@rowsR
            \global\advance\adl@totalheight-\@tempcnta}%
        \adl@org@LT@kill}
\def\adl@LTkill#1;#2{\def\@tempa{#2}%
        \ifx\@tempa\@nnil\def\next{\adl@LTkillend#1}%
        \else\edef\@tempb{\@tempb#1;}\def\next{\adl@LTkill#2}\fi
        \next}
\def\adl@LTkillend(#1/#2)#3{\global\let#3\@tempb \@tempcnta#2\relax}

%%^L
%    \end{macrocode}
% 
% 
% 
% \subsubsection{Horizontal Lines and \texttt{p}-Boxes}
% \label{sec:imp-lt-hlpbox}
% 
% \begin{macro}{\LT@hline}
% \begin{macro}{\adl@LThdashline}
% \changes{v1.7-1}{2004/5/21}
% 	{Keep original without shift up because it is done by
%	\cs{adl@LThdline}.}
% \changes{v1.73-2}{2016/04/28}
% 	{Renamed from \cs{adl@hdashline} to make it effective only in
%	 \texttt{longtable} environments.}
% \begin{macro}{\adl@LTihdashline}
% \begin{macro}{\adl@LTinactivehdl}
% \begin{macro}{\adl@LThdlrow}
% \SpecialIndex{\hline}
% \SpecialIndex{\adl@hdashline}
% \SpecialIndex{\adl@ihdashline}
% \SpecialIndex{\adl@inactivehdl}
% \SpecialIndex{\adl@hcline}
%
% The macro |\LT@hline|, \textsf{longtable} version of |\hline|, is
% redefined to add pseudo row information to $R^{L/R}$ and to check mixed
% sequence of |\hline| and |\hdashline|\footnote{
%
% In the original \textsf{longtable}, a sequence of three \cs{hline}-s are not 
% recognized.  This buggy feature is fixed in this implementation.}.
%
% We also define the macro |\adl@LTihdashline[|\meta{dash}|/|\meta{gap}|]|
% and its inactive counterpart |\adl@LTinactivehdl| as the
% \textsf{longtable} version of |\adl@ihdashline| and |\adl@inactivehdl|.
% These two macros, the main part of |\hdashline|, are redefined to make it
% possible that |\hdashline| can be broken into two part by \TeX's page
% breaker.
% 
% These three macros call a common routine |\adl@LThdline| after defining
% |\adl@LThdlrow| which makes a row of horizontal (dash) line drawn by
% |\multispan| and |\leaders\hrule| or
% |\adl@hcline[|\meta{dash}|/|\meta{gap}|]|.
% 
% Note that we define |\adl@LThdashline| to make |\adl@hdashline|
% \cs{let}-equal to it in \textsf{longtable} environments because its
% version without \textsf{longtable} performs a part of the job done by
% |\adl@LThdline| as shown soon.
% \end{macro}\end{macro}\end{macro}\end{macro}\end{macro}
%
%    \begin{macrocode}

%% Compatibility with longtable: horizontal lines and p-boxes

\def\LT@hline{\noalign{\ifnum0=`}\fi
        \gdef\adl@LThdlrow{\multispan{\LT@cols}\unskip
                \leaders\hrule\@height\arrayrulewidth\hfill\cr}%
        \adl@LThdline}
\def\adl@LThdashline#1{\noalign{\ifnum0=`}\fi
        \@ifnextchar[%]
                     {#1}%
                     {#1[\dashlinedash/\dashlinegap]}}
\def\adl@LTihdashline[#1/#2]{%
        \gdef\adl@LThdlrow{\multispan{\LT@cols}\unskip
                \adl@hcline\z@[#1/#2]}%
        \adl@LThdline}
\def\adl@LTinactivehdl[#1/#2]{%
        \gdef\adl@LThdlrow{\multispan{\LT@cols}\unskip
                \leaders\hrule\@height\arrayrulewidth\hfill\cr}%
        \adl@LThdline}

%    \end{macrocode}
%
% 
% \begin{macro}{\adl@LThdline}
% \changes{v1.7-1}{2004/5/21}
% 	{Modified to shift up if null-wide.}
% \begin{macro}{\adl@LTxhline}
% \begin{macro}{\adl@LTixhline}
% \SpecialIndex{\@medpenalty}
% \SpecialIndex{\@lowpenalty}
% \SpecialIndex{\adl@hline}
% \SpecialIndex{\adl@connect}
% \SpecialIndex{\adl@discard}
%
% The macro |\adl@LThdline| called by above three macros first
% inserts a vertical penalty 10000  to inhibit page break between the
% horizontal line and preceding row.  Then it inserts
% |\vskip-\arrayrulewidth| with another break inhibitor if
% |\ADLnullwidehline| is in effect, or adds the
% pseudo row information {\it connect}(|\arrayrulewidth|) to $R^{L/R}$ by
% |\adl@hline|\footnote{
%
% Or do noting if inactive and thus it is \cs{let}-equal to \cs{@gobbletwo}.}.
%
% Next, it draw a horizontal (dash) line by
% |\adl@LThdlrow| and checks if the following control sequence is |\hline|
% or |\hdashline| by |\futurelet| and |\adl@LTxhline|.  If |\hline| or
% |\hdashline| is the next token, |\adl@LTixhline| is called to insert a
% vertical penalty of $-$|\@medpenalty| and a vertical space of
% |\doublerulesep|.  The macro |\adl@LTixhline| also adds {\it
% disconnect}(|\doublerulesep|) to $R^{L/R}$ and makes |\adl@LThdlrow| void.
% Otherwise, |\adl@LThdline| inserts a vertical penalty of $-$|\@lowpanalty|
% and a vertical space of $-$|\arrayrulewidth| and draws the horizontal
% (dash) line again by |\adl@LThdlrow|.  Thus a page can be broken between
% two overlaid horizontal (dash) lines\footnote{
% 
% If the page is broken, the horizontal line at the beginning of the
% succeeding page has a width even if \cs{ADLnullwidehline} is in effect.}.
% 
% Two pseudo row information, {\it
% discard}($-$|\arrayrulewidth|) for the negative vertical space which may
% be discarded and {\it connect}(|\arrayrulewidth|) for the second
% horizontal line, are also added to $R^{L/R}$.
% \end{macro}\end{macro}\end{macro}
%
%    \begin{macrocode}
\def\adl@LThdline{\penalty\@M
        \ifadl@zwhrule \vskip-\arrayrulewidth \penalty\@M
        \else          \adl@hline\adl@connect\arrayrulewidth \fi
        \ifnum0=`{\fi}%
        \adl@LThdlrow
        \noalign{\ifnum0=`}\fi
        \futurelet\@tempa\adl@LTxhline}
\def\adl@LTxhline{\ifx\@tempa\hline \adl@LTixhline
        \else\ifx\@tempa\hdashline \adl@LTixhline
        \else \penalty-\@lowpenalty \vskip-\arrayrulewidth
                \adl@hline\adl@discard{-\arrayrulewidth}%
                \adl@hline\adl@connect\arrayrulewidth
        \fi\fi \ifnum0=`{\fi}%
        \adl@LThdlrow \noalign{\penalty\@M}}
\def\adl@LTixhline{\penalty-\@medpenalty \vskip\doublerulesep
        \adl@hline\relax\doublerulesep \global\let\adl@LThdlrow\@empty}

%    \end{macrocode}
%
% \begin{macro}{\adl@LTstartpbox}
% \begin{macro}{\adl@LTendpbox}
% \begin{macro}{\adl@LTendmbox}
% \changes{v1.72-1}{2016/03/19}
% 	{Added to process footnotes in \texttt{m}-type columns appropriately.}
% 
% \SpecialIndex{\@startpbox}
% \SpecialIndex{\@@startpbox}
% \SpecialIndex{\@@endpbox}
% \SpecialIndex{\LT@startpbox}
% \SpecialIndex{\adl@act@@startpbox}
% \SpecialIndex{\adl@act@@endpbox}
%
% Macros for opening/closing |p|-boxes are fairly simple.  The macro
% |\adl@LTstartpbox{|\meta{w}|}| is |\let|-assigned to |\@@startpbox| by
% |\LT@make@row| to open a |p|-box of $w$ wide by our own
% |\adl@act@@startpbox| and performs a footnote related operation introduced
% by \textsf{longtable}, when \textsf{array} is not in use.  Note that if
% \textsf{array} is in use, a |p|-box is opened by codes embedded in
% |\@preamble| and its initialization is done by
% |\@startpbox|${}={}$|\LT@startpbox|, unnecessitating our own version of
% opening macros.
% 
% On the other hand, the closing macro |\adl@LTendpbox| for |p|(or
% |d|)-boxes is |\let|-equal to |\@endpbox| and |\@@endpbox| for the cases
% with\slash without \textsf{array}, and performs the footnote operations
% after doing our own ones by |\adl@act@@endpbox|.  Similarly,
% |\adl@LTendmbox| for |m|-boxes is |\let|-equal to |\adl@endmbox| and
% performs our own operations by |\adl@@endmbox| in which the originnal
% definition of |\adl@enmbox| is kept.
% \end{macro}\end{macro}\end{macro}
%
%    \begin{macrocode}
\def\adl@LTstartpbox#1{%
        \adl@act@@startpbox{#1}\let\@footnotetext\LT@p@ftntext}
\def\adl@LTendpbox{\adl@act@@endpbox \the\LT@p@ftn \global\LT@p@ftn{}}
\def\adl@LTendmbox{\adl@@endmbox \the\LT@p@ftn \global\LT@p@ftn{}}

%%^L
%    \end{macrocode}
% 
% 
% 
% \subsubsection{First Chunk}
% \label{sec:imp-lt-start}
% 
% \begin{macro}{\LT@start}
% \SpecialIndex{\LT@firsthead}
% \SpecialIndex{\adl@LTpagetotal}
%
% The macro |\LT@start| which puts (first) head and controls the page break
% of the first page is modified for the followings.
%
% \begin{itemize}
% \item
% After it inserts a vertical skip |\LTpre|, |\endgraf| is performed so that
% the skip contributes to |\pagetotal|\footnote{
%
% This modification is necessary for the original \textsf{longtable}, or it
% underestimates the room of the first page and leaves head and foot only.}.
% 
% \item
% When the |\box2| is |\vsplit| to get first item of the first chunk,
% |\vbadness| is saved into |\@tempcnta|, set to 10000 to avoid unnecessary
% |underfull| message\footnote{
%
% This is also necessary for the original version.},
%
% and restored from |\@tempcnta|.
% 
% \item
% The |\dimen| register |\adl@LTpagetotal| is set to |\pagetotal| to know
% the total height of the items preceding |longtable|.  Since the assignment 
% is performed after the inserted |\endgraf| and the intentional page break, 
% it should have real total height.
% 
% \item
% The box |\LT@firsthead| is put by |\copy| rather than |\box| because it is 
% referred to in the |\output| routine.
% \end{itemize}
%
% This macro does not have inactive counterpart because the modification
% shown above is desirable (first two) or not-harmful\footnote{
%
% Logically, at least.}
%
% (last two) to the original version.
% \end{macro}
%
%    \begin{macrocode}

%% Compatibility with longtable: first chunk

\def\LT@start{%
        \let\LT@start\endgraf
        \endgraf \penalty\z@ \vskip\LTpre \endgraf
        \dimen@\pagetotal
        \advance\dimen@ \ht\ifvoid\LT@firsthead\LT@head\else\LT@firsthead\fi
        \advance\dimen@ \dp\ifvoid\LT@firsthead\LT@head\else\LT@firsthead\fi
        \advance\dimen@ \ht\LT@foot
        \dimen@ii\vfuzz \@tempcnta\vbadness
        \vfuzz\maxdimen \vbadness\@M
        \setbox\tw@\copy\z@
        \setbox\tw@\vsplit\tw@ to \ht\@arstrutbox
        \setbox\tw@\vbox{\unvbox\tw@}%
        \vfuzz\dimen@ii \vbadness\@tempcnta
        \advance\dimen@\ht
                \ifdim\ht\@arstrutbox>\ht\tw@\@arstrutbox\else\tw@\fi
        \advance\dimen@\dp
                \ifdim\dp\@arstrutbox>\dp\tw@\@arstrutbox\else\tw@\fi
        \advance\dimen@ -\pagegoal
        \ifdim \dimen@>\z@\vfil\break \fi
        \global\adl@LTpagetotal\pagetotal
        \global\@colroom\@colht
        \ifvoid\LT@foot\else
                \advance\vsize-\ht\LT@foot
                \global\advance\@colroom-\ht\LT@foot
                \dimen@\pagegoal\advance\dimen@-\ht\LT@foot\pagegoal\dimen@
                \maxdepth\z@
        \fi
        \copy\ifvoid\LT@firsthead \LT@head \else \LT@firsthead \fi
        \output{\LT@output}}

%%^L
%    \end{macrocode}
% 
% 
% 
% \subsubsection{Output Routine}
% \label{sec:imp-lt-output}
% 
% \begin{macro}{\adl@org@LT@output}
% \begin{macro}{\LT@output}
% \SpecialIndex{\LT@firsthead}
% \SpecialIndex{\LT@head}
% \SpecialIndex{\LT@lastfoot}
% \SpecialIndex{\LT@foot}
%
% The output routine is the heart of the \textsf{longtable} compatible
% implementation.  The macro |\LT@output| which is set to |\output| by
% |\LT@start| is modified from its original (and thus inactive) version
% |\adl@org@LT@output| as follows.
%
% \begin{itemize}
% \item
% Three fractions of the original version to compile the final output image
% of the table portion into |\box255| or the main vertical list are modified
% to set the image into |\box255| unconditionally and to call
% |\adl@LTdraw|\meta{foot}\meta{tail} which is the real heart of the
% compatible implementation.  The argument \meta{foot} is |\LT@foot| or
% |\LT@lastfoot| according to the portion of the |longtable| to be output.
% The argument \meta{tail} is |\vss| if the last item is it which is not
% included in |\box255| yet, or |\@empty| otherwise.  Since |\adl@LTdraw|
% builds final output image drawing vertical (dash) lines in |\box255|, it is
% put to the main vertical list if the |longtable| portion is the last one.
% 
% \item
% Since the boxes |\LT@head|, |\LT@foot| and |\LT@lastfoot| are referred to in
% |\adl@LTdraw|, they are put by |\copy| rather than |\box|.
% \end{itemize}
% \end{macro}\end{macro}
%
%    \begin{macrocode}

%% Compatibility with longtable: output routine

\let\adl@org@LT@output\LT@output
\def\LT@output{%
        \ifnum\outputpenalty <-\@Mi
            \ifnum\outputpenalty > -\LT@end@pen
                \LT@err{floats and marginpars not allowed in a longtable}\@ehc
            \else
                \setbox\z@\vbox{\unvbox\@cclv}%
                \ifdim \ht\LT@lastfoot>\ht\LT@foot
                    \dimen@\pagegoal
                    \advance\dimen@-\ht\LT@lastfoot
                    \ifdim\dimen@<\ht\z@
                        \setbox\@cclv\vbox{\unvbox\z@\copy\LT@foot}%
                        \adl@LTdraw\LT@foot\vss
                        \@makecol
                        \@outputpage
                        \setbox\z@\vbox{\copy\LT@head}%
                    \fi
                \fi
                \global\@colroom\@colht
                \global\vsize\@colht
                \setbox\@cclv\vbox{\unvbox\z@
                    \copy\ifvoid\LT@lastfoot\LT@foot\else\LT@lastfoot\fi}%
                \adl@LTdraw\LT@lastfoot\@empty \box\@cclv
            \fi
        \else
            \setbox\@cclv\vbox{\unvbox\@cclv\copy\LT@foot}%
            \adl@LTdraw\LT@foot\vss
            \@makecol
            \@outputpage
            \global\vsize\@colroom
            \copy\LT@head
        \fi}

%    \end{macrocode}
%
% 
% \begin{macro}{\adl@LTdraw}
% \begin{macro}{\adl@LTinit}
% \begin{macro}{\adl@LTheadL}
% \begin{macro}{\adl@LTheadR}
% \begin{macro}{\adl@LTfootL}
% \begin{macro}{\adl@LTfootR}
% \SpecialIndex{\adl@LTpagetotal}
% \SpecialIndex{\ifadl@LTfirstpage}
%
% The macro |\adl@LTdraw|\meta{foot}\meta{tail} draws vertical
% (dash) lines onto the image in |\box255|.  First it measures the total
% height $H$ (|\adl@totalheight|) of |longtable| rows in |\box255| and the
% total height $H_b$ (|\@tempdima|) of its {\em body} which consists of the
% rows without the header and footer, as follows where $H_{255}$, $H_h$ and
% $H_t$ are the height plus depth of |\box255| and the effective header and
% footer of the page respectively.
%
% \begin{eqnarray*}
% T&=&\cases{\hbox{\cs{adl@LTpagetotal}}&if \cs{ifadl@LTfirstpage}\cr
%	0&otherwise}\\
% t&=&\cases{
%	\hbox{\cs{topskip} glue}&
%		if \texttt{longtable} is the first item of the page\cr
%	&	($\lnot(\hbox{\cs{ifadl@firstpage}}\land T{>}0)$)\cr
%	0&	otherwise}\\
% H&=&H_{255}-t-T\\
% H_b&=&H-H_h-H_t
% \end{eqnarray*}
%
% The hard part is to measure $t$ because it is not |\topskip| but that
% minus the first box of |\box255|.  Thus we do not measure $t$ but remove
% it from the box by the following tricky way.  First we copy |\box255|
% items into |\box0| adding a |\hrule| of 1\,sp high as its first item.
% Then |\box0| is |\vsplit| to 1\,sp setting |\splittopskip| to 0.
% Since the |\topskip| glue is the first item of |\box255| and the |\vsplit|
% discards it at the breakpoint, |\box0| must have all the items in
% |\box255| lead by 0 (|\splittopskip|) glue rather than |\topskip| glue.
% Thus the height of |\box0| is $H_{255}-t$.
% 
% \SpecialIndex{\LT@firsthead}
% \SpecialIndex{\LT@head}
% \SpecialIndex{\LT@lastfoot}
% \SpecialIndex{\LT@foot}
% \SpecialWeirdIndex{\\adl@LTth\LT@firsthead}
% \SpecialWeirdIndex{\\adl@LTth\LT@head}
% \SpecialWeirdIndex{\\adl@LTth\LT@lastfoot}
% \SpecialWeirdIndex{\\adl@LTth\LT@foot}
% \SpecialWeirdIndex{\\adl@rowsL\LT@firsthead}
% \SpecialWeirdIndex{\\adl@rowsL\LT@head}
% \SpecialWeirdIndex{\\adl@rowsL\LT@lastfoot}
% \SpecialWeirdIndex{\\adl@rowsL\LT@foot}
% \SpecialWeirdIndex{\\adl@rowsR\LT@firsthead}
% \SpecialWeirdIndex{\\adl@rowsR\LT@head}
% \SpecialWeirdIndex{\\adl@rowsR\LT@lastfoot}
% \SpecialWeirdIndex{\\adl@rowsR\LT@foot}
%
% Subtraction of $H_h$ and $H_t$ is done by the macro
% |\adl@LTinit{|\meta{hf}|}|\meta{box}, where \meta{hf} is |head| or |foot|
% and \meta{box} is one of |\LT@firsthead|, |\LT@head| and \meta{foot}
% (|\LT@lastfoot| or |\LT@foot|).  This macro also copies the contents of
% weirdly named structure such as |\\adl@rowsL\LT@head| into |\adl@LTheadL|
% and so on\footnote{
%
% Copying by \cs{edef} can be replaced by \cs{let} with many
% \cs{expandafter} but it is not comprehensible.}
%
% if \meta{box} is not void.  Otherwise, |\adl@LTheadL| etc. is kept to
% their initial value, |\@empty|.
%
% \SpecialIndex{\adl@makevlrL}
% \SpecialIndex{\adl@makevlrL}
% \SpecialIndex{\adl@rows}
% \SpecialIndex{\adl@rowsL}
% \SpecialIndex{\adl@rowsR}
% \SpecialIndex{\adl@vlrow}
% \SpecialIndex{\adl@vlrowL}
% \SpecialIndex{\adl@vlrowR}
% \SpecialIndex{\adl@addvl}
% \SpecialIndex{\adl@addvlL}
% \SpecialIndex{\adl@addvlR}
% \SpecialIndex{\adl@currentcolumn}
%
% Next, we make rows for vertical lines by |\adl@makevlrL|\slash|R| after
% extracting the leading part of $R^{L/R}$ corresponding to the {\em body}
% by the macro
% |\adl@LTsplit|\meta{$R^{L/R}$}\meta{$R^{L/R}_h$}\meta{$R^{L/R}_f$}, where
% $R^{L/R}_h$ and $R^{L/R}_f$ are |\adl@LTheadL| and so on.  Since the macro
% defines |\adl@rows| given to |\adl@makevlL|\slash|R| to the sequence of
% $R^{L/R}_h$, the extracted part of $R^{L/R}$ and $R^{L/R}_f$, the rows for
% vertical lines for all the rows including header and footer are build in
% |\adl@vlrowL| and |\adl@vlrowR| as in the ordinary case without
% |longtable|.
%
% \SpecialIndex{\LT@bchunk}
% \SpecialIndex{\LT@save@row}
% \SpecialIndex{\c@LT@chunks}
% \SpecialIndex{\LT@rows}
% \SpecialIndex{\adl@vl}
% \SpecialIndex{\adl@@vl}
% \SpecialIndex{\adl@drawvl}
%
% Then the rows are put into |\box0| by calling |\LT@bchunk| with
% |\adl@drawvl| (line drawing) and |\LT@save@row| (column widths
% adjustment), saving\slash restoring counters |\LT@rows| and |\c@LT@chunks|
% which |\LT@bchunk| globally updates.  Since we refer to potentially
% immature |\LT@save@row| here, some weird looking vertical lines could be
% drawn but the result after convergence should be correct.  Finally, the
% contents of |\box255| followed by the vertical lines in |\box0| are put
% back into |\box255| keeping its original depth and adding \meta{tail}
% (|\vss| or nothing) to its end.
% \end{macro}\end{macro}\end{macro}\end{macro}\end{macro}\end{macro}
%
%    \begin{macrocode}
\def\adl@LTdraw#1#2{%
        \@tempswatrue
        \ifadl@LTfirstpage\ifdim\adl@LTpagetotal>\z@\@tempswafalse \fi\fi
        \if@tempswa
                \setbox\z@\vbox{\hrule height1sp\unvcopy\@cclv}
                \splittopskip\z@
                \setbox\@ne\vsplit\z@ to1sp\relax
                \@tempdima\ht\z@
        \else   \@tempdima\ht\@cclv \fi
        \advance\@tempdima\dp\@cclv
        \adl@totalheight\@tempdima
        \let\adl@LTheadL\@empty \let\adl@LTheadR\@empty
        \let\adl@LTfootL\@empty \let\adl@LTfootR\@empty
        \ifadl@LTfirstpage
                \global\adl@LTfirstpagefalse
                \advance\@tempdima-\adl@LTpagetotal
                \adl@totalheight\@tempdima
                \ifvoid\LT@firsthead
                        \adl@LTinit{head}\LT@head
                \else   \adl@LTinit{head}\LT@firsthead
                \fi
        \else   \adl@LTinit{head}\LT@head \fi
        \ifvoid#1%
                \adl@LTinit{foot}\LT@foot
        \else   \adl@LTinit{foot}#1\fi
        \let\adl@vl\relax \def\adl@discard{\adl@connect}%
        \def\adl@vlrow{}\adl@currentcolumn\@ne
                \adl@LTsplit\adl@rowsL\adl@LTheadL\adl@LTfootL
                \let\adl@addvl\adl@addvlL
                \adl@makevlrL \let\adl@vlrowL\adl@vlrow
        \def\adl@vlrow{}\adl@currentcolumn\adl@columns
                \adl@LTsplit\adl@rowsR\adl@LTheadR\adl@LTfootR
                \let\adl@addvl\adl@addvlR
                \adl@makevlrR \let\adl@vlrowR\adl@vlrow
        \let\adl@vl\adl@@vl
        \@tempcnta\LT@rows
        \LT@bchunk \adl@drawvl
        \LT@save@row\cr \egroup \setbox\@ne\lastbox \unskip \egroup
        \global\advance\c@LT@chunks\m@ne
        \global\LT@rows\@tempcnta
        \@tempdima\dp\@cclv
        \setbox\@cclv\vbox{\unvbox\@cclv \box\z@ \vskip-\@tempdima
                \hrule\@width\z@\@height\z@\@depth\@tempdima#2}}
\def\adl@LTinit#1#2{\ifvoid#2\else
        \advance\@tempdima-\csname\string\adl@LTth\string#2\endcsname sp%
        \expandafter\edef\csname adl@LT#1L\endcsname{%
                \csname\string\adl@rowsL\string#2\endcsname}%
        \expandafter\edef\csname adl@LT#1R\endcsname{%
                \csname\string\adl@rowsR\string#2\endcsname}\fi}

%    \end{macrocode}
%
% \begin{macro}{\adl@LTsplit}
% \begin{macro}{\adl@LTxsplit}
% \begin{macro}{\adl@LTrowrelax}
% \begin{macro}{\adl@LTrowdiscard}
% \begin{macro}{\adl@LTysplit}
% \begin{macro}{\adl@LTisplit}
% \begin{macro}{\adl@LTiisplit}
% \begin{macro}{\adl@LTsplitend}
% \SpecialIndex{\adl@rows}
% \SpecialIndex{\adl@discard}
%
% The macro
% |\adl@LTsplit|\meta{$R^{L/R}$}\meta{$R^{L/R}_h$}\meta{$R^{L/R}_f$} moves
% leading elements in $R^{L/R}$ into $R'$ (|\adl@rows|) until total heights
% of the elements summed in $h$ (|\@tempdimb|) reaches to $H_b$
% (|\@tempdima|)\footnote{
%
% Although $h$ must become $H_b$ exactly in usual case, we stop the loop
% when $h\geq H_b$ to avoid accidental overrun in unusual cases.}
%
% by a straightforward loop with the macros |\adl@LTisplit| to fetch the
% $i$-th element and |\adl@LTiisplit| to get $h_i$.  Before moving, however, 
% we have to remove discardable item(s)\footnote{
%
% Must be only one but the implementation allows two or more.}
%
% from the top of $R^{L/R}$.  Since an element for a discardable item is
% {\it disconnect} (|\relax|) or {\it discard} (|\adl@discard|), we check
% the first part of the element by |\ifx|-comparison with |\adl@LTrowrelax|
% and |\adl@LTrowdiscard| whose bodies are |\relax| and |\adl@discard| if the
% |longtable| portion does not have a header ($R^{L/R}_h$ is |\@empty|).
% Otherwise, the discardable item was not discarded because the first
% item of the page is not it but the header.
%
% Note that since moving from $R^{L/R}$ to $R'$ is done by |\edef| and
% |\adl@discard| is |\def|-ined as |\adl@connect| in |\adl@LTdraw|,
% non-discarded {\it discard} transforms into {\it connect} in $R'$.  Also
% note that since the remaining part of $R^{L/R}$ is |\def|-ined as the body
% of |\@tempb| which is globally |\let|-assigned to $R^{L/R}$ again,
% |\adl@discard| survives in the new $R^{L/R}$.
% \end{macro}\end{macro}\end{macro}\end{macro}\end{macro}
% \end{macro}\end{macro}\end{macro}
%
%    \begin{macrocode}
\def\adl@LTsplit#1#2#3{\def\adl@rows{}\@tempdimb\z@
        \expandafter\adl@LTxsplit#1\@nil;%
        \edef\adl@rows{#2\adl@rows#3}%
        \global\let#1\@tempb}
\def\adl@LTxsplit#1;{\def\@tempa{#1}%
        \ifx\@tempa\@nnil \def\@tempb{}\let\next\relax
        \else\ifx\adl@LTheadL\@empty \def\next{\adl@LTysplit#1}%
        \else \def\next{\adl@LTisplit#1;}\fi \fi
        \next}
\def\adl@LTrowrelax{\relax}
\def\adl@LTrowdiscard{\adl@discard}
\def\adl@LTysplit(#1/#2){\def\@tempa{#1}%
        \ifx\@tempa\adl@LTrowrelax \let\next\adl@LTxsplit
        \else\ifx\@tempa\adl@LTrowdiscard \let\next\adl@LTxsplit
        \else \def\next{\adl@LTisplit(#1/#2);}\fi \fi
        \next}
\def\adl@LTisplit#1;{\def\@tempa{#1}%
        \ifx\@tempa\@nnil \def\@tempb{}\let\next\relax
        \else\ifdim\@tempdimb<\@tempdima
                \adl@LTiisplit#1\let\next\adl@LTisplit
        \else   \def\next{\adl@LTsplitend#1;}\fi \fi
        \next}
\def\adl@LTiisplit(#1/#2){\edef\adl@rows{\adl@rows(#1/#2);}%
        \advance\@tempdimb#2sp}
\def\adl@LTsplitend#1;\@nil;{\def\@tempb{#1;}}
\fi

%%^L
%    \end{macrocode}
% 
% 
%
% \subsection{Compatibility with \textsf{colortbl}}
% \label{sec:imp-colortbl}
% \changes{v1.7}{2003/05/21}
%	{\S\ref{sec:imp-colortbl} is added.}
% 
% The implementation to make \textsf{arydshln} compatible with
% \textsf{colortbl} consists of the following three (almost
% independent) issues.
% 
% \begin{description}
% \item[Cell coloring]
% is the easiest part because it does not affect dash line drawing.  Another
% reason of the easiness is that \textsf{colortbl} packs each cell in a box
% to measure its height for painting in the modified version of |\@classz|.
% Thus we do not need to code |\@classz| for both of \textsf{colortbl} and
% \textsf{arydshln}, but may sneak our own height\slash depth measurement
% into |\@classz| of \textsf{colortbl}.  Almost everything we have to pay
% attention to is the compatibility of the initialization and finalization
% of \textsf{colortbl} and \textsf{arydshln}.
% 
% \item[Horizontal line coloring]
% is relatively easy because it is almost enough to insert coloring macro
% |\CT@arc@| before the line drawing.  A little bit complicated part is the
% gap coloring which is done by drawing a solid line of gap color before
% dash line is drawn.
%
% \item[Vertical line coloring]
% is the hardest part but almost everything is done in previous sections to
% attach dash\slash gap color to each vertical line segment $e^i_j$ in the
% list $C_i^L$ and $C_i^R$ of the $i$-th row information $r_i$.  What we do
% here is to fix the bugs of |\arrayrulecolor| and |\doublerulesepcolor| in
% \textsf{colortbl} implementation and to add |\dashgapcolor|.  If you put
% |\arrayrulecolor| in |>{...}| construct to specify the color
% of the vertical lines following the construct as the manual of
% \textsf{colortbl} says, you will have an error message ``Misplaced
% |\noalign|'' because the macro is expanded with |\noalign| in a column
% body.  Even if you somehow remove |\noalign| to avoid the error, you will
% have a mysterious line coloring as follows:
%
% \begin{itemize}
% \item
% If you have |\arrayrulecolor| before the |\array|\slash|\tabular| starts,
% |\arrayrule#color| in the preamble has no effect to vertical lines but
% decides the color of horizontal lines except for those at the top of the
% environment.  Additional |\arrayrulecolor| at the beginning of a row has
% no effect to vertical lines (as expected) but decides horizontal lines
% following it (also as expected).  The effect of |\doublerulesepcolor| is 
% same as |\arrayrulecolor|.
% 
% \item
% Otherwise, i.e. without |\arrayrulecolor| outside the environment,
% |\arrayrule#color| in the preamble decides the color of vertical and
% horizontal lines except for verticals preceding columns in the first row
% and horizontals at the top of the environment.  Additional
% |\arrayrulecolor| at the beginning of a row decides all the vertical and
% horizontal lines following it.  On the other hand, |\doublerulesepcolor|
% acts as if |\doublerulesepcolor{white}| is done outside the environment.
% \end{itemize}
%
% The reason of the mysterious behavior is as follows.  An
% |\arrayrulecolor|, which globally |\def|-ines a macro |\CT@arc@| with a
% body containing |\color|, in the preamble is not expanded nor evaluated in
% the preamble construction phase but done when the first (and succeeding)
% row is build.  On the other hand, |\CT@arc@| attached to vertical line
% drawing is expanded in the preamble construction phase.  Thus if
% |\CT@arc@| has been defined before the environment starts, vertical lines
% are colored following the outside definition.  Otherwise, since |\CT@arc@|
% is |\let|-equal to |\relax|, it remains unchanged in the preamble
% construction phase and expanded when each row is build referring to its
% definition that |\arrayrulecolor| modifies in the row building phase.
% Since the macro |\CT@drsc@| defined by |\doublerulesepcolor| is examined
% if it is |\relax| or not in the preamble construction phase,
% |\doublerulesepcolor| in the preamble has no effect regardless the
% existence of the outside definition.
% 
% Thus we have to expand and evaluate |\arrayrulecolor| and
% |\doublerulecolor| in the preamble construction phase to define |\CT@arc@| 
% and |\CT@drsc@|.  We also have to initialize |\CT@arc@| as an expandable
% but non-operative token (e.g. a macro with a body of |\relax| as we do) to 
% make it is expanded in the preamble construction phase rather than the row
% building.
% \end{description}
% 
% 
% 
% \subsubsection{Initialization, Cell Coloring and Finalization}
% \label{sec:impl-colortbl-init}
% 
% \begin{macro}{\CT@arc@}
% \begin{macro}{\adl@dashgapcolor}
% First of all, we initialize the macro |\CT@arc@|, which will be
% |\def|-ined as |\color| to specify the color of solid lines and dash
% segments by |\arrayrulecolor|, with a body of |\relax| because it will be
% referred to by the vertical line drawing process even if \textsf{colortbl} is 
% not in use.  We also initialize the macro |\adl@dashgapcolor| for the
% color of gaps of dash lines similarly.  Note that these macros are not
% |\let|-equal to |\relax| but have bodies of |\relax| so that they are
% replaced with |\relax| in the preamble construction phase rather than
% surviving with their own name.
% \end{macro}\end{macro}
%
%    \begin{macrocode}

%% Compatibility with colortbl

\def\CT@arc@{\relax}
\def\adl@dashgapcolor{\relax}
%    \end{macrocode}
%
% \SpecialIndex{\@ifpackageloaded}
% Next we examine if \textsf{colortbl} is in use by |\@ifpackageloaded|, and 
% skip everything if not, or we have some errors especially when
% \textsf{array} is not in use.
%
%    \begin{macrocode}
\@ifpackageloaded{colortbl}\@tempswatrue\@tempswafalse
\if@tempswa
%    \end{macrocode}
%
% \begin{macro}{\adl@org@inactivate}
% \begin{macro}{\adl@org@activate}
% \begin{macro}{\adl@inactivate}
% \begin{macro}{\adl@activate}
% \begin{macro}{\CT@setup}
% \begin{macro}{\@endpbox}
% Then we redefine |\adl@inactivate| and |\adl@activate| referring their
% original version |\adl@org@inactivate| and |\adl@org@activate| so that 
% they make |\CT@setup| |\let|-equal to its original version |\adl@CT@setup| 
% if |\ADLinactivate| is in effect, or to our own version
% |\adl@act@CT@setup| which will be defined soon.  New |\adl@activate| also
% {\em inactivates} \hbox{|\@endpbox|} because our own one for column
% height\slash
% depth measuremnt is inappropriate with \textsf{colortbl} as explained soon.
% \end{macro}\end{macro}\end{macro}\end{macro}\end{macro}\end{macro}
%
%    \begin{macrocode}
\let\adl@org@inactivate\adl@inactivate
\let\adl@org@activate\adl@activate
\def\adl@inactivate{\adl@org@inactivate \let\CT@setup\adl@CT@setup}
\def\adl@activate{\adl@org@activate \let\CT@setup\adl@act@CT@setup
        \let\@endpbox\adl@org@endpbox}

%    \end{macrocode}
%
% \begin{macro}{\adl@CT@setup}
% \begin{macro}{\CT@setup}
% \begin{macro}{\adl@act@CT@setup}
% \SpecialIndex{\adl@@colhtdp}
% Cell coloring is done by |\@classz| preamble of \textsf{colortbl} in which
% a column is packed in |\box0|.  On the other hand, our own |\@classz| one
% with \textsf{array} packs the column in |\adl@box| so that we measure its
% height
% and depth.  Thus we have choices; to insert height\slash depth measurement
% into \textsf{colrotbl}'s version; or to insert coloring into our own version.
% Since the code of height\slash depth measurement is much simpler than the
% coloring, we choose the first way.  Thus the macro |\adl@act@CT@setup|,
% which is |\let|-equal to |\CT@setup| and is invoked from |\@classz|
% preamble after the column is packed into |\box0|, measures the height and
% depth of |\box0| and sets |\adl@height| and/or |\adl@depth| to them if
% they break the records as |\adl@@colhtdp| does with |\adl@box|, after it
% invokes its original version |\adl@CT@setup|.  Note that we compare
% |\adl@height| with the height of |\box0| plus |\minrowclearance| because
% it is the real height.  Also note that we could insert the measurement
% code into the modified version of \textsf{colortbls}'s |\@classz| placing
% it just before the |\box0| is put where |\ht0| plus |\minrowclearance| is
% caluculated, but did not because the author wished to make it clear that
% |\@classz| is modified only for the bug fix of |\arrayrulecolor| and
% |\doublerulesepcolor| (and to introduce |\dashgapcolor|).
% \end{macro}\end{macro}\end{macro}
%
%    \begin{macrocode}
\let\adl@CT@setup\CT@setup
\def\CT@setup{\adl@CT@setup
        \@tempdima\ht\z@ \advance\@tempdima\minrowclearance
        \ifdim\adl@height<\@tempdima \global\adl@height\@tempdima \fi
        \ifdim\adl@depth<\dp\z@ \global\adl@depth\dp\z@\fi}
\let\adl@act@CT@setup\CT@setup

%    \end{macrocode}
%
% \begin{macro}{\adl@activatepbox}
% \SpecialIndex{\CT@cell@color}
% \SpecialIndex{\CT@row@color}
% \SpecialIndex{\CT@column@color}
% \SpecialIndex{\CT@do@color}
% \SpecialIndex{\multicolumn}
% Another job for cell coloring is to make |\CT@|$x$|@color|
% ($x\in\{\texttt{cell},\texttt{column},\texttt{do}\}$)
% |\let|-equal to |\relax| before the body of |\multicolumn| is put so that
% the |\columncolor| in the environment preamble does not affect the
% |\span|-ned column.  Note that resetting |\CT@cell@color| will be
% unnecessary (but safe) because it is always reset after its invocation.
% Also note that resetting |\CT@row@color| in \textsf{colortbl}'s
% |\multicolumn| is a buggy feature because it should be effective, and thus 
% we remove it.  Although we have our own |\multicolumn| for dash lines, we
% keep it unchanged.  Instead we redefine |\adl@activatepbox|, which is
% usually |\relax| with \textsf{array}, to do the color resetting to
% minimize recoding.
% \end{macro}
%
%    \begin{macrocode}
\def\adl@activatepbox{\let\CT@cell@color\relax
        \let\CT@column@color\relax
        \let\CT@do@color\relax}

%    \end{macrocode}
%
% \begin{macro}{\adl@CT@start}
% \begin{macro}{\CT@start}
% \begin{macro}{\adl@dashgapcolor@save}
% \begin{macro}{\adl@CT@end}
% \begin{macro}{\CT@end}
% \begin{macro}{\endarray}
% \begin{macro}{\endArray}
% Yet another job is the save\slash restore of color information at the
% beginning and end of the environment.  Since this is done by |\CT@start| and
% |\CT@end|, we modify them to save\slash restore |\adl@dashgapcolor|
% to/from |\adl@dashgapcolor@save| referring their original version
% |\adl@CT@start| and |\adl@CT@end|.  We also modify our own |\endarray| and 
% its shorthand active version |\endArray| so that |\CT@end| is invoked at
% the end of environment.  Note that we may not modify |\endtabular| because 
% it refers |\endarray|.  Also note that |\CT@start| is invoked from
% |\@tabarray| which we keep unchanged.
% \end{macro}\end{macro}\end{macro}\end{macro}\end{macro}\end{macro}\end{macro}
%
%    \begin{macrocode}
\let\adl@CT@start\CT@start
\def\CT@start{\adl@CT@start \let\adl@dashgapcolor@save\adl@dashgapcolor}
\let\adl@CT@end\CT@end
\def\CT@end{\adl@CT@end \global\let\adl@dashgapcolor\adl@dashgapcolor@save}
\def\endarray{\adl@endarray \egroup \adl@arrayrestore \CT@end \egroup}
\ifx\adl@notdefinable\undefined \let\endArray\endarray \fi

%    \end{macrocode}
% 
% \subsubsection{Horizontal Line Coloring}
% \label{sec:impl-colortbl-hline}
% 
% \begin{macro}{\hline}
% \begin{macro}{\adl@inactivehdl}
% \begin{macro}{\adl@ixhline}
% \SpecialIndex{\CT@arc@}
% \SpecialIndex{\CT@drsc@}
% \SpecialIndex{\cline}
% To color |\hline| and inactivated |\hdashline|, we modify our own |\hline|
% and |\adl@inactivehdl| inserting the line coloring macro |\CT@arc@| before
% drawing by |\hrule| and pushing the coloring\slash drawing into a group.
% We also modify |\adl@ixhline| to draw a colored horizontal rule of
% |\doublerulesep| wide with the color defined in |\CT@drsc@| if it is not
% |\relax|, rather than to insert a vertical skip.  Note that the |\cline|
% coloring is done by \textsf{colortbl}'s |\cline| renamed as
% |\adl@org@cline| and invoked from our own one.
% \end{macro}\end{macro}\end{macro}
%
%    \begin{macrocode}
\def\hline{\noalign{\ifnum0=`}\fi
        \ifadl@zwhrule \vskip-\arrayrulewidth
        \else \adl@hline\adl@connect\arrayrulewidth \fi
        {\CT@arc@ \hrule\@height\arrayrulewidth}%
        \global\adl@finaldepth\z@
        \futurelet\@tempa\adl@xhline}
\def\adl@inactivehdl[#1/#2]{\ifadl@zwhrule \vskip-\arrayrulewidth \fi
        {\CT@arc@ \hrule\@height\arrayrulewidth}%
        \futurelet\@tempa\adl@xhline}
\def\adl@ixhline{{\ifx\CT@drsc@\relax \vskip \else
        \CT@drsc@\hrule\@height \fi \doublerulesep}%
        \adl@hline\relax\doublerulesep}
%    \end{macrocode}
%
% \begin{macro}{\adl@ihdashline}
% \begin{macro}{\adl@act@ihdashline}
% \begin{macro}{\adl@cdline}
% \begin{macro}{\adl@act@cdline}
% To draw a horizontal dash line with colored dashes and also colored gaps,
% we drastically modified |\adl@ihdashline| for |\hdashline| and
% |\adl@cdline| for |\cdashline|.  First, they invoke |\adl@hclinesetup|
% that makes the prefix of a |\multispan|-ned row from the first to last
% columns for |\hdashline| or given columns for |\cdashline|.  Then the line 
% is drawn by the modified version of |\adl@hcline|.  We have to declare
% these macros are active ones again.
% \end{macro}\end{macro}\end{macro}\end{macro}
%
%    \begin{macrocode}
\def\adl@ihdashline[#1/#2]{\adl@hclinesetup\@ne\adl@columns
        \adl@hcline\z@[#1/#2]%
        \noalign{\ifnum0=`}\fi
        \futurelet\@tempa\adl@xhline}
\let\adl@act@ihdashline\adl@ihdashline
\def\adl@cdline[#1-#2]{\ifadl@zwhrule \vskip-\arrayrulewidth \fi
        \adl@hclinesetup{#1}{#2}%
        \adl@hcline{-\arrayrulewidth}}
\let\adl@act@cdline\adl@cdline
%    \end{macrocode}
%
% \begin{macro}{\adl@hclinesetup}
% \begin{macro}{\adl@cdlinea}
% \begin{macro}{\adl@cdlineb}
% The macro |\adl@hclinesetup|\meta{f}\meta{t} makes the prefix of a
% |\multispan|-ned row from the column $f$ to $t$ and |\global|-ly defines
% it as |\@gtempa|.  This is done by a code very similar to original
% |\adl@cdline| (and thus \LaTeX-2.09's |\cline|) but the invocation of
% |\adl@hcline| is removed from |\adl@cdliena| and |\adl@cdlineb|, one of
% which is |\@gtempa|.
% \end{macro}\end{macro}\end{macro}
%
%    \begin{macrocode}
\def\adl@hclinesetup#1#2{\global\adl@cla#1\relax
        \global\advance\adl@cla\m@ne
        \ifnum\adl@cla>\z@ \global\let\@gtempa\adl@cdlinea
        \else              \global\let\@gtempa\adl@cdlineb\fi
        \global\adl@clb#2\relax
        \global\advance\adl@clb-\adl@cla \ifnum0=`{\fi}}
\def\adl@cdlinea{\multispan\adl@cla &\multispan\adl@clb \unskip}
\def\adl@cdlineb{\multispan\adl@clb \unskip}
%    \end{macrocode}
%
% \begin{macro}{\adl@hcline}
% \begin{macro}{\adl@paintdashgap}
% \SpecialIndex{\adl@draw}
% The modified version of |\adl@hcline|\meta{w}|[|\meta{d}|/|\meta{g}|]|
% draws a colored horizontal dash line of dash size $d$ and gap size $g$ and 
% insert vertical skip of $w$.  First it |\span|-s columns by |\@gtempa| and 
% checks if the body of |\adl@dashgapcolor| is something other than
% |\relax|.  If so, i.e.\ it has |\color|,
% |\adl@paintdashgap| is invoked to
% draw a horizontal rule of |\color| by |\leaders| as the background
% of the dash line, to insert |\nobreak| (for \textsf{longtable}) and a
% negative space for canceling the width of the
% rule, and to |\span| the columns again.  Then |\adl@hcline| draws the
% colored dash line, over the background if the gaps are colored, by
% inserting |\CT@arc@| before the invocation of |\adl@draw|.
% \end{macro}\end{macro}
%
%    \begin{macrocode}
\def\adl@hcline#1[#2/#3]{\@gtempa
        \ifx\adl@dashgapcolor\adl@nocolor \else \adl@paintdashgap \fi
        {\@tempdima#2\relax \@tempdimb#3\relax
                \CT@arc@ \adl@draw\adl@vrule\hskip\hbox}\cr
        \noalign{\global\adl@finaldepth\z@ \ifdim#1=\z@\else
                \ifadl@zwhrule\else \vskip#1\fi\fi}}
\def\adl@paintdashgap{{\adl@dashgapcolor
        \leaders\hrule\@height\arrayrulewidth\hfill}\cr
        \noalign{\penalty\@M \vskip-\arrayrulewidth}\@gtempa}

%    \end{macrocode}
% 
% \subsubsection{Vertical Line Coloring}
% \label{sec:impl-colortbl-vline}
% 
% \begin{macro}{\arrayrulecolor}
% \begin{macro}{\CT@arc@}
% \begin{macro}{\doublerulesepcolor}
% \begin{macro}{\CT@drsc@}
% \begin{macro}{\dashgapcolor}
% \begin{macro}{\adl@dashgapcolor}
% \begin{macro}{\adl@defcolor}
% \begin{macro}{\adl@idefcolor}
% \begin{macro}{\adl@noalign}
% \begin{macro}{\nodashgapcolor}
% \SpecialIndex{\@array}
% A bug of \textsf{colortbl}'s |\arrayrulecolor| and |\doublerulesepcolor|
% is that they are defined like;
%
% \begin{quote}
% |\ifdim\baselineskip=\z@ \noalign \fi{\gdef\CT@arc@{\color...}}|
% \end{quote}
%
% This aims to do |\noalign{\gdef...}| in \textsf{array}\slash
% \textsf{tabular} and do |{\gdef...}| outside but has two problems:  First, 
% if they are in |>{...}| construct, they are expanded with |\noalign|
% inappropriately when the argument of |>| is expanded.  Second, they may
% appear at a place where |\baselineskip| is 0 but is outside of
% \textsf{array}\slash\textsf{tabular} and will cause the misplaced
% |\noalign| error.  To solve the second problem, we introduced
% |\adl@noalign| which is set to |\noalign| in the environment by our own
% |\@array|, and |\relax| outside.  We also introduced
% |\adl@defcolor|\meta{cs}\meta{opt} for the common job to define \meta{cs}
% as |\color| with \meta{opt}, in |\noalign| if necessary, by
% |\adl@idefcolor|.  Thus |\arrayrulecolor| and |\doublerulesepcolor| are
% modified to define |\CT@arc@| and |\CT@drsc@| using |\adl@defcolor|, and
% our own |\dashgapcolor| is defined 
% similarly to define |\adl@dashgapcolor|.  Another macro |\nodashgapcolor|
% to nullify |\dashgapcolor| is also defined with |\adl@noalign| to reset
% |\adl@dashgapcolor| to |\relax|.
% \end{macro}\end{macro}\end{macro}\end{macro}\end{macro}
% \end{macro}\end{macro}\end{macro}\end{macro}\end{macro}
%
%    \begin{macrocode}
\def\arrayrulecolor{\adl@defcolor\CT@arc@}
\def\doublerulesepcolor{\adl@defcolor\CT@drsc@}
\def\dashgapcolor{\adl@defcolor\adl@dashgapcolor}
\def\adl@defcolor#1#2#{\adl@idefcolor{#1}{#2}}
\def\adl@idefcolor#1#2#3{\adl@noalign{\gdef#1{\color#2{#3}}}}
\let\adl@noalign\relax
\def\nodashgapcolor{\adl@noalign{\gdef\adl@dashgapcolor{\relax}}}

%    \end{macrocode}
%
% \begin{macro}{\@classz}
% \begin{macro}{\adl@act@classz}
% \begin{macro}{\adl@org@classz}
% The tougher bug of \textsf{colortbl} is the expansion timing of
% |\arrayrulecolor| and |\dobule#rule#sep#color| in a |>|-argument.  We
% have to modify |\@classz| to extract them from
% |\toks|\break|\@tempcnta| as its
% original version does for |\columncolor|.  Thus we inserted the invocation 
% of |\adl@extract@arc| for |\arrayrulecolor|, |\adl@extract@drsc| for
% |\doublerulesep#color|, and |\adl@extract@dgc| for |\dashgapcolor| just
% after the invocation of |\CT@extract|.  Note that the other part of
% |\@classz| is not modified logically, but done for author's preference of
% indentation.  Also note that both |\adl@act@classz| and |\adl@org@classz|
% are |\let|-equal to the modified |\@classz| because we have to be bug free 
% even if |\ADLinactive| is in effect.
% \end{macro}\end{macro}\end{macro}
%
%    \begin{macrocode}
\def\@classz{\@classx
        \@tempcnta\count@ \prepnext@tok
        \expandafter\CT@extract\the\toks\@tempcnta\columncolor!\@nil
        \expandafter\adl@extract@arc\the\toks\@tempcnta\arrayrulecolor!\@nil
        \expandafter\adl@extract@drsc
                \the\toks\@tempcnta\doublerulesepcolor!\@nil
        \expandafter\adl@extract@dgc\the\toks\@tempcnta\dashgapcolor!\@nil
        \@addtopreamble{%
                \setbox\z@\hbox\bgroup\bgroup
                \ifcase \@chnum
                        \hskip\stretch{.5}\kern\z@
                        \d@llarbegin
                        \insert@column
                        \d@llarend\hskip\stretch{.5}%
                \or \d@llarbegin \insert@column \d@llarend \hfill
                \or \hfill \kern\z@ \d@llarbegin \insert@column \d@llarend
                \or  $\vcenter
                        \@startpbox{\@nextchar}\insert@column \@endpbox $%
                \or  \vtop \@startpbox{\@nextchar}\insert@column \@endpbox
                \or  \vbox \@startpbox{\@nextchar}\insert@column \@endpbox
                \fi
                \egroup\egroup
                \begingroup
                        \CT@setup
                        \CT@column@color
                        \CT@row@color
                        \CT@cell@color
                        \CT@do@color
                \endgroup
                \@tempdima\ht\z@
                \advance\@tempdima\minrowclearance
                \vrule\@height\@tempdima\@width\z@
                \unhbox\z@}%
        \prepnext@tok}
\let\adl@act@classz\@classz
\let\adl@org@classz\@classz

%    \end{macrocode}
%
% \begin{macro}{\adl@def@extract}
% \begin{macro}{\adl@extract@arc}
% \begin{macro}{\adl@extract@arc@b}
% \begin{macro}{\CT@arc@}
% \begin{macro}{\adl@extract@drsc}
% \begin{macro}{\adl@extract@drsc@b}
% \begin{macro}{\CT@drsc@}
% \begin{macro}{\adl@extract@dgc}
% \begin{macro}{\adl@extract@dgc@b}
% \begin{macro}{\adl@dashgapcolor}
% The definitions of |\adl@extract@|$x$ ($x\in\{\texttt{arc}, \texttt{drsc}, 
% \texttt{dgc}\}$) are quite similar to each other.  For example
% |\adl@extract@arc| is defined as follows.
%
%\begin{verbatim}
%\def\adl@extract@arc#1\arrayrulecolor#2#3\@nil{%
%        \if!#2\toks\@tempcnta{#1}\let\@tempa\relax%
%        \else\if[#2%]
%                \def\@tempa{\adl@extract@arc@b{#1}#3\@nil}%
%        \else   \def\CT@arc@{\color{#2}}%
%                \def\@tempa{\adl@extract@arc#1#3\@nil}%
%        \fi\fi \@tempa}
%\def\adl@extract@arc@b#1#2]#3{%
%        \def\CT@arc@{\color[#2]{#3}}%
%        \adl@extract@arc#1}
%\end{verbatim}
%
% This code extracts {\em all the} occurrences of
% |\arrayrulecolor[|\meta{m}|]|\Meta{c} from the token register and
% |\def|-ines |\CT@arc@| as |\color[|\meta{m}|]|\Meta{c}.  Note that
% |\CT@extract| does a similar job for |\columncolor| but it mistakingly
% ignores the possibility that the token register has two or more
% |\columncolor|\footnote{
%
% Fixing this bug is not our business.}.
%
% Anyway, if we copy the code above and replace
% `|@arc|' with `|@drsc|', |\arrayrulecolor| with |\doublerulesepcolor|, and 
% |\CT@arc@| with \hbox{|\CT@drsc@|}, we will have |\adl@extract@drsc|(|@b|)
% for
% |\doublerulesepcolor|.  The code for |\adl@extract@dgc|(|@b|) will be also 
% obtained similarly.  However, having three relatives for a almost common
% job is too awful.  Thus we introduce;
%
% \begin{quote}
% |\adl@def@extract|\meta{key}\meta{umac}\meta{cmac}
% \end{quote}
%
% to define the macros |\adl@extract@|\textit{key} and
% |\adl@extract@|\textit{key}|@b| for the user interface macro \meta{umac}
% in which a color macro \meta{cmac} is defined with |\color|.  For example,
% we will obtain |\adl@extract@arc|(|@b|) shown above by;
%
% \begin{quote}
% |\adl@def@extract{arc}\arrayrulecolor\CT@arc@|
% \end{quote}
%
% Note that |\color| is made |\relax| in the preamble construction phase
% by \textsf{colortbl}'s \break |\@mkpream| and regain its proper meaning
% after the phase.
% \end{macro}\end{macro}\end{macro}\end{macro}
% \end{macro}\end{macro}\end{macro}
% \end{macro}\end{macro}\end{macro}
%
%    \begin{macrocode}
\def\adl@def@extract#1#2#3{%
        \expandafter\def\csname adl@extract@#1\endcsname##1#2##2##3\@nil{%
                \if!##2\toks\@tempcnta{##1}\let\@tempa\relax
                \else\if[##2%]
                        \def\@tempa{\@nameuse{adl@extract@#1@b}{##1}##3\@nil}%
                \else   \def#3{\color{##2}}%
                        \def\@tempa{\@nameuse{adl@extract@#1}##1##3\@nil}%
                \fi\fi \@tempa}
        \expandafter\def\csname adl@extract@#1@b\endcsname##1##2]##3{%
                \def#3{\color[##2]{##3}}%
                \@nameuse{adl@extract@#1}##1}}
\adl@def@extract{arc}\arrayrulecolor\CT@arc@
\adl@def@extract{drsc}\doublerulesepcolor\CT@drsc@
\adl@def@extract{dgc}\dashgapcolor\adl@dashgapcolor

%    \end{macrocode}
% 
% 
%
% \subsubsection{Compatibility with \textsf{longtable}}
% \label{sec:impl-colortbl-longtable}
% 
% \begin{macro}{\LT@hline}
% \begin{macro}{\adl@LTihdashline}
% \begin{macro}{\adl@LTinactivehdl}
% \begin{macro}{\adl@LTixhline}
% Yet another compatiblity issue is to cope with both \textsf{longtable} and 
% \textsf{colortbl}.  We redefine |\LT@hline| and |\LT@inactivehdl| in order 
% to put |\CT@arc@| before line drawing and to push them in a group.
% Modified |\adl@LTidashline| first
% invokes |\adl@hclinesetup| and open |\noalign| because it is closed by
% |\adl@hclinesetup|.  The contents of |\adl@LThdlrow| for
% |\adl@LTidashline| is simply |\adl@hcline| because it does |\multispan|
% now.  The macro |\adl@LTixhline| is modified to paint the |\doublerulesep| 
% gap by |\leaders\hrule| with color of |\CT@drsc@| if it is not |\relax|.
% \end{macro}\end{macro}\end{macro}\end{macro}
% 
%
%    \begin{macrocode}
\ifx\longtable\undefined\else
\def\LT@hline{\noalign{\ifnum0=`}\fi
        \gdef\adl@LThdlrow{\multispan{\LT@cols}\unskip{\CT@arc@
                \leaders\hrule\@height\arrayrulewidth\hfill}\cr}%
        \adl@LThdline}
\def\adl@LTihdashline[#1/#2]{\adl@hclinesetup\@ne\adl@columns
        \noalign{\ifnum0=`}\fi
        \gdef\adl@LThdlrow{\adl@hcline\z@[#1/#2]}%
        \adl@LThdline}
\def\adl@LTinactivehdl[#1/#2]{%
        \gdef\adl@LThdlrow{\multispan{\LT@cols}\unskip{\CT@arc@
                \leaders\hrule\@height\arrayrulewidth\hfill}\cr}%
        \adl@LThdline}
\def\adl@LTixhline{%
        \ifx\CT@drsc@\relax \gdef\adl@LThdlrow{\noalign{
                \penalty-\@medpenalty \vskip\doublerulesep}}
        \else \gdef\adl@LThdlrow{\noalign{\penalty\@M}%
                \multispan{\LT@cols}\unskip{\CT@drsc@
                \leaders\hrule\@height\doublerulesep\hfill}\cr}\fi
        \ifnum0=`{\fi}\adl@LThdlrow \noalign{\ifnum0=`}\fi
        \adl@hline\relax\doublerulesep \global\let\adl@LThdlrow\@empty}
\fi
\fi
%    \end{macrocode}
% 
% \iffalse
%</package>
% \fi
% 
% 
% 
% \IndexPrologue{\newpage\section*{Index}
% Italicized number refers to the page where the specification and usage of
% corresponding entry are described, while underlined is for the
% implementation of the entry.
%
% To find a control sequence, remove prefixes \cs{@}, \cs{adl@} and 
% \cs{ifadl@} from its name if it has one of them.}
% \Finale
% \newpage
% \def\EQ{=} \def\GT{>} \def\BAR{|} \def\NEQ{\neq}
% \PrintChanges
\endinput
